\section{Conclusions}

We conclude that the version control concept has great potential for distributed
private data storage, and for tracking a large collection of files across time
and space. This is especially true in situations where data locality is
important, such as where nodes have intermittent connectivity or differing
security requirements.

The cryptographic directed acyclic graph is an excellent data structure for
long-term data storage, because of the way it naturally de-duplicates identical
data and provides checksums to verify data integrity. It also lends itself to
sharding, because it can be divided by time or by subset of data, or both. The
key to using a DAG for larger files is to break them into smaller chunks by
content, so that identical pieces of files can naturally be de-duplicated as
well. But when storing a large number of small chunks, they should be
re-aggregated into packed files to use disk space efficiently.

The DMV prototype is not ready for production use yet, but we believe that it
has potential to be a useful and versatile data storage platform with uses in
long term-bulk data storage and collaborative work.
