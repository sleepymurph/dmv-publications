\chapter{Conclusion and Summary of Contributions}

\glsreset{DMV}

In this dissertation, we have examined the cryptographic \acrfull{DAG} as a data
structure for data storage, and the ways that it can be sliced to shard data
across nodes in a distributed system, according to what data is needed locally
at each location.

We have performed experiments to probe the scalability limits of existing
\gls{DAG}-based \acrlongpl{DVCS}. We have shown that the maximum size of file
that Git and Mercurial can store is limited by the amount of available memory in
the system. We conclude that this is because those systems calculate deltas of
files to de-duplicate data, and they load the entire file into memory in order
to do so.

We have also rediscovered the limits of the Unix filesystem for storing many
small files. We saw that writing files smaller than the filesystem block size
incurs storage overhead, that splitting files among too many subdirectories
takes \glspl{inode} that are needed to store files, and that jumping between
directories when writing files incurs write-speed penalties.

We have shown that any \gls{VCS} that stores objects as individual files on the
filesystem will encounter these filesystem limitations as they try to scale in
terms of number of files. A \gls{VCS} that also breaks files into chunks will
turn the problem of storing large files into the problem of storing many files,
again encountering these limitations. However, the limitations can be avoided by
aggregating objects into \glspl{packfile} as Bup does.

We have performed experiments on the rolling hash algorithm used for chunking,
and we have determined that adjusting the \gls{divisor} has the most direct
effect on chunk size. Larger \glspl{divisor} result in smaller chunks. And we
have shown that adjusting \gls{windowsize} has a lesser effect on chunk size,
but we reason that smaller \glspl{windowsize} will be able to find smaller
common chunks in the code.

And finally, we have described the idea, architecture, design, and
implementation of a distributed data storage system we call \gls{DMV} that
expands on the \glsdisp{DVCS}{distributed version control} concept to store
larger and more diverse data sets, with a high degree of control over data
locality, and an availability to write updates for any data held locally. Though
time constraints prevented us from implementing the network features we had
planned, the \gls{DMV} prototype has enough functionality to be experimented on
against existing \acrlongpl{DVCS} and to demonstrate the addition of new
\glspl{commit} to a partial \gls{DAG}.
