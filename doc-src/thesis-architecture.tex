\chapter{Architecture}

TODO: Working directories and object stores

TODO: Network decentralized and unstructured

TODO: Each store autonomous

\begin{figure}[h!]
  \caption{Distributed Version Control}
  \label{fig:dia_obj_db_and_wd}
  \centering
    \includegraphics[width=0.5\textwidth]{dia_obj_db_and_wd}
\end{figure}

TODO: Replication

TODO: Ad-hoc network topography
    - Decentralized and unstructured
    - Human scale
    - No need for complicated network membership schemes
    - Replica network topologies reflect the connections between their users and
    their workflows.
    - Small projects use hub and spoke, large projects hierarchy

TODO: Different workflow diagrams?

\begin{figure}[h]
  \caption{Stores in an ad-hoc network}
  \label{fig:dia_architecture}
  \centering
    \includegraphics[width=0.95\textwidth]{dia_architecture}
\end{figure}

\subsection{Important assumptions}

\begin{itemize}

  \item Contact between repositories is intermittent. Repositories may be on
    removable drives or mobile devices. Updates may require physical connection
    and reconnection. It is important to track the state of other repositories,
    so that the user can know what needs to be synchronized.

  \item Assume all actors are honest for now. No malicious components.

  \item However, components can and will fail. The system must discover and
    recover from errors (checksums, replication).

\end{itemize}
