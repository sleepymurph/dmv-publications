\chapter{Architecture}

Data will be distributed across several \newterm{stores}, which will each hold a
portion of the data set. Stores can be diverse, from mobile devices to storage
clusters. Distribution of the data will be determined by the end user or client
application according to available resources and the needs of the user.

The data store network will be decentralized and unstructured (Figure
\ref{fig:dia_architecture}). Stores can be connected to other stores in an ad-hoc
manner, with a user- or application-defined topography. The number of
connections will not be strictly limited, but extreme scalability of individual
connections is not a priority. We are aiming for one, to tens, to perhaps
hundreds of connections.

\begin{figure}[h]
  \caption{Stores in an ad-hoc network}
  \label{fig:dia_architecture}
  \centering
    \includegraphics[width=0.95\textwidth]{dia_architecture}
\end{figure}

Each store will be autonomous, and local applications will be able to access and
perform processing on any data that is stored locally. Data not currently
available can be listed and requested. The listing will give hints about the
estimated availability of each piece of data, with metrics including connection
uptime, latency, and bandwidth to stores that have copies of the data. This
metadata will naturally fall out of date between syncs with connected nodes. The
age of that data can be part of the hint.

We are envisioning a filesystem with an \lstinline{ls} command that lists these
hints as part of its output (in this example presented as a single estimated
time to retrieve):

\begin{lstlisting}[caption=Example ls output]
    -rw-r--r-- 1 user user  121306 Oct 21 18:28 local   filex
    -rw-r--r-- 1 user user   25475 Oct 21 17:52 100ms   filey
    -rw-r--r-- 1 user user   32031 Oct 21 17:52 20min   filez
    -rw-r--r-- 1 user user   74968 Oct 18 17:12 missing filexx
    -rw-r--r-- 1 user user   83977 Sep 22 21:23 unknown fileyy
\end{lstlisting}

A primary goal for the system is to detect storage errors and never lose data
inadvertently. Local data stores will detect corruption, and data loss will be
prevented by replication between stores. However, data can be explicitly removed
from the data set. Data can be removed from the local store, relying on remote
stores to keep replicas, though it will be up the end user/application to ensure
that there are enough copies of the data to ensure its durability. Number of
known replicas will be one of the metrics tracked by the system.
