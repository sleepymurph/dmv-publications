\documentclass[12pt,a4paper,two-side]{book}
% vim: set ts=2 sts=2 sw=2 :

\usepackage{todonotes}

\usepackage{graphicx}
\usepackage[utf8]{inputenc}

\usepackage{amsmath}    % Math
\usepackage[binary-units]{siunitx}    % SI units

% Use a blank space between paragraphs instead of an indent.
\usepackage[parfill]{parskip}

% Smart quotes
% Provides \enquote command
\usepackage [strict=true]{csquotes}
\MakeOuterQuote{"}  % Automatically treat quoted strings as quotes

% Source code listings
\usepackage{listings}
\lstset{basicstyle=\ttfamily\footnotesize,breaklines=false}

% Add bibliography, listings, and other pseudo-chapters to table of contents
\usepackage[
  nottoc  % skip the table of contents itself
]{tocbibind}
% Add list of listings to toc via tocbibind
\renewcommand{\lstlistoflistings}{\begingroup
\tocfile{\lstlistlistingname}{lol}
\endgroup}
% Add list of todos to toc via tocbibind
\renewcommand{\listoftodos}{\begingroup
\tocfile{Todo List}{tdo}
\endgroup}

\usepackage[
  backend=bibtex,
  backref=true, % add back-references from bib to citations
  urldate=long, % use date like "Apr. 7, 2017" instead of American "04/07/2017"
]{biblatex}
\addbibresource{research.bib}

% Provide \url command
\usepackage{url}

% Link URLs in the PDF, and link references within the PDF itself
%
% The hyperref docs recommend declaring it after the other \usepackage
% declarations, because it has to redefine several commands to work properly,
% and other later redefinitions might interfere.
\usepackage[
  hidelinks,    % Do not style links. I think this is classier.
  pdfusetitle,  % Use doc title metadata for PDF title metadata
  bookmarksnumbered,      % use section numbers in PDF index
]{hyperref}


% UiT official font: Open Sans
\usepackage[default]{opensans}

% Adjust caption spacing
\usepackage{caption}
\captionsetup[table]{skip=10pt}

% Nicer hlines for tables
\usepackage{booktabs}


% Custom commands:

% Newly defined terms
\newcommand{\newterm}{\textit}

% An enumerated list with less spacing between items
\newenvironment{tight_enumerate}{
\begin{enumerate}
  \setlength{\itemsep}{0pt}
  \setlength{\parskip}{0pt}
}{\end{enumerate}}

% to-write
\usepackage{xcolor}
\newcommand\towrite[1]{\textcolor{blue}{TODO: #1}}


% Current Git revision information
\input{vc_version.tex}


% Metadata

\title{Distributed Media Versioning \\
  \large Master's Thesis in Computer Science \\
        University of Tromsø}
\author{Michael J. Murphy}
\date{May 2017}

% Metadata for PDF
\hypersetup{
  pdftitle=Master's Thesis: Distributed Media Versioning,
}


\begin{document}



\frontmatter
\pagestyle{empty}


% Put title page in PDF index
\phantomsection
\currentpdfbookmark{Title Page}{title}

\maketitle


% Colophon on reverse of title page
\vspace*{\fill}
% Colophon

\clearpage
\vspace*{\fill}

Typeset by \LaTeX.

This document, \gls{DMV} source code, and other related materials are archived
in Munin, the University of Tromsø's open research archive: \\
\muninurl

Active source repositories for the \gls{DMV} project, this document, and other
related related materials can be found via the author's website: \\
\dmvurl

This document generated \GITAuthorDate\ from revision
\ifx\GITTags\empty
\GITAbrHash
\else
\GITAbrHash\ \GITTags
\fi
{} of the dmv-publications repository.

This work is licensed under the Creative Commons Attribution-NoDerivs 3.0 United
States License. To view a copy of this license, visit \\
\url{http://creativecommons.org/licenses/by-nd/3.0/us/} \\
or send a letter to \\
Creative Commons, PO Box 1866, Mountain View, CA 94042, USA.

\clearpage


\pagestyle{headings}
\chapter{Abstract}

\glsunsetall
\glsreset{DMV}

%\perottoinline{1. What's wrong with the world?}

It is still strangely difficult to backup and synchronize data. Cloud computing
solves the problem by centralizing everything and letting someone else handle
the backups. But what about situations with low connectivity or sensitive data?

%\perottoinline{2. Motivation (why bother?)}

For this, software developers have an interesting distributed, decentralized,
and partition-tolerant data storage system right at their fingertips:
\glsdisp{DVCS}{distributed version control}.

%\perottoinline{3. How we did better: idea, architecture, design, implementation.}

Inspired by \glsdisp{DVCS}{distributed version control}, we have researched and
developed a prototype for a scalable high-availability system called \gls{DMV}.
\gls{DMV} expands Git's data model (the \acrshort{DAG}) to allow files to be
broken into more digestible chunks via a \gls{rollinghash}. \gls{DMV} will allow
data to be sharded according to data locality needs, slicing the data set in
space (subset of data with full history), time (subset of history for full data
set), or both. \gls{DMV} \glspl{repository} will be able to read and to update
any subset of the data that they have locally, and then synchronize with other
\glspl{repository} in an ad-hoc network.

%\perottoinline{4. Experiments; results}

We have performed experiments to probe the scalability limits of existing
version control systems, specifically what happens as file sizes grow ever
larger or as the number of files grow. We found that processing files whole
limits maximum file size to what can fit in RAM, and that storing millions of
objects loose as files with hash-based names incurs disk space overhead and
write speed penalties. We have observed a system needing \SI{24}{\second} to
store a \SI{6.8}{\kib} files.

%\perottoinline{5. Lessons learned}



%\perottoinline{6. Main conclusions}

We conclude that the key to storing large files is the break them into many
small chunks, and that the key to storing many chunks is to aggregate them into
pack files. And, if one uses a \gls{rollinghash} to split the files by content
and \gls{contentaddressablestorage} to store them, common file chunks are
naturally de-duplicated and many versions of a file can be stored efficiently.
\perotto{Does DMV do this? Do results back it up?}

%


\glsresetall



% Put table of contents in the PDF index
\cleardoublepage
\phantomsection
\currentpdfbookmark{Contents}{contents}

\tableofcontents

\listoffigures
\listoftables

\renewcommand{\lstlistlistingname}{List of Code Listings}
\lstlistoflistings

\listoftodos



\mainmatter

\chapter{Introduction}

It is still surprisingly difficult to keep backups and synchronize data. Many of
us have several computers, perhaps a laptop, a phone, and a work computer, and
we would like to synchronize data between them. We want to keep a Word document
synchronized between home and work. We want to put new music on our phones, and
pull photos off of camera SD cards. We have backups on removable drives, but we
don't remember when it was that we last did a backup, or what is new since then.
We have these sets of files that tend to fragment themselves across our devices,
and we lose track of what is where.

Cloud computing offers to centralize and safeguard our data, keeping it all in
one place and taking care of the backups for us. Google Drive gives us a shared
document that many people can edit in real time. Spotify offers endless music
streams. Instagram lets us save and share photos. DropBox gives us a folder that
syncs. But many of these solutions are specialized applications for specific
media, which can limit their general usefulness; most require constant network
connectivity, making them ill-suited for intermittent or high-latency
connections; and all require entrusting your data to a third-party service,
which raises concerns about privacy and storage longevity. Why can't we simply
track the files we have across the devices we have?

\note{Cite "folder that syncs"}
% https://www.quora.com/Why-is-Dropbox-more-popular-than-other-programs-with-similar-functionality

Software programmers have an excellent system for backup and sync right at their
fingertips: \newterm{distributed version control systems (DVCSs)}, such as Git
and Mercurial. Version control systems track all changes to a collection of
files, allowing collaborators to work independently and then synchronize and
share their work. Additionally, in a distributed version control system, every
collaborator has a full copy of the project's history. That redundancy not only
allows collaborators to work offline, but it also functions as a backup. Linus
Torvalds, the creator of Linux and Git, once famously joked that he doesn't keep
backups, he simply publishes his work on the internet and lets others copy it
\cite{linus_no_backups}.

The major limitation distributed version control systems is that they are
designed to store program source code: plain text files in the range of tens of
kilobytes. They often have trouble with larger media files, becoming sluggish or
wasteful of disk space. Many a web-design team has come to regret checking their
graphical assets into version control. In addition, many will have trouble
scaling up as the number of files increases or the history grows increasingly
long.

What if we could generalize the distributed version control concept to store a
wide variety of file sizes, from kilobyte text files to multi-gigabyte videos?
In addition, what if we relaxed the assumption that every replica contain the
complete history, and allowed each replica to choose what subsets of the files
and the history to store, according to the replica's capacity and need? The
answer could be a new abstraction for tracking a data set and its history as a
cohesive whole, even when it is physically spread over many different nodes.


\section{CAP Theorem and the Importance of Availability}

Distributed systems are ruled by the \newterm{CAP-theorem} \cite{cap_origin},
which states that a system cannot be completely consistent (\newterm{C}),
available (\newterm{A}), and tolerant of network partitions (\newterm{P}) all at
the same time. One area must always suffer, and since network partitions are
always a possibility, a distributed system must make trade-offs between
availability and consistency.

\towrite{Traditional ACID}

\towrite{Modern NoSQL/BASE}

\towrite{New thinking (CAP years later)
   - partitions rare
   - partitions are just latency
   - "the partition decision": cancel and decrease availability, proceed and
   risk inconsistency
   - general problem of resolving conflicts is not solvable
   \cite{cap_years_later}
}

\towrite{DVCS is always available
    - Branches: No global concept of recent version
}


\section{Git and the DAG}

\towrite{Explain Git and DAG
    - Append only
    - Easy to sync
    - Trick is merging
}

\towrite{Diagram of Git DAG -- remember I have a git history visualizer script}

\chapter{Idea: Distributed Media Versioning (DMV)}

\newterm{Distributed Media Versioning (DMV)} is our new
low-level\perotto{explain why "low-level"} distributed
data storage platform. The core idea is relatively simple-- store data in a
Git-like DAG, but make the following changes:

\begin{tight_enumerate}

    \item{Store data at a finer granularity than the file}

    \item{Allow notes to store only a portion of the DAG as a whole}

\end{tight_enumerate}

Doing so allows a data set to be replicated or sharded across many nodes
according to the capacity of nodes and the needs of local users. The focus is on
data locality: tracking what data is where, presenting that information to the
user, and making it easy to transfer data to other nodes as desired. The
ultimate goal is to create a new abstraction, of \newterm{many devices, one data
item} in varying states of synchronization.


\paragraph{General storage}

DMV is a generalized storage platform that places no restriction on file types.
Its data model is a classic hierarchical filesystem, but with history.
Applications on each node can read and write to the files via the filesystem as
normal. DMV is dogmatic about the end-to-end argument \cite{endtoendargument},
that it cannot anticipate all the needs of end users and applications. So it
aims to be as general and neutral as possible, focusing on the core task of
storage and tracking, and providing a platform for other applications to build
on. DMV stores files as binary large objects (BLOBs), and it can handle
\perotto{vague. Handle "well"?} a wide variety of file sizes, from empty files
to files tens of gigabytes\perotto{artificial limit? Document/verify} in size.
It also stores a wide range of file quantities, from one to hundreds to
millions\perotto{artificial limit? Document/verify}. Where the data set is too
large to fit on one node, it can be spread over many nodes, divided with the
user's guidance according to data locality needs.

\perottoinline{Cost? Performance? Failures? Ease of use?}


\paragraph{Based on version control}

DMV is inspired by distributed version control. Its core data structure is a
DAG, based on Git's but modified to store a wider range of file sizes. The key
difference is that large files are broken into smaller chunks (around
\SI{52}{\kibi\byte}) for easier handling\perotto{as in...}. Breaking files into chunks also allows
the data structure to naturally de-duplicate parts of files that do not change.
For example, if a large media file has its metadata block updated, only the
chunk containing the updated metadata is new. The other chunks will simply be
reused.


\paragraph{Always writable}

Like in version control, the DAG structure records all history of the data set
and allows many different branches of development to exist in parallel. This
allows high availability. Any node can always make updates autonomously, without
coordinating with other nodes. Reconciliation of conflicting writes happens
later via explicit, user-driven merging. DMV only requires a connection to another node during
explicit synchronize operations, and so it is well-suited for applications with
intermittent or high-latency connectivity.


\paragraph{Configurable sharding}

The DAG structure tracks the data set in three dimensions:

\begin{tight_enumerate}
    \item The set of files themselves
    \item The history of the files
    \item The parallel branches of development in the history
\end{tight_enumerate}

Traditional distributed version control systems tend to assume that each replica
has the full history of all files, though not every branch of development. In
contrast, each DMV node can store a subset of the data along any of those
dimensions, configurable by the user. A node could keep the full history of only
a small subset of files, or only the most recent few versions of the full set of
files, or only a few branches, or any combination.
\todo{Be consistent about "node" vs "replica"}


\paragraph{Data integrity}

Because the DAG is append-only, and DAG objects are addressed by a cryptographic
hash of their contents, it is easy\perotto{easy how? cost?} to verify data integrity and detect corrupted
objects. A corrupted object can easily\perotto{easy how?} be replaced by an intact copy from
another replica. DMV should never lose data accidentally. However, because DMV
tolerates an incomplete DAG, objects can be deliberately deleted from all nodes
to save space or to redact sensitive information.


\paragraph{Security Model}
\perotto{treat this as a limitation as well as an assumption}

The DAG's append-only nature and cryptographic content-addressing provide
protection from tampering. As long as the complete DAG is available, its
integrity can be verified. Allowing an incomplete DAG does introduce an opening
for tampering, because not all objects are available to verify, but we ignore
such a possibility for now. Because DMV is primarily meant to allow individuals
or organizations to manage their own data on their own hardware, we assume that
all nodes will be under the user's control, and that users will only accept
additions to the DAG from trusted collaborators. This makes the aforementioned
tampering less of a concern. It is also why we do not consider byzantine
failures or guard against malicious nodes. Data can be kept private by keeping
all DMV nodes on a private network\perotto{though, the nodes can be compromised}.

%


\section{What's in a Name?}

We chose the name Distributed Media Versioning because it is a concise way to
describe the system, emphasizing its distributed nature, its roots in version
control, and its goal of storing a wide range of media rather than just source
code. The acronym DMV makes for a short and easy-to-type base command for
command line control, in the grand tradition of \lstinline{cvs},
\lstinline{svn}, \lstinline{git}, and \lstinline{hg}. And though in many places
the acronym is associated with a Department of Motor Vehicles, it does not seem
to have any prior conflicting uses in the computing domain.\footnote{Possibly
because of negative associations with the Department of Motor Vehicles} It is
also a nod to Michael's home town of Washington DC, where the Washington
metropolitan area is sometimes referred to as "The DMV" as it spills out of the
District of Columbia and into Maryland and Virgina\perotto{bold or other marker
for initial letters?}.

\chapter{DMV Architecture}

DMV's data set and its history are represented as a DAG, and its architecture
flows from that.

Each DMV node is a \newterm{repository} consisting of a content-addressable
\newterm{object store} for immutable DAG objects and a \newterm{working
directory} for active file editing (\autoref{fig:dia_obj_db_and_wd}).
Repositories that are used only for storage can omit the working directory,
similar to a Git bare repository.

\begin{figure}[h]
  \centering
    \includegraphics[width=0.5\textwidth]{dia_obj_db_and_wd}
  \caption{Repositories, object stores, and working directories}
  \label{fig:dia_obj_db_and_wd}
\end{figure}

Each repository is autonomous, storing a portion of the DAG, and able to add to
it at any time. However, it can transfer DAG objects to and from other
repositories, and it can cache data about what DAG objects are available at a
remote repository. Thus, DMV forms an ad-hoc, unstructured network of
repositories, and each repository can inform the user about what data is
available where. Together, the repositories hold the entire data set
(\autoref{fig:dia_architecture}).

\begin{figure}[h]
  \centering
    \includegraphics[width=0.95\textwidth]{dia_architecture}
  \caption{Repositories in an ad-hoc network}
  \label{fig:dia_architecture}
\end{figure}

DMV assumes that each node will connect to a human-scale number of other nodes,
maybe tens or hundreds. DMV does not dictate network structure. The user or a
higher-level application may determine the network topology and workflow
according to their needs.

Actual network connects need only be made when transferring objects or updating
lists of available nodes. So intermittent or high-latency networks are not a
problem\perotto{already said}.

\chapter{DMV Design}


\section{DMV's DAG}

The heart of DMV is the directed acyclic graph it uses to store data and
history. DMV's DAG is based on Git's, but it adds a new object type, the
\newterm{chunked blob}, which represents a blob\todo{Consistent capitalization
of "blob"} that has been broken into several smaller chunks. The objects that
make up the DMV DAG are as follows:

\begin{description}

    \item[Blob] As with Git, a blob simply holds binary data.

    \item[Chunked blob] Unlike with Git, larger blobs in DMV are broken into
        chunks. Each chunked blob is simply an index of blobs (or other chunked
        blobs) that make up the larger blob. A file that is stored in the system
        may be represented by a blob or a chunked blob.

    \item[Tree] As with Git, a directory that is stored in the system is
        represented by a tree object. The tree refers to the blob, chunked blob,
        or tree that represents each file or subdirectory, along with metadata
        such as the filename.

    \item[Commit] As with Git, a commit represents a given state of the data
        set. It refers to the tree that represents the top-level directory of
        the data set at that state, along with metadata such as author, date,
        and description. It also refers to the previous commit (or commits) that
        represented the previous state of the data set.

    \item[Ref] As with Git, a ref is a reference to a particular commit. A ref
        might represent the current state of a branch of development, or a
        particular state to save for later (a tag).

\end{description}

The DAG begins with refs. Each repository has a list of refs that lead into the
DAG. Each ref refers to a commit. Each commit refers to a tree that represents
the state at the time of committing, plus one or more parent commits. A tree
refers to the blob, chunked blob, or tree that makes up each entry in the
directory. A chunked blob can refer to other chunked blobs or to blobs. And
finally, the blobs are the leaves of the graph. This relationship is illustrated
in \autoref{fig:dia_new_dag}.


\begin{figure}[h]
    \caption{DMV DAG Object Types}
    \label{fig:dia_new_dag}
    \centering
        \includegraphics[width=0.5\textwidth]{dia_new_dag}
\end{figure}

\missingfigure{Example DMV DAG (Modify visualizer script?)}


\towrite{Low-level}

\towrite{Wide variety of platforms}


\towrite{Not required to store all of the DAG}

\towrite{Subtree commits}

\towrite{Sparse commits}

\towrite{Usage assumptions
    - Assume more reads than writes
    - Assume more appends or rewrites than in-place updates
}


\subsection{What the system should not do}
\todo{Update tense}
\todo{Break this up?}
\todo{Move it to Idea section? As a sidebar?}

We want to focus on the problem of storing file history and synchronizing files
between replicas.
We should be careful not to expand across the wrong abstraction boundaries or to
try to do too much.
In particular:

\begin{itemize}

  \item We do not want to reinvent the filesystem. The system should place and
    update files on the filesystem (or offer a filesystem view, such as with
    FUSE) for applications to use normally. Applications such as editors should
    not have to be rewritten to use our system.

  \item We do not want to create new exotic file formats. We believe that the
    classic tree of files is our best chance for long-term storage.

  \item We hope this system could eventually be used as a piece of
    infrastructure on which to build useful applications. It should not
    incorporate functionality that would better be left to an application.

  \item We do not want to deal with media metadata and categorization. Metadata
    and categorization is best left to the applications that produce and consume
    those media formats. We will merely provide the storage.

  \item However, knowledge of media formats might be used for behind-the-scenes
    optimization such as more efficient compression. E.g. recognizing that only
    tag data has changed in an audio file.

\end{itemize}

\chapter{Implementation}

We have written a DMV prototype as a proof-of-concept. The DMV prototype is
written in the Rust programming language and it runs from the Unix command line,
though it can also be compiled as a library to be used by other applications.
DMV is used on the command line like other version control systems and includes
familiar subcommands such as branch, commit, and checkout
(\autoref{prototype-help-output}\todo{Actually rename the prototype and
recapture output}).

\begin{figure}[h]
    \caption{DMV help output, listing subcommands}
    \label{prototype-help-output}
    \lstinputlisting[nolol,basicstyle=\scriptsize]{lst-prototype-help-output.txt}
\end{figure}



\section{Rust}

The implementation language, Rust, is a new C-like systems language that uses a
sophisticated type system to guarantee memory safety\todo{Cite Rust background},
avoiding data races in concurrent code, as well as the segmentation faults and
buffer overflows that are so easy to write in C. We chose Rust primarily because
it is a compiled language. Compiling to machine code gives us the best chance of
porting DMV to low-powered or mobile devices, and it allows us to compile DMV as
a library.



\section{Working Directory and Object Store}\label{dir-impl}

The DMV prototype stores its objects as regular files on the file system, using
the same structure that Git does. The object store is in a hidden directory
inside the working directory (\lstinline{.dmv/objects}). Objects are stored
using their SHA-1 hash as their filename, with the first two hex digits removed
to create a directory name (\autoref{dir-scheme-example}). This leads to
\num{256} subdirectories, each holding roughly \num{1/256}th of all the objects
stored. We experimented with other schemes to divide the objects
(\autoref{dir-experiment}), but found Git's to be the most reasonable.

\begin{figure}[h]
    \caption{Example object file name}
    \label{dir-scheme-example}
    \begin{tabular}{ l r }
        Object SHA-1 hash & \lstinline{c6e2f43ddee3c00041cdae8fedc3bd6961e61f69} \\
        Object file name & \lstinline{.dmv/objects/c6/e2f43ddee3c00041cdae8fedc3bd6961e61f69} \\
    \end{tabular}
\end{figure}

%


\section{Chunking algorithm}\label{chunking-algoritm}

Files are broken into chunks using the same rolling hash algorithm used by Gzip
and Rsyncrypto\cite{rsyncrypto_algorithm} to respectively compress and encrypt
files by chunks so that the result is "rsyncable"-- a remote copy of the
compressed or encrypted file can be updated by transferring only those chunks
that have changed.

The algorithm keeps a sum of the previous \num{8196} bytes of input data, and
creates a chunk boundary when that sum is evenly divisible by \num{4096}. We
increased the parameters to yield larger chunks, summing the previous
\num{32768} bytes (\SI{32}{\kibi\byte}) and dividing by \num{16384}
(\SI{16}{\kibi\relax}).

\begin{equation*}
    \left( \sum_{i = n - \num{32768} }^{n}{P_i} \right) \mod \num{16384} = 0
\end{equation*}

This gives a mean chunk size of \SI{52}{\kibi\byte} with a standard deviation of
\SI{21}{\kibi\byte}. The experiments used to measure and tune chunk sizes are
described in \autoref{rolling-hash-exp}.

\chapter{VCS Scaling Experiments}
\label{num-files-exp-desc}
\label{file-size-exp-desc}

\perottoinline{Otto does not like the term "test". These are \emph{experiments}.}

We performed experiments to probe the limits of existing \acrlongpl{VCS}, to see
how they would cope with file sizes and file numbers in ranges beyond what would
be expected in a source code tree. We wanted to see how long it would take for
each \gls{VCS} to store that amount of data, how much disk space it used, and
what CPU utilization was like during storage. And since \glspl{VCS} track
changes, so we also wanted to see what would happen when a small subset of that
data was modified and then committed again.

We conducted two major experiments. To measure the effect of file size, we would
\gls{commit} a single file of increasing size to each target \gls{VCS}. And to
measure the effect of file numbers, we would \gls{commit} increasing number of
small (\SI{1}{\kibi\byte}) files to each target \gls{VCS}.



\section{Version Control Systems Evaluated}

We ran each experiment using four different \glspl{VCS}: Git, Mercurial (also
known as \newterm{hg}), Bup, and the \gls{DMV} prototype (specific versions
listed in \autoref{vcs-versions}). As a control we also ran the experiments
using a dummy \gls{VCS} that simply made a copy of the files. We chose Git and
Mercurial because they are the two most widely used \glspl{DVCS} available. We
also included Bup because it uses Git's storage format but, like \gls{DMV}, it
also breaks files into chunks. For discussion on the similarities and
differences between \gls{DMV} and Bup, see \autoref{related_bup}.

\towrite{Include more background about Git, Mercurial, and Bup: Git and Hg
conceptually use same DAG but implemented differently. Bup uses Git's file
format directly but is rewritten.}

\begin{table}
    \caption{Version control systems tested and their versions}
    \label{vcs-versions}
    \centering
    \begin{tabular}{ l l }
        Git & 2.1.4 \\
        Mercurial (hg) & 3.1.2 \\
        Bup & debian/0.25-1 \\
        DMV prototype & exp\_prototype2x1mem (c9baf3a) \\
    \end{tabular}
\end{table}


\section{Procedure}

For each experiment, the procedure for a single trial was as follows:

\begin{tight_enumerate}

    \item Create an empty repository of the target \gls{VCS} in a temporary
        directory

    \item Generate target data to store, either a single file of the target
        size, or the target number of \SI{1}{\kibi\byte} files

    \item \Gls{commit} the target data to the repository, measuring wall clock
        time to \gls{commit}

    \item Verify that the first \gls{commit} exists in the repository, and if
        there was any kind of error, run the repository's integrity check
        operation

    \item Measure the total repository size

    \item Overwrite a fraction of each target file

    \item \Gls{commit} again, measuring wall clock time to \gls{commit}

    \item Verify that the second \gls{commit} exists in the repository, and if
        there was any kind of error, run the repository's integrity check
        operation

    \item Measure the total repository size again

    \item Delete temporary directory and all trial files

\end{tight_enumerate}

We increased file sizes exponentially by powers of two from \SI{1}{\byte} up to
\SI{128}{\gibi\byte}, adding an additional step at \num{1.5} times the base size
at each order of magnitude. For example, starting at \SI{1}{\mebi\byte}, we
would run trails with \SI{1}{\mebi\byte}, \SI{1.5}{\mebi\byte},
\SI{2}{\mebi\byte}, \SI{3}{\mebi\byte}, \SI{4}{\mebi\byte}, \SI{6}{\mebi\byte},
\SI{8}{\mebi\byte}, \SI{12}{\mebi\byte}, and so on.

We increased numbers of files exponentially by powers of ten from one file to
ten million files, adding additional steps at \num{2.5}, \num{5}, and \num{7.5}
times the base number at each order of magnitude. For example, starting at
\num{100} files we would run trials with \num{100}, \num{250}, \num{500},
\num{750}, \num{1000}, \num{2500}, \num{5000}, \num{7500}, \num{10000}, and so
on.

Test data files consisted simply of pseudo-random bytes taken from the operating
system's pseudo-random number generator (\lstinline{/dev/urandom} on Linux).

When updating data files for the second \gls{commit}, we would overwrite a
single contiguous section of each file with new pseudo-random bytes. We would
start one-quarter of the way into the file, and overwrite \num{1/1024}th of the
file's size (or 1 byte if the file was smaller than \SI{1024}{\kibi\byte}). So a
\SI{1}{\mebi\byte} file would have \SI{1}{\kibi\byte} overwritten, a
\SI{1}{\gibi\byte} file would have \SI{1}{\mebi\byte} overwritten, and so on.


\section{Automation and Measurement}

The trials were run via a Python script that would set up, run, and clean up
each trial in a loop, covering the full range of sizes or numbers for a given
\gls{VCS}. The script would measure the wall-clock time duration taken by each
\gls{commit} command and collect CPU utilization metrics. It would also
terminate any individual \gls{VCS} operation that ran longer than five hours.
After \gls{commit} and verification, the script would also measure repository
size.

The script measured the wall-clock time duration for each \gls{commit} by
checking the system time (\lstinline{time.time()}) just before and just after
using Python's \lstinline{subprocess} module to execute the necessary \gls{VCS}
command. CPU utilization was measured by sampling the CPU status lines provided
in Linux's \lstinline{/proc/stat} information. The status lines show a
cumulative count of time slices that the CPU has spent in user mode, system
mode, and waiting for I/O. Like with the time measurements, the script samples
CPU utilization before and after executing a \gls{VCS} command, and then
subtracts to get the number of time slices spent in each state during execution.
We then compare the relative number of time slices in each state to get an idea
of whether the operation is CPU-bound or I/O-bound.

The script measures repository size using the standard Unix disk usage command
(\lstinline{du}) and measures the size of the trial's entire temporary
directory, which includes the generated test data itself along with the
\gls{VCS}'s storage.


\section{Experiment Platform}

We ran the trials on four dedicated computers with no other
load. Each was a typical office desktop with a \SI{3.16}{\giga\hertz}
\num{64}-bit dual-core processor and \SI{8}{\gibi\byte} of RAM, running Debian
version 8.6 ("Jessie"). Each computer had one normal SATA hard disk (spinning
platter, not solid-state), and trials were conducted on a dedicated
\SI{197}{\gibi\byte} LVM partition formatted with the ext4 filesystem. All came
from the same manufacturer with the same specifications and were, for practical purposes,
identical.
Additional details can be found in \autoref{test-machine-specs}.

\begin{table}
    \caption{Test computer specifications}
    \label{test-machine-specs}
    \begin{tabular}{ l l }
        Vendor & Hewlett Packard \\
        CPU & Intel(R) Core(TM)2 Duo CPU     E8500  @ 3.16GHz \\
        RAM & \SI{8}{\gibi\byte} \\
        Hard disk & ATA model ST3250318AS \\
        \midrule
        Operating system & Debian 8.6 ("Jessie") \\
        Kernel & Linux 3.16.0 \\
        \midrule
        Test partition & \SI{197}{\gibi\byte} LVM partition \\
        Filesystem & ext4 \\
        I/O scheduler & cfq (unless otherwise noted) \\
    \end{tabular}
\end{table}

We ran every trial four times, once on each of the test computers, and took the
mean and standard deviation of each time and disk space measurement. However,
because the test machines are practically identical, there was little real variation.

\tweak{\clearpage~} % Push these two tables to the next (left) page

\begin{table}[p]
    \caption{Observations as file size increases}
    \label{file-sizes-table}
    \centering
    \begin{tabular}{r l}
        Size & Observation \\
        \midrule
        \SI{1.5}{\gibi\byte} & Largest successful commit with Mercurial \\
        \SI{2}{\gibi\byte} & Mercurial commit rejected \\
        \SI{8}{\gibi\byte} & Largest successful commit with Git \\
        \SI{12}{\gibi\byte} & Git false-alarm errors begin, but commit still intact \\
        \SI{16}{\gibi\byte} & Largest successful Git fsck command \\
        \SI{24}{\gibi\byte} & Git false-alarm errors begin during fsck, but commit still intact \\
        \SI{64}{\gibi\byte} & Largest successful DMV commit \\
        \SI{96}{\gibi\byte} & DMV timeout after \SI{5.5}{\hour} \\
        \SI{96}{\gibi\byte} & Last successful commit with Bup (and Git, ignoring false-alarm errors) \\
        \SI{128}{\gibi\byte} & All fail due to size of test partition \\
    \end{tabular}
\end{table}

\begin{table}[p]
    \caption{Effective size limits for VCSs evaluated}
    \label{vcs-size-limits-table}
    \centering
    \begin{tabular}{l l}

        VCS & Effective limit \\
        \midrule

        Git & Commit intact at all sizes, UI reports errors at \SI{12}{\gibi\byte} and larger \\

        Mercurial & Commit rejected at \SI{2}{\gibi\byte} and larger \\

        Bup & Successful commits at all sizes tried, up to \SI{96}{\gibi\byte} \\

        DMV & Successful commits up to \SI{64}{\gibi\byte}, timeout at
        \SI{5.5}{\hour} during \SI{96}{\gibi\byte} trial

    \end{tabular}
\end{table}

\cleardoublepage

\section{Results: File Size}
\label{file-size-results}

\subsection{File Size Limits}
\label{file-size-limits-results}

Both Git and Mercurial had limits to the size of file they could store
successfully.

With a \SI{2}{\gibi\byte} file, Mercurial's \gls{commit} operation would exit
with an error code and message, saying "up to 6442 MB of RAM may be required to
manage this file," and the \gls{commit} would not be stored. However, the
\gls{repository} would be left in a consistent empty state. The atomicity of the
\gls{commit} operation held. All commits of files \SI{2}{\gib} and larger would
be rejected in a similar manner.

Git's behavior was more erratic. Starting with a file \SI{12}{\gibi\byte} and
larger, Git's \gls{commit} operation would exit with an error code, reporting a
fatal out-of-memory error saying that
\lstinline{malloc}
failed to allocate \SI{12}{\gibi\byte}.
However, the \gls{commit} would be successfully stored with no consistency
errors in the \gls{repository} reported by
\lstinline{git fsck}.
Starting at \SI{24}{\gibi\byte}, the \gls{commit} operation would report the
same error, the \gls{commit} would still be written, but then the
\lstinline{git fsck}
integrity check itself would also exit with an error code.
However, the error it reported in its output was a similar "fatal" error as
Git's
\lstinline{malloc}
error as the \gls{commit} operation, and it did not report any actual integrity
errors in the \gls{repository}.

So to check the \gls{commit}, we extracted the \SI{24}{\gibi\byte} file from the
\gls{repository} and compared it. It was the same as the original. So the
\gls{commit} was intact. We also deliberately corrupted the Git \gls{packfile}
that stored the \SI{24}{\gibi\byte} by overwriting one \SI{1}{\mebi\byte} block
at an offset of \SI{22}{\gibi\byte} with new pseudo-random data. When we ran the
\lstinline{fsck} command again with the corrupted \gls{repository}, it reported the
integrity error, but it did not report the \lstinline{malloc} error that it did
before.

The \lstinline{git fsck} command found the integrity error surprisingly quickly,
reporting the error and exiting instantaneously (from the user's perspective),
whereas it had taken \SI{7}{\minute} to complete before. So it does not seem to
have hashed the whole file again to find the error. We took care to preserve the
modification time for the pack file when we corrupted it, so we are not sure how
Git detected the error so quickly. Perhaps our corruption overwrote some
internal indexing structure of the pack file along with the file data. More
investigation would be necessary to know for sure.

We continued the experiment under the assumption that Git's errors were a false
alarm, and allowed the trials to continue at even larger sizes.

The \gls{DMV} prototype was able to store a file up to \SI{64}{\gibi\byte} in
size, but time became a limiting factor as file size increased. At
\SI{96}{\gibi\byte}, our experiment script timed out and terminated the
\gls{commit} after five and a half hours. This sluggishness is due to the way
DMV stores chunks of the file as individual files on the filesystem, turning the
problem of storing one large file into the problem of storing many small files.
Storing many small files in this way incurs filesystem overhead, as we
discovered in the results of the number-of-files experiment
(\autoref{results-num-files--c1-time}), and later performed more experiments to
examine in detail (\autoref{perf-tuning-exp-chapter}).

Our experiment environment itself limited the largest file stored by any
\gls{VCS} to \SI{96}{\gibi\byte}. Any larger and it was simply impossible to
store a second copy of the file on our \SI{197}{\gibi\byte} test partition. Bup
was able to store a \SI{96}{\gibi\byte} file with no errors in just under two
hours. Git could also store such a large file, but one must ignore the
false-alarm "fatal" errors being reported by the user interface.

These findings are summarized in \autoref{file-sizes-table} and
\autoref{vcs-size-limits-table}.

%

\clearpage ~~~

\begin{figure}[p]
    \begin{leftfullpage}
        \caption{Wall-clock time to commit one large file to a fresh repository}
        \label{fig:plot-file-size--c1-time}
        \centering

        \explainlogsubfig

        \includegraphics[]{plot-file-size--c1-time}
    \end{leftfullpage}
\end{figure}

\cleardoublepage

\subsection{Time for File-Size Initial Commit}
\label{results-file-size--c1-time}

\autoref{fig:plot-file-size--c1-time} shows the wall-clock time required for the
initial \gls{commit}, adding a single file of the given size to a fresh
\gls{repository}. Over all, the trend is clear and unsurprising: \gls{commit}
time increases with file size. It increases linearly for Git, Mercurial, and
Bup. DMV's commit times increase in a more parabolic fashion, which is most
apparent in \autoref{fig:plot-file-size--c1-time}e.

At file sizes below around \SI{2}{\mebi\byte}
(\autoref{fig:plot-file-size--c1-time}a and b), \gls{commit} times are dominated
by overhead --- around \SI{5}{\ms} for Git, \SI{100}{\ms} for Mercurial,
\SI{180}{\ms} for \gls{DMV}, and \SI{900}{\ms} for Bup, vs only \SI{2}{\ms} for
the copy.

Bup, after starting with the highest overhead, goes on to have the fastest
initial \gls{commit} of all the systems evaluated for large files. It takes the
lead at \SI{2}{\gibi\byte}, where Mercurial drops out
(\autoref{fig:plot-file-size--c1-time}d). To \gls{commit} the \SI{2}{\gibi\byte}
file, Git's average time is \SI{91.1}{\s}, Bup's is \SI{89.1}{\s}, and
\gls{DMV}'s is \SI{90.8}{\s}. All of these are a factor of around ten times
slower than the direct copy at \SI{9.1}{\s}. The differences get more pronounced
as the file sizes continue to increase. At \SI{64}{\gibi\byte}, Git's average
time is \SI{110}{\minute}, Bup's is \SI{72}{\minute}, \gls{DMV}'s is
\SI{298}{\minute}. The average \SI{64}{\gibi\byte} copy takes \SI{35}{\minute}.

DMV's parabolic increase is due to the way it breaks the large file into chunks
and stores objects as individual files on the filesystem. While it is reading
one large file, it is writing many small files, which incurs filesystem
overhead. So its performance characteristic for storing a large file is closer
to that of storing many files (\autoref{results-num-files}). Bup also breaks the
file into many chunks, but it avoids the filesystem overhead by recombining the
chunks into \glspl{packfile}. We investigate the filesystem overhead further in
\autoref{perf-tuning-exp-chapter}.

Bup's commit times behave strangely in that there are places where Bup is
actually faster that it was with a smaller file. This is most apparent in the
slow downward slope of \autoref{fig:plot-file-size--c1-time}b and the zig-zag of
\autoref{fig:plot-file-size--c1-time}c. Git also has a point where the time
decreases, taking \SI{393}{\s} (SD \SI{.76}{\s}) to commit a \SI{4}{\gib} file
and only \SI{361}{\s} (SD \SI{3.90}{\s}) to commit an \SI{8}{\gib} file. Even
more interestingly, these decreases are consistent across the four trials on
separate hardware. We do not know what might be causing this.

%


\begin{figure}[p]
    \begin{leftfullpage}
        \caption{Wall-clock time to commit one updated large file}
        \label{fig:plot-file-size--c2-time}
        \centering

        X axis shows the total size of the file. The updated portion was
        \SI{1/1024}{th} of the total file size.
        \explainlogsubfig

        \includegraphics[]{plot-file-size--c2-time}
    \end{leftfullpage}
\end{figure}

\cleardoublepage

\subsection{Time for File-Size Update Commit}
\label{results-file-size--c2-time}

\autoref{fig:plot-file-size--c2-time} shows the wall-clock time required for the
second \gls{commit}, after updating \num{1/1024}th of the file. Ideally this
operation should be faster than the first \gls{commit}, because the system
should only be storing the changed portion of the file. Indeed this is the case
for Mercurial, Bup, and \gls{DMV}, which do store only the changed portion. Git,
however, copies the entire updated file into its \gls{repository} as a new
object, and so its \gls{commit} time is virtually identical. The same is true of
the copy control, though for sizes smaller than \SI{8}{\gib} it is still faster
than all the other systems.

As with the initial commit, Bup gets faster as file size increases at certain
points, with the same gradual downward slope in the sub-megabyte and low
megabyte-ranges, leading to a prominent jump up then fall back down that occurs
just outside the range of \autoref{fig:plot-file-size--c2-time}c. The jump comes
at a slightly larger size with this update commit, at \num{8}, \num{12},
\num{16}, and \SI{24}{\mib}, as opposed to \num{4}, \num{6}, \num{8}, and
\SI{12}{\mib} in the initial commit. It can still be seen in miniature in
\autoref{fig:plot-file-size--c2-time}a.

The shift to larger sizes in the update
commit suggests that the decrease is related to the amount of data written to
the disk, since Bup breaks the file into chunks and only writes the updated
chunks.

Git also shows a commit time decrease between \SI{4}{\gib} (\SI{518}{\s} with SD
\SI{11.1}{\s}) and \SI{8}{\gib} (\SI{429}{\s} with SD \SI{4.0}{\s}) just as it
did with the initial commit. Unlike Bup, its decrease is not shifted to higher
file sizes, which is another hint that the decrease has something to do with the
amount of data written, since Git writes the whole file again, rather than just
the updated portion.

%


\begin{figure}[p]
    \caption{CPU utilization while committing one large file to a fresh repository}
    \label{fig:plot-file-size--c1-cpu}
    \centering
    \includegraphics[]{plot-file-size--c1-cpu}
\end{figure}

\begin{figure}[p]
    \caption{CPU utilization while committing changes to one large file}
    \label{fig:plot-file-size--c2-cpu}
    \centering
    \includegraphics[]{plot-file-size--c2-cpu}
\end{figure}

\cleardoublepage

\subsection{CPU Usage During File-Size Commits}

\autoref{fig:plot-file-size--c1-cpu} shows CPU usage during the initial
\gls{commit}, and \autoref{fig:plot-file-size--c2-cpu} shows CPU usage during
the update commit. "User" indicates user-space computation, "system" indicates
kernel-space computation, and "iowait" indicates that the CPU was waiting on an
I/O operation.

We expected the commit operations to be I/O bound, and that might not be the
case. It is with DMV and with the copy, especially at file sizes \SI{1}{\gib}
and larger. But there is also a significant amount of user-space work going on
in all of the version-control systems, such as hashing the data and, in
Mercurial's case, calculating deltas. The low I/O wait activity in Git,
Mercurial, and Bup is surprising. It's possible that the commit is actually CPU
bound in those systems, but that seems unlikely.

DMV's high I/O wait is probably primarily due to the filesystem overhead that is
slowing down its commit times, but the current single-threaded implementation
might also be contributing, and the prototype might benefit from having separate
threads to load data, hash it, and write it to dish.

There is a small peak in the DMV I/O wait percentage for the first commit
(\autoref{fig:plot-file-size--c1-cpu}d), peaking at \SI{386}{\mib} with
\SI{337}{ticks} (SD \SI{13.8}{ticks}), then falling at \SI{512}{\mib} to
\SI{89}{ticks} (SD \SI{49.2}{ticks}), and then rising again at \SI{768}{\mib} to
\SI{972}{ticks} (SD \SI{54.3}{ticks}). We are not sure what is causing this. We
also do not know why the DMV I/O wait during the update commit actually
decreases between \SI{1}{\gib} and \SI{32}{\gib}
(\autoref{fig:plot-file-size--c2-cpu}d).

  %402653184         337.25   13.7727085208
  %536870912          89.25   49.1801535175
  %805306368         972.25   54.2972144774

The copy operation shows an erratic amount of system activity at low sizes,
which is surprising since the copy operation is almost pure I/O. This is a
result of how the data was collected rather than any unexpected behavior in the
copy operation. At sizes below about \SI{10}{\mib}, the copy operation is faster
than the \lstinline{/proc/stat} ticks used to CPU measure activity
(\num{1/100}{th} of a second~\cite{proc_man_page}). At such small intervals, most
measurements will yield zero ticks, though one of the states will occasionally
measure one tick (see \autoref{copy-cpu-data}). Then, when the graph normalizes
the total usage to a percentage, \num{1} system tick out of \num{1} total ticks
makes \SI{100}{\percent}, so the graph will jump from \SI{0}{\percent}
\SI{100}{\percent}, or somewhere in between depending on how many of the four
trials measured one tick. It isn't until around \SI{48}{\mib} that the
measurement includes enough CPU ticks to yield useful percentages, which is
where we see the graph start to stabilize.


\begin{table}[hp]
    \caption{Selected CPU usage data for copy operation}
    \label{copy-cpu-data}

    \centering
    \begin{tabular}{r r r r r}
        Size & User & System & Idle & Iowait \\
        \midrule
  4.0MiB &    0    &      0    &      0    &      0 \\
  6.0MiB &    0    &      0    &      1    &      0 \\
  8.0MiB &    0    &      1    &      1    &      0 \\
 12.0MiB &    1    &      1    &      2    &      0 \\
 16.0MiB &    0    &      2    &      2    &      0 \\
 24.0MiB &    1    &      3    &      3    &      0 \\
 32.0MiB &    0    &      3    &      4    &      0 \\
 48.0MiB &    0    &      5    &      5    &      0 \\
 64.0MiB &    0    &      7    &      6    &      0 \\
 96.0MiB &    0    &     10    &     10    &      0 \\
    \end{tabular}
\end{table}

%


\begin{figure}[p]
    %\begin{leftfullpage}
        \caption{Total repository size after committing, editing, and committing again}
        \label{fig:plot-file-size--repo-size}
        \centering

        \explaindiskspaceplot

        \includegraphics[]{plot-file-size--repo-size}
    %\end{leftfullpage}
\end{figure}

\cleardoublepage

\subsection{Repository Size after File-Size Update Commit}

\autoref{fig:plot-file-size--repo-size} shows the total \gls{repository} size
after the update \gls{commit}, including the original file. This is after
committing, updating \num{1/1024}th of the file, and committing again.

The stored data overtakes the initial \gls{repository} overhead after a file
size of around \SI{1}{\mib}, and the \gls{repository} size for all systems
converges to about twice the size of the file. This is to be expected, since
each measurement includes the original file, the first copy of the file, and the
updated \num{1/1024}th. The exception is Git, which stores the entire updated
file during the update \gls{commit}, leading to a total disk space usage of
three times the file size. However, Git has a separate garbage collection stage
where it cleans up the \gls{repository} and aggregates similar objects together
in \glspl{packfile}. The post-garbage collection size for Git is shown as a
separate line on the graph. This post-GC size converges to double the original
file size, but then jumps to three times at a file size of \SI{1.5}{\gib}. This
suggests that the \glsdisp{packfile}{pack} step is failing silently at
\SI{1.5}{\gib} and larger. This is probably related to the way Mercurial's
\glspl{commit} begin failing at \SI{2}{\gib} and larger. Both operations are
trying to load multiple versions of the file into memory to calculate deltas for
\glsdisp{packfile}{packing}.

%


\cleardoublepage

\section{Results: Number of Files}
\label{results-num-files}

\subsection{File Quantity Limits}

Git, Mercurial, DMV, and the copy operation all failed when trying to store
\num{7.5} million files or more, reporting that the disk was full. However, it
wasn't actually out of space. The disk was out of \emph{\glspl{inode}}.

\glsreset{inode} % This is the para where we actually define what an inode is

Unix filesystems, ext4 included, store file and directory metadata in a data
structure called an \gls{inode}, which reside in a fixed-length
table~\cite{unix_timesharing_system}. When all of the \glspl{inode} in the table
are allocated, the filesystem cannot store any more files or directories.

Git, Mercurial, DMV, and the copy all create one file in their
\glspl{objectstore} for each input file. So to store \num{7.5} million files,
they will create \num{7.5} million more, resulting in \num{15} million files on
the filesystem, plus directories. However, the \SI{197}{\gib} experiment
partition has \num{13107200} total \glspl{inode}, so storing \num{15} million
files is impossible.

Bup is able to store more files because it does not write a separate file for
each input file. Bup aggregates its DAG objects into \glspl{packfile}, writing
several large files instead many small files. As such, it does not exhaust the
disk's \glspl{inode}, and can continue until the experiment itself exhausts the
system's \glspl{inode} when trying to run a trial with \num{25} million files.

%

\begin{figure}[p]
    \begin{leftfullpage}
        \caption{Wall-clock time to commit many 1KiB files to a fresh repository}
        \label{fig:plot-num-files--c1-time}
        \centering

        \explainlogsubfig

        \includegraphics[]{plot-num-files--c1-time}
    \end{leftfullpage}
\end{figure}

\begin{figure}[p]
    \begin{leftfullpage}
        \caption{Wall-clock time to commit many updated files}
        \label{fig:plot-num-files--c2-time}
        \centering

        X axis shows the total number of files. 1 out of every 16 files was updated.
        \explainlogsubfig

        \includegraphics[]{plot-num-files--c2-time}
    \end{leftfullpage}
\end{figure}

\cleardoublepage

\subsection{Time for Number-of-Files Initial Commit}
\label{results-num-files--c1-time}

\autoref{fig:plot-num-files--c1-time} shows the time required for the initial
\gls{commit}, storing all files into a fresh empty \gls{repository}. Here we see
the commit times for Git and DMV increasing quadratically with the number of
files, while Mercurial, Bup, and the copy increase linearly.

We saw in the file-size commit times (\autoref{results-file-size--c1-time}) that
DMV's time increased quadratically, and we suspected that was because it was
creating many small files and incurring filesystem overhead. This effect would
explain why both Git and DMV do so poorly here while Bup would fare much better.
But why then would Mercurial and the copy also have a linear increase instead of
an quadratic one?

The difference is the naming schemes of stored files. Git and DMV name each
object file according to the SHA-1 hash of the object's contents, while
Mercurial, like the copy, uses the original input file's name. This means that
Git and DMV write files in a random order with respect to their names, jumping
between different \gls{objectstore} subdirectories, while Mercurial and the copy
can write files in the order they read them, one subdirectory at a time. The
filesystem is most likely optimized for that kind of sequential write.

At \num{500}, \num{750}, and \num{1000} files
(\autoref{fig:plot-num-files--c1-time}b), Bup's commit times have an unusually
high variance compared to the other systems and the other experiments. At those
file counts, the standard deviations for Bup's commit times are
\SI{18.3}{\percent}, \SI{29.6}{\percent}, \SI{28.3}{\percent} of the means,
respectively (see \autoref{bup-commit-times-high-variance}), whereas the mean of
the other such standard deviation percentages is only \SI{6.1}{\percent}. We are
not sure what is causing this.

\begin{table}[b]
    \caption{Bup initial commit times with unusually high variance}
    \label{bup-commit-times-high-variance}
    \centering
    \begin{tabular}{rccc}
        Num. files & Mean time (s) & SD (s) & SD as \% of mean \\
           500  &   0.965  &   0.177  &  18.3 \\
           750  &   1.199  &   0.355  &  29.6 \\
          1000  &   1.407  &   0.398  &  28.3 \\
    \end{tabular}
\end{table}

%


\cleardoublepage
\subsection{Time for Number-of-Files Update Commit}

\autoref{fig:plot-num-files--c2-time} shows the wall-clock time required for the
second \gls{commit}, after updating \num{1} out of every \num{16} files. As with
the file-size experiment (\autoref{results-file-size--c2-time}), storing only
the updated files is faster, and in this case the difference is more pronounced.
This is because every system understands the file as a unit of data, and can
naturally separate the changed and files from the unchanged files.

The exception is the copy operation, which blindly copies all files again. This
is why it is actually slower than the other systems at some points. It would be
interesting to run this experiment with Rsync as well (\autoref{related-rsync}),
to get a baseline for comparing and copying only the files that have changed.

Here all commit times appear to increase linearly with respect to number of
files, except for Git, which shows some quadratic growth as the number of files
gets into the millions.

Bup does not exhibit the interesting variance here that it did in the initial
commit. All standard deviations are small enough that the error bars are barely
discernible on the graph.

%


\begin{figure}[p]
    \caption{CPU utilization while committing many 1KiB files to a fresh
    repository}
    \label{fig:plot-num-files--c1-cpu}
    \centering
    \includegraphics[]{plot-num-files--c1-cpu}
\end{figure}

\begin{figure}[p]
    \caption{CPU utilization while committing many 1KiB files after one of every
        \num{16} files has been updated}
    \label{fig:plot-num-files--c2-cpu}
    \centering
    \includegraphics[]{plot-num-files--c2-cpu}
\end{figure}

\cleardoublepage

\subsection{CPU Usage During Number-of-Files Commits}

\autoref{fig:plot-num-files--c1-cpu} shows CPU utilization during the initial
\gls{commit} and \autoref{fig:plot-num-files--c1-cpu} shows CPU utilization
during the update \gls{commit}.

With the initial commit, Git, DMV, and the copy spend more time waiting on I/O
than Bup or Mercurial, and they also spend more time in system mode. In the case
of the copy, this is probably because the operation is almost pure I/O. In the
case of Git and \gls{DMV}, this is more evidence to suggest that writing files
with effectively random names incurs more filesystem overhead than writing
sequentially, as Mercurial does, or appending to large files, as Bup does. We
can see that Bup and Mercurial both spend more time processing in user mode than
waiting for I/O.

With the update commit, Mercurial loses its sequential write advantage, since it
has to seek to the \gls{filelog} that corresponds to the current input file and
append to it. And so we see much more I/O wait with Mercurial. Bup continues to
simply append objects to its \glspl{packfile} as always, and so it retains a low
I/O wait profile.

Interestingly, Mercurial's and \gls{DMV}'s I/O wait in the update commit has a
gradual rise starting around 100 thousand files, while Git's is a sudden rise
starting just before 100 thousand files. The copy is somewhere in between. We
are not sure why that is, since \gls{DMV} has much more in common with Git than
with Mercurial in terms of disk usage patterns.

Git, \gls{DMV}, and the copy all decrease in I/O wait from \num{1} million to
\num{5} million files. And Mercurial also shows a slight decrease from \num{2.5}
million to \num{5} million files. Again, we are not sure what would be causing
that.

%


\begin{figure}[p]
    \begin{leftfullpage}
        \caption{Real time required to check the status of many files after
        initial commit}
        \label{fig:plot-num-files--stat2-time}
        \centering

        \explainlogsubfig

        \includegraphics[]{plot-num-files--stat2-time}
    \end{leftfullpage}
\end{figure}

\cleardoublepage

\subsection{Time for Number-of-Files Status Check}

With the number-of-files experiment, we also timed how long it would take each
\gls{VCS} to run its status command and check which files had changed.
\autoref{fig:plot-num-files--stat2-time} shows the time required to check the
status of all files just after updating them.

\gls{DMV} and Mercurial seem to slow quadratically with number of files. Bup
seems to have a general overhead of \SI{2}{\second}, jumping to \SI{3}{\second}
at \num{7500} files, but after that increasing linearly.

%  100000           0.383   0.0334888040993
%  250000        11.80575     6.16018239888
%  500000        32.14525     10.6227303311
%  750000        30.20975     4.70526202113
% 1000000        41.62125     8.74158247616

%                                   stat2_time
%   git-2017-03-22-murphytest04.txt:   20.330
%   git-2017-03-31-murphytest02.txt:   46.362
%   git-2017-04-04-murphytest01.txt:   23.638
%   git-2017-04-13-murphytest03.txt:   38.251

Git has an interesting drop where it actually gets faster from \num{500}
thousand to \num{750} thousand files, dropping from a mean of \SI{32.145}{\s} to
\SI{30.210}{\s}. However, the measurement at \num{500} thousand files has a high
standard deviation, \SI{10.623}{\s}, compared to only \SI{4.705}{\s} at
\num{750} thousand files. The four measurements at \num{500} thousand files are
\SI{20.330}{\s}, \SI{23.638}{\s}, \SI{38.251}{\s} and \SI{46.362}{\s}. So it is
the two unusually high measurements that are pulling the average up, but we do
not know why those measurements are especially high.

%


\begin{figure}[p]
    \caption{Total repository size after committing, editing, and committing again}
    \label{fig:plot-num-files--repo-size}
    \centering

    \explaindiskspaceplot

    \includegraphics[]{plot-num-files--repo-size}
\end{figure}

\cleardoublepage

\subsection{Repository Size after Number-of-Files Update Commit}

% Analyze:
%
% ../data/meantable.py --columns filecount c2_size \
%       --files ../data/exp--num-files--v03--less-dirs/git-* \
%   | tail -n+2 \
%   | awk -e '{printf "%7d  %5.3f\n", $1, $2/($1*1024)}'

% At 5M files...
% git: 8.478
% hg: 8.186
% bup: 5.374
% prototype2x1mem: 8.538
% copy: 8.043

\autoref{fig:plot-num-files--repo-size} shows the total \gls{repository} size
after the update commit, including the original files. This is after committing
once, changing a single byte in every sixteenth file, and committing again.

Git, Mercurial, DMV, and the copy all converge to using just over \num{8} times
the theoretical size of the data set, while Bup is closer to \num{5} times. This
has to do with the block size of the filesystem. The underlying filesystem uses
a \SI{4}{\kib} block size, so each \SI{1}{\kib} file uses one \SI{4}{\kib}
block. So the input data itself takes \num{4} times its theoretical size. So
each will end up with \num{4} times for the input files, plus \num{4} times for
the copied files, plus some overhead. So a storage ratio of just over \num{8} is
to be expected for those systems that store objects as individual files.

Naturally the copy would have the least overhead, with a ratio of \num{8.043} at
\SI{5} million files. This shows that the directory hierarchy of the input files
itself probably adds around \num{0.021} times, or \SI{2.1}{\percent}. Mercurial
has the second lowest overhead, with a storage ratio of \num{8.186}. This makes
sense because, while Mercurial creates a \gls{filelog} for each input files, it
reuses \glspl{filelog} to store the updates versions, so it does not create any
new files during the second commit.

Git and \gls{DMV} have similar, higher ratios because they do create new object
files for each new version of each input file. At \SI{5} million files Git's
ratio is \num{8.478} and \gls{DMV}'s is \num{8.538}. It would be interesting to
see Git's ratio after garbage collection and \glsdisp{packfile}{packing}, but
unfortunately we did not run Git's garbage collection as part of this experiment
as we did in the file-size experiment. We assume the results would be similar to
Bup's.

Bup uses significantly less disk space, with a ratio of \num{5.374} at \num{5}
million files. And since the input files themselves account for just over
\num{4} times the theoretical size, we can see that Bup is storing the data in a
form that is much closer to its theoretical size, taking just under \num{1.374}
times the space.

%

\chapter{Performance Tuning Experiments}
\label{perf-tuning-exp-chapter}

After noticing \gls{DMV}'s long commit times, we tried tuning certain \gls{DMV}
parameters to investigate their effects. We re-ran the many-files experiment
on \gls{DMV} several times, varying first the object store directory layout,
then Linux I/O scheduler, and finally chunk size. We also ran new, more targeted
experiments to investigate the effects of directory layout and chunk size.

%



\begin{figure}[p]
    \caption{DMV output showing varying object write times}
    \label{write-times-log-output}
    \centering

    Not shown: many objects written in under \SI{10}{\ms}, which are logged at
    \lstinline{TRACE} level.

    \programoutput{lst-storing-long-write-times.txt}
\end{figure}

\begin{table}[p]
    \caption{Sample object store directory variations}
    \label{sample-directory-scheme-variations}
    \centering
    \begin{tabular}{c c l}
        Hex digits & Depth & Example \\
        \midrule
        0 & 0 & \lstinline{03d37679d1fab86e5286decd6cd2a94efcfe083f} \\
        1 & 1 & \lstinline{7/9332ca7ce9163f78e3c115a2173bd8fd853d286} \\
        1 & 3 & \lstinline{6/8/c/40e64f3e74e6ebefdcf2f5f30fb8da004792c} \\
        2 & 1 & \lstinline{9f/4ec22c3e0289b29eefefe4728dece14e67e6ac} \\
        2 & 2 & \lstinline{dd/52/bcccff156a179cdac0793ef049039372d8a1} \\
        3 & 1 & \lstinline{cc5/199084d70f7c5ba325a240e1927579ee24bb1} \\
        3 & 4 & \lstinline{472/e98/e88/0b1/c5905065c70cbe806361d32f6429} \\
        4 & 3 & \lstinline{1ed2/bd51/01fe/5b23763e8c76852739f59201280f} \\
    \end{tabular}
\end{table}

\section{Object Store Directory Layout}
\label{dir-experiment}

During initial runs of the "many files" experiment
(\autoref{num-files-exp-desc}), we would often notice the disk being reported as
full even though the total bytes used was less than the capacity of the disk
partition. This had to do with how each system stores its objects as files on
the filesystem and how it organizes them into directories. Each file and
directory on a Unix filesystem requires one \gls{inode}, of which the filesystem
has only a finite number. A storage scheme that allocates too many files or
directories will exhaust the filesystem's available \glspl{inode} before it uses
all the available disk space.

We also noticed that average write speed would slow down as the operation
progressed. The progress meter we added to \gls{DMV}'s \gls{commit} operation
would show a rate of \SIrange{30}{40}{\mib\s} at the beginning of an operation
but slow to less than \SI{300}{\kib} by the end of a long one. We added log
output to print the write times for individual objects, and we discovered that
while most objects would be written in milliseconds, occasionally a single
object write would take multiple seconds or tens of seconds, even though there
was no appreciable difference in size between the objects
(\autoref{write-times-log-output}).

\gls{DMV} stores its objects as individual files in an \gls{objectstore}
directory, in the same manner as Git. The object's SHA-1 hash is used as its
file name, except that the first two hex digits are removed and used as a
subdirectory (also described in \autoref{dir-impl}). Our prototype originally
took the first four hex digits to create two levels of subdirectories, under the
assumption that we would store more objects than Git and need to spread them out
with more subdirectories. That original prototype was showing this odd behavior,
and it stored files much more slowly than Git. We suspected that the number of
subdirectories could be at fault, so we experimented with different subdirectory
schemes to see their effects.

%


\subsection{Procedure}

To measure the effects of different object storage schemes, we created a new
\SI{100}{\mebi\byte} partition on one of the dedicated experiment computers, and
then generated a series of pseudo-random files of \SI{4}{\kibi\byte} each until
the disk was reported full. For each file, we would give it a pseudo-random name
that resembled an SHA-1 hash, and store it according to the object storage
scheme under test. We increased a counter each time we created a file, and
another each time we created a new directory. We also checked the number of
files already in the target directory before writing, timed the write, and used
the Unix \lstinline{df} utility to measure free disk space in bytes and the
number of free \glspl{inode}.

The directory schemes we evaluated were all variations of the basic scheme of
taking leading hex digits of the SHA-1 hash to form directories. We varied the
number of directories taken (depth) and the number of hex digits per directory
(see \autoref{sample-directory-scheme-variations} for examples). We tried depths
from \num{0} to \num{6} and digits per directory from \num{0} to \num{16},
discarding combinations that did not make sense, such as combinations involving
\num{0} and another number (which would all simply be undivided), or those that
required more than the \num{40} hex digits of a \num{160}-byte SHA-1 hash.

\subsection{Environment}

Like the "many files" experiment, this was automated as a Python script and run
on one of the dedicated computers used for that experiment (specs shown in
\autoref{test-machine-specs}). However, rather than spending hours to fill the
\SI{197}{\gibi\byte} partition used for the other experiments, this experiment
used a new \SI{100}{\mebi\byte} LVM partition.


\begin{figure}[b!]
    %\begin{leftfullpage}
        \caption{Number of Files vs. number of directories filling a disk}
        \label{fig:plot-filesystem-limits--directory-takeover}
        \centering

        The number of files and directories present when the disk reported that it
        was full under the given directory scheme, shown by number of hex digits per
        directory (the different plots) and levels of depth (x axis)

        \includegraphics[]{plot-filesystem-limits--directory-takeover}
    %\end{leftfullpage}
\end{figure}

\cleardoublepage
\subsection{Results}
\label{object-dir-layout-results}

\subsubsection{Out of Inodes}

\autoref{fig:plot-filesystem-limits--directory-takeover} shows how quickly
directories overtake files as subdirectory nesting goes deeper. Presented
visually, the connection between files and directories becomes obvious. The
maximum number of files plus directories is constant and limited by the number
of \glspl{inode} on the filesystem, which on the \SI{100}{\mib} test partition
is \num{25688}. However, the number of directories created increases
exponentially with both the number of hex digits per directory and then again by
directory depth. This can be expressed mathematically.

Let $h$ denote the number of hex digits per subdirectory and let $n$ denote the
subdirectory depth. Then the total number of directories created by the scheme,
$d$ is given by

\begin{equation}
    d = \sum_{i=1}^n \left( 16^h \right)^i \quad.
\end{equation}

The directories are not created all at once, only when a file that should be
placed in that directory is stored. But because files are named according to a
uniformly distributed hash function, no particular directory will be favored and
the number of directories will trend towards $d$.

Let $o$ denote the number of \glspl{inode} available on the filesystem, and let
$f$ denote the number of files that can be stored on the filesystem when the
directory scheme creates $d$ directories. Then,

\begin{equation}
    f = o-d \quad,
\end{equation}

And therefore,

\begin{equation}
    f = o - \sum_{i=1}^n \left( 16^h \right)^i \quad.
\end{equation}

So we can see that \gls{DMV}'s original scheme, with two hex digits per
directory and a depth of two, would yield \SI{65792} subdirectories, which by
itself is more than \num{2.5} times the total number of \glspl{inode} available
on the \SI{100}{\mib} test partition. So of course it ran out of \glspl{inode}
long before running out of disk space.


\begin{figure}[p]
    \caption{Unusually high write times}
    \label{plot-seek-times}
    \centering

    \SI{4}{\kib} files that took \SI{1}{\ms} or longer to write, plotted
    according to the number of files and directories on the disk already, and
    colored by subdirectory depth. \\
    Not shown: The \SI{99.1}{\percent} of writes that were faster than
    \SI{1}{\ms}.

    \includegraphics[]{plot-seek-times}
\end{figure}

\begin{table}[p]
    \caption{Top-ten longest writes}
    \label{longest-writes}
    % To generate:
    % awk -e '$7=="ok" && $8 >= 0.5 {print $0}' \
    %   exp--filesystem-limits--micro/*murphytest04.txt \
    %   | sort -r -k 8 \
    %   | awk -e '{ printf "%05.3f & %d & %d & %5d & %5d & %2d & %04.1f \\\\ \n", \
    %                       $8, $2, $3, $4, $5, $6, $13*100/$12 }' \
    %   | head -n10
    \begin{tabular}{r r r r r r r}
        Time (\si{\s}) & Digits & Depth & Files & Dirs & Files in dir & \% inodes used \\
        \midrule
2.306 & 1 & 4 & 10699 & 13997 &  1 & 96.2 \\
2.180 & 1 & 4 & 11025 & 14289 &  1 & 98.6 \\
1.775 & 3 & 1 & 16321 &  4008 &  5 & 79.2 \\
1.654 & 1 & 5 &  5834 & 14755 &  1 & 80.2 \\
1.646 & 3 & 2 & 10831 & 14635 &  1 & 99.2 \\
1.466 & 1 & 5 &  5389 & 13790 &  1 & 74.7 \\
1.456 & 2 & 3 &  8393 & 16550 &  1 & 97.1 \\
1.443 & 4 & 2 &  7823 & 15225 &  1 & 89.8 \\
1.434 & 1 & 5 &  5922 & 14949 &  1 & 81.3 \\
1.379 & 1 & 6 &  5302 & 18885 &  1 & 94.2 \\
    \end{tabular}
\end{table}

\subsubsection{Long Write Times}

From there, we turn our attention to the mysterious, intermittent long write
times. In the experiment, across all directory schemes, there were \num{315601}
total writes. Of those, \num{312813} (\SI{99.1}{\percent}) completed \SI{1}{\ms}
or less. The others are plotted in \autoref{plot-seek-times}, and data about the
top ten longest writes is listed in \autoref{longest-writes}. The spikes in the
graph are arranged in curves radiating out from zero files and zero directories.
Each curve represents a cluster of directory schemes that filled up the disk in
the same pattern, corresponding more to subdirectory depth than to number of hex
digits per subdirectory.

No single directory scheme stands out as worse than the others, though longer
writes seem correlated with having more directories and having more inodes
already used (shown by distance from origin). The scheme with the fewest and
shortest long writes is the one that has no subdirectories at all (shown by the
short green spikes along the files axis). So we conclude that there is no
penalty for storing many thousands of files in one directory.

The two longest writes are clustered together near the center of the plot, near
\num{11000} files and \num{14000} directories. Both occur in the directory
scheme with \num{1} hex digit and a depth of \num{4}. The longest was
\SI{2.306}{\second} at \num{10699} files and \num{13997} directories
(\SI{96.2}{\percent} of inodes used), and the second was \SI{2.180}{\second} at
\num{11025} files and \num{14289} directories (\SI{98.6}{\percent} of inodes
used).

% Analyze:
%
% awk -e '$7=="ok" && $8 > 0.01 {print $0}' \
%       ../data/exp--filesystem-limits--micro/*murphytest04.txt \
%   | wc -l

Write times seem to follow a power law distribution. The two peaks are the only
two writes out of \num{315601} total that took longer than \SI{2}{\second}. Five
took longer than \SI{1.5}{\s} (including the two over \SI{2}{\s}). Sixteen are
longer than \SI{1}{\s}, \num{155} are longer than \SI{0.5}{\s}, \num{340} are
longer than \SI{0.1}{\s}, and \num{1381} are longer than \SI{0.01}{\s}. The vast
majority (\num{312813}, \SI{99.1}{\percent}) completed \SI{1}{\ms} or less.

The long writes appear to be spaced apart somewhat regularly. This suggests that
they are caused by upkeep that the filesystem has to do periodically, and that
there is no obvious way to avoid them, at least not while storing many small
files. Aggregating objects into \glspl{packfile} as discussed in
\autoref{chunk-then-recombine} might be a better strategy.


\subsubsection{Directory Schemes in Action}

We built the DMV prototype and re-ran the full "file-size" experiments
(\autoref{file-size-exp-desc}) with two different directory schemes. First an
early DMV version (fb2f43d) that used \num{2} hex digits per directory and a
depth of \num{2}, and also with the reference DMV prototype (c9baf3a, as noted
in \autoref{vcs-versions}) that used \num{2} hex digits per directory and a
depth of \num{1}. \autoref{plot-dir-schemes-file-size--c1-time} shows the
initial commit times for both prototype versions, plus Bup for comparison.

\begin{figure}
    \caption{Time to commit one large file, with different object directory
    schemes}
    \label{plot-dir-schemes-file-size--c1-time}
    \centering

    \explainlogsubfig

    \includegraphics{plot-dir-schemes-file-size--c1-time}
\end{figure}

As with the other runs of this experiment, the commit time for files under about
\SI{6}{\mib} is dominated by overhead. Here the depth-\num{2} version of DMV
outperforms the depth-\num{1} version, and especially so at sizes up to
\SI{8}{\kib}, where the depth-\num{2} version completes its commit in under
\SI{5}{\ms}. There is no obvious explanation for such a fast commit, but perhaps
the amount of data being processed is so small that operating system
optimizations are allowing the process to finish before the write is completely
flushed to disk.

Up to \SI{6}{\mib} the depth-\num{2} version has a consistent commit time of
\SIrange{103}{104}{\ms} while the depth-\num{1} version has a consistent commit
time of \SIrange{201}{202}{\ms}. This difference might be caused by additional
work on DMV that occurred made between running the experiments on the
depth-\num{2} version and switching to depth-\num{1}, including refactoring and
adding some statistics collecting code to the verify (\lstinline{fsck})
procedure. None of this should have impacted commit times directly, but it may
have caused changes to the DMV executable's size or layout that made it take
longer to load from disk and start up.

From \SIrange{8}{384}{\mib} there is no noticeable difference between the two
versions of DMV, but at \SI{512}{\mib} and above, the commit results in enough
chunk files that the directory layout starts to make a difference. The slight
trend we noticed in \autoref{plot-seek-times} for more directories to result in
more long write times seems to have a more pronounced effect, and the
depth-\num{2} version starts to lag behind depth-\num{1}. At \SI{768}{\mib}, the
depth-\num{1} version of DMV finally starts to lag behind Bup.

At \SI{768}{\mib} and above, the commit times for both versions of DMV increase
linearly with file size. They appear to have the same slope, with the
depth-\num{1} version shifted down. Bup, though also increasing linearly, does
so with a flatter slope. This is further evidence to suggest that aggregating
objects into \glspl{packfile} is not only less wasteful of disk space but also
faster as the number of objects grows into the millions.

%



\section{Linux I/O Scheduler}

Since the anticipatory I/O scheduler was removed in version
2.6.33~\cite{as_removed_linux_release_notes}, the Linux kernel has included
three different I/O schedulers to choose from~\cite{ioschedulers}:

\begin{description}

    \item[Completely Fair Queueing] The \lstinline{cfq} scheduler is the default
        I/O scheduler as of Linux 2.6.18~\cite{cfq_default_linux_release_notes}.
        It creates a separate queue for each process and handles requests in a
        round, preventing any one process from dominating I/O.

    \item[Deadline] The \lstinline{deadline} scheduler tries to set hard limits
        on wait time for scheduled I/O operations.

    \item[No-op] The \lstinline{noop} scheduler does as little as possible,
        passing requests directly to the device for it to manage. So this is the
        null benchmark for this experiment.

\end{description}

We were curious if the choice of scheduler would have any effect on performance.
In particular, we aimed to document if it might alleviate the high seek times we
were seeing. So we ran extra trials of the VCS scaling experiments using the
\gls{DMV} prototype and different I/O schedulers.

\begin{figure}[]
    \caption{Time for DMV prototype to commit an increasing number of 1KiB files
    to a fresh repository, by I/O scheduler}
    \label{fig:plot-iosched-num-files--c1-time}
    \centering

    \explainlogsubfig

    \includegraphics[]{plot-iosched-num-files--c1-time}
\end{figure}

The results of running the "many files" experiment with different schedulers are
shown in \autoref{fig:plot-iosched-num-files--c1-time}. The I/O scheduler used
made little difference. At \num{100000} files, the average initial \gls{commit}
times were \SI{19.666}{\s} for CFQ, \SI{19.708}{\s} for deadline, and
\SI{19.598}{\s} for noop. The difference between each pair is less than any of
the standard deviations at that number of files: \SI{0.674}{\s},
\SI{0.3153}{\s}, and \SI{0.447}{\s}, respectively.

In retrospect, these results are not surprising. The I/O scheduler deals mainly
with juggling I/O access between different processes on the system, but the
current \gls{DMV} prototype is a single process. A multi-threaded or
multi-process version of the prototype could give the scheduler something to
work with.

%

\chapter{Discussion}


\section{Prototype development}


\section{Prototype performance tuning}

\begin{figure}[]
  \caption{Increasing file size: prototype improvements}
  \label{fig:plot-file-size--c1-time--prototype-improvements}
  \centering
    \includegraphics[]{plot-file-size--c1-time--detail-high-end--prototype-improvements}
\end{figure}

\begin{figure}[p]
  \caption{Real time required to commit many 1KiB files to an empty repository
  (detail)}
  \label{fig:plot-num-files--prototype-improvements--c1-time-detail}
  \centering
    \includegraphics[]{plot-num-files--prototype-improvements--c1-time-detail}
\end{figure}



\section{Possible Applications}

\begin{itemize}

  \item Individual users might use it to maintain a collection of important
    documents, photos, and media, making it easier to keep up-to-date backups
    and to synchronize between computers, mobile devices, and removable drives.

  \item Professional users that work with files too large for traditional
    version control, such as graphic designers, audio engineers, or maybe even
    video editors, might finally be able to adopt a version-control workflow.

  \item Corporate or government users might use it to maintain large archives of
    data with full history.

  \item Far-flung networks with high-latency or rare connectivity, such as
    remote wildlife sensors or Mars rovers, could use it to manage and
    synchronize data.

\end{itemize}



\section{As an Abstraction of Data Space and Time}

We are thinking about data across a number of dimensions:

\begin{description}

  \item[Coverage of data set] How much of the data set is available locally or
    in neighboring nodes?

  \item[Coverage of data history] How much of the data set's history is
    available locally or in neighboring nodes?

  \item[Divergence of versions] How many different branches has this data been
    forked into, and how different are they?

  \item[Number of replicas] How many times is the data replicated across
    neighboring nodes? Is any data in danger of being permanently lost?

  \item[Availability of or distance to replicas] Of the replicas available, how
    available are they? What is the bandwidth of the connection to the
    neighboring nodes? What is the latency?

\end{description}



\section{What the system should not do}

\todo{Update tense}
We want to focus on the problem of storing file history and synchronizing files
between replicas.
We should be careful not to expand across the wrong abstraction boundaries or to
try to do too much.
In particular:

\begin{itemize}

  \item We do not want to reinvent the filesystem. The system should place and
    update files on the filesystem (or offer a filesystem view, such as with
    FUSE) for applications to use normally. Applications such as editors should
    not have to be rewritten to use our system.

  \item We do not want to create new exotic file formats. We believe that the
    classic tree of files is our best chance for long-term storage.

  \item We hope this system could eventually be used as a piece of
    infrastructure on which to build useful applications. It should not
    incorporate functionality that would better be left to an application.

  \item We do not want to deal with media metadata and categorization. Metadata
    and categorization is best left to the applications that produce and consume
    those media formats. We will merely provide the storage.

  \item However, knowledge of media formats might be used for behind-the-scenes
    optimization such as more efficient compression. E.g. recognizing that only
    tag data has changed in an audio file.

\end{itemize}



\section{Limitations}

\towrite{non-programmers (and even some programmers) cannot handle Git
    - Key to usability would be to make as linear a history as possible with
        auto-updates, but that is the job of a separate app
    - Cite redesign of Git\cite{redesign_of_git}
}

\chapter{Related Works}

\perrolfinline{Include systems you tested against as well: Git, Mercurial, Bup}

\section{Distributed storage and synchronization systems}

\paragraph{Camlistore}

Camlistore \cite{camlistore_homepage} is an open-source project to create a
private long-term data storage system for personal users. It allows storage of
diverse types of data and it synchronizes between multiple replicas of the data
store. However, it eschews normal filesystems and creates its own schemas to
store various media.


\paragraph{Dat Data}

Dat \cite{dat_homepage} is an open-source project for publishing and sharing
scientific data sets for research. This project has a lot of overlap with ours,
and several of the core ideas are similar, including breaking files into smaller
chunks, and tracking changes via a Git-like \gls{DAG}. However, their focus is
different. The Dat team is concentrating on publishing research data, and making
that specific task as simple as possible for non-technical researchers who might
not be familiar with version control. By contrast, our project operates at a
lower level of abstraction, offering the full power of version control in a very
general way, exposing and illuminating the complexities rather than trying to
hide them or automate them away.

Where Dat focuses on publishing on the open internet, we focus on ad-hoc
networks and data that may be private. Where Dat has components for automating
peer discovery and consensus, we work at a lower level, trying to perfect and
generalize the storage aspect first. Dat seems to assume that data sets will be
small enough to fit on a typical disk on a workstation, while we want to scale
even larger.

We hope that our system could be used as a base to build something like Dat, but
we intend for \gls{DMV} to be more general than the Dat core.


\paragraph{Eyo}

Eyo \cite{Strauss:2011:EDP:2002181.2002216} is system for storing personal media
and synchronizing it between devices. It utilizes a Git-like content-addressed
\gls{objectstore} behind the scenes, but it works more like a networked filesystem
than version control. It focuses on organizing media by metadata, which requires
agreement on metadata formats, and it requires applications to be rewritten to
access files via Eyo rather than the filesystem, both of which are thorny and
ambitious problems. We focus purely on storage and synchronization.


\paragraph{Git-Annex and Git-Media}

Git-annex \cite{git_annex_homepage} and git-media \cite{git_media_github} are
open-source projects that extend Git with special handling for larger files.
Both store information about the larger files in the normal Git repository and
then store the files themselves in a separate location. Git-media stores all the
larger files in a separate data store which may be remote. Git-annex is more
flexible. Annex files may be spread across several different remote repository
clones or data stores, and git-annex has features for tracking the locations of
annex files in different remote repositories and moving them from one repository
to another. These tracking and distribution features are very similar to our
goals. However, git-annex is not quite as flexible as we aim for with \gls{DMV}.
It considers the large files atomic units, and it does not break them into
smaller chunks for de-duplication. Also, because metadata is processed by Git,
it has the same limitations that Git does. All repositories must have all
metadata, and performance suffers when metadata is too large to fit into RAM.


\paragraph{IPFS: The Interplanetary Filesystem}

IPFS \cite{ipfs_github_main} is an open-source project to create a global
content-addressed filesystem. By its global nature, all files are stored
publicly, in a global network of nodes with global addressing. IPFS should be an
excellent resource for storing published information, but we want \gls{DMV} to
work with private data sets. We want individuals and organizations to be able
manage their own data stores privately on their own hardware.

It should be noted that IPFS does have support for storing private objects by
way of object-level encryption. However, this seems wasteful of disk space,
since small changes in the plain text of a file would completely change the
ciphertext, leaving no way to compress the redundancy.


\paragraph{Kademlia}

Kademlia \cite{Maymounkov2002} is an advanced distributed hash table system that
updates its network topology information as part of normal lookups. It is an
advanced piece\perotto{"piece" not great} of infrastructure, but like other distributed hash tables, it
focuses on system-wide consistency, rather than the version-control paradigm we
are trying to achieve.



\paragraph{Rsync}

\towrite{Rsync}


\section{Content-Addressed Storage and Backup}

\paragraph{Boar}

Boar \cite{boar_homepage} is an open-source project to create a version control
system for large binary files. It is one of the main inspirations for our
project. It stores file versions in a content-addressed way, and provides
de-duplication for large files that only change in small pieces, and it can
truncate history to reclaim disk space. However, Boar retreats to a centralized
version control paradigm, with a central repository that working directories
must connect to to check files in or out. We want to provide the advantages of
Boar in a flexible distributed version control model. Boar also has practical
limitations on repository size and number of files. Repositories are assumed to
fit on one disk volume, and file metadata is assumed to fit into Ram. \gls{DMV}
tries to avoid both of those limitations.


\paragraph{Bup}\label{related_bup}

Bup \cite{bup_homepage} is an open-source file backup system that is based on
Git's repository format. A Bup backup is a valid Git repository and it can be
read by Git, but Bup is a separate program written from scratch to read and
write files to Git's \gls{packfile} format directly, skipping Git's separate
store and \glsdisp{packfile}{pack} steps that use double the disk space. It has
many features that we want for our low-level storage of the \gls{objectstore}.
It breaks files into chunks by rolling checksum, and it has considerations for
metadata that is larger than RAM. However, it is locked into a backup-based
workflow. History is linear and based on clock time of backup. And it assumes
that the whole data set and the whole repository can fit onto one filesystem.


\paragraph{Time Machine}

\towrite{Time Machine}

\chapter{Summary of Contributions}

In this paper, we have examined the cryptographic directed acyclic graph (DAG)
as a data structure for data storage, and the ways that it can be sliced to
shard data across nodes in a distributed system, according to what data is
needed locally at each location.

We have run experiments to probe the scalability limits of existing DAG-based
version control systems and formulated ways of overcoming them, especially by
breaking large files into chunks and ensuring that algorithms do not assume that
whole files can always fit into RAM. We have also bumped up against and
experimented with some of the limitations of the ext4 filesystem for storing
large numbers of small files.

And finally, we have designed a distributed data storage system we call
\newterm{Distributed Media Versioning (DMV)} that expands on the distributed
version control concept to store larger and more diverse data sets, with a high
degree of control over data locality, and an availability to write updates for
any data held locally. Though time constraints prevented us from implementing
the network features we had planned, the DMV prototype has enough functionality
to be tested against existing version control systems and to demonstrate the
addition of new commits to a partial DAG.

\chapter{Conclusion and Summary of Contributions}

\glsreset{DMV}

In this dissertation, we have examined the cryptographic \acrfull{DAG} as a data
structure for data storage, and the ways that it can be sliced to shard data
across nodes in a distributed system, according to what data is needed locally
at each location.

We have performed experiments to probe the scalability limits of existing
\gls{DAG}-based \acrlongpl{DVCS}. We have shown that the maximum size of file
that Git and Mercurial can store is limited by the amount of available memory in
the system. We conclude that this is because those systems calculate deltas of
files to de-duplicate data, and they load the entire file into memory in order
to do so.

We have also rediscovered the limits of the Unix filesystem for storing many
small files. We saw that writing files smaller than the filesystem block size
incurs storage overhead, that splitting files among too many subdirectories
takes \glspl{inode} that are needed to store files, and that jumping between
directories when writing files incurs write-speed penalties.

We have shown that any \gls{VCS} that stores objects as individual files on the
filesystem will encounter these filesystem limitations as they try to scale in
terms of number of files. A \gls{VCS} that also breaks files into chunks will
turn the problem of storing large files into the problem of storing many files,
again encountering these limitations. However, the limitations can be avoided by
aggregating objects into \glspl{packfile} as Bup does.

We have performed experiments on the rolling hash algorithm used for chunking,
and we have determined that adjusting the \gls{divisor} has the most direct
effect on chunk size. Larger \glspl{divisor} result in smaller chunks. And we
have shown that adjusting \gls{windowsize} has a lesser effect on chunk size,
but we reason that smaller \glspl{windowsize} will be able to find smaller
common chunks in the code.

And finally, we have described the idea, architecture, design, and
implementation of a distributed data storage system we call \gls{DMV} that
expands on the \glsdisp{DVCS}{distributed version control} concept to store
larger and more diverse data sets, with a high degree of control over data
locality, and an availability to write updates for any data held locally. Though
time constraints prevented us from implementing the network features we had
planned, the \gls{DMV} prototype has enough functionality to be experimented on
against existing \acrlongpl{DVCS} and to demonstrate the addition of new
\glspl{commit} to a partial \gls{DAG}.

\chapter{Future Work}

The primary goal of \gls{DMV} is to be a general low-level storage platform for
storing and tracking a data set across many nodes. The next step would be to
complete the networking features of the prototype so that it can actually be
used in real-life applications. Applications could layer additional technologies
on top of \gls{DMV} to create interesting systems. For example, a gossip
protocol could spread information about object availability on remote data
stores that are not directly connected, allowing data to be spread and
transfered across far-flung networks. To focus on usability, daemons could
automatically \gls{commit} changes and sync with other nodes, shielding users
from the complexity of branching where possible and trying to present a coherent
current state of the data set. An important optimization would be to create a
virtual filesystem that is a view into a \gls{tree} in the \gls{DMV} repository.
The virtual filesystem could be used as the working directory, eliminating the
wasted disk space of having a second writable copy of all files, and it would
eliminate the copying of those files back into the immutable data store on
\gls{commit}. A virtual filesystem would also make it much easier to write an
auto-\gls{commit} daemon, since writes would have to go through the filesystem.

We look forward to continuing to work on and expand the system. The project is
open source, and development continues online at \dmvurl~.




\backmatter

%\nocite{*}  % Print all references even if they're not used
\printbibliography[heading=bibintoc]



\end{document}
