% Style: Capitalize first letter of all descriptions,
%   but don't capitalize all initialze in acronym unless it's a proper name
%   or if it's not obvious where the acronym is from.


% Dist sys

\newacronym[
    description={Atomicity, consistency, durability, and isolation, the
    guarantees of a traditional database commit},
]{ACID}{ACID}{atomicity, consistency, durability, and isolation}

\newglossaryentry{CAP-theorem}{
    name={CAP-theorem},
    description={The fundamental theorem of distributed systems, that no system
    can simultaneously be consistent (C), available (A), and tolerant of network
    partitions (P)},
    see={partitiondecision},
}

\newglossaryentry{partitiondecision}{
    name={partition decision},
    description={The dilemma faced by a distributed system during a network
    partition: to decrease availability or risk inconsistency},
    see={CAP-theorem},
}

% General Version Control

\newacronym[
    description={Distributed version control system, such as Git, where
    individual repositories can operate independently without having to connect
    to a central repository},
    see={VCS},
]
{DVCS}{DVCS}{distributed version control system}

\newacronym[
    description={Version control system, a program that stores many versions of
    a file or set of files, commonly used to track changes to source code},
    see={SCM},
]
{VCS}{VCS}{version control system}

\newacronym[
    description={Source code manager, a version control system that is designed
    primarily to store source code},
    see={VCS},
]
{SCM}{SCM}{source code manager}


% Specific systems

\newacronym[
    description={Distributed Media Versioning, the new distributed data storage
    platform described and introduced in this paper}
]
{DMV}{DMV}{Distributed Media Versioning}


% VCS architecture

\newglossaryentry{repository}{
    name={repository},
    plural={repositories},
    description={A location where data is stored in a version control system.
    Early systems would have a central repository that clients would check out
    from. In distributed version control, every client is a separate
    repository},
}

\newglossaryentry{objectstore}{
    name={object store},
    description={Content-addressable storage for DAG objects},
    see={DAG},
}

\newglossaryentry{workdir}{
    name={working directory},
    description={A directory where files that are tracked by a version control
    system are actively worked on and edited},
}

\newglossaryentry{branch}{
    name={branch},
    plural={branches},
    description={In a version control system, separate concurrent lines of
    update history},
    see={head,merge},
}

\newglossaryentry{head}{
    name={head},
    description={In a version control system, the most recent revision in a
    branch},
    see={branch},
}

\newglossaryentry{merge}{
    name={merge},
    description={In a version control system, an operation that combines two
    branches and reconciles conflicting changes},
    see={branch},
}


% DAG

\newglossaryentry{contentaddressablestorage}{
    name={content addressable storage},
    description={Storage that stores immutable objects named by a hash of their
    content, which naturally de-duplicates identical objects},
}

\newacronym[
    description={Directed acyclic graph, the type of graph data structure used
    to represent history in many distributed version control systems. Directed
    meaning all the edges have a direction, from one node to another, and
    acyclic meaning that there are no cycles, no paths that revisit any node},
    see={blob,chunkedblob,tree,commit,ref},
]
{DAG}{DAG}{directed acyclic graph}

\newacronym[
    description={Binary large object, a sequence of unstructured binary data. In
    Git and DMV, a DAG object holding file data},
    first={blob (binary large object)},
    see={DAG},
]
{blob}{blob}{binary large object}

\newglossaryentry{chunkedblob}{
    name={chunked blob},
    description={In DMV, a DAG object that is an index of blobs that make up a
    larger blob},
    see={DAG,blob},
}

\newglossaryentry{tree}{
    name={tree},
    description={In Git and DMV, a DAG object representing a particular state of
    a file hierarchy},
    see={DAG},
}

\newglossaryentry{commit}{
    name={commit},
    description={In version control, the operation for storing a particular
    version of the data. Also, the resulting DAG object that represents that
    version in the history},
    see={DAG},
}

\newacronym[
    description={A reference to a commit object in the DAG},
    see={DAG},
]
{ref}{ref}{reference}

\newglossaryentry{packfile}{
    name={pack file},
    description={An object store file format that aggregates many objects in one
    file},
    see={objectstore},
}

\newglossaryentry{filelog}{
    name={filelog},
    description={Mercurial's file format that stores different versions of the
    same file as a base version followed by a series of delta},
}


% Rolling Hash

\newglossaryentry{rollinghash}{
    name={rolling hash},
    first={rolling hash algorithm},
    description={A hash checksum that operates over a moving window of data in a
    byte stream that can be used to find repeating patterns},
    see={windowsize,divisor},
}

\newglossaryentry{windowsize}{
    name={window size},
    symbol={\ensuremath{w}},
    description={In a rolling hash algorithm, the number of previous bytes used
    in the rolling sum},
    see={rollinghash,divisor},
}

\newglossaryentry{divisor}{
    name={divisor},
    symbol={\ensuremath{d}},
    description={In a rolling hash algorithm, the divisor in the modulus
    operation. A chunk boundary is created when the sum of the bytes in the
    window, modulo this divisor, is equal to zero},
    see={rollinghash,windowsize},
}


% filesystems

\newglossaryentry{inode}{
    name={inode},
    description={A data structure in a Unix filesystem that stores file
    metadata. Each filesystem has a fixed number of inodes, which limits the
    total number of files and directories the filesystem can hold},
}
