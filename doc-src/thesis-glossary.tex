% General Version Control

\newacronym[
    description={distributed version control system, such as Git, where
    individual repositories can operate independently without having to connect
    to a central repository},
    see={VCS},
]
{DVCS}{DVCS}{distributed version control system}

\newacronym[
    description={version control system, a program that stores many versions of
    a file or set of files, commonly used to track changes to source code},
    see={SCM},
]
{VCS}{VCS}{version control system}

\newacronym[
    description={source code manager, a version control system that is designed
    primarily to store source code},
    see={VCS},
]
{SCM}{SCM}{source code manager}


% Specific systems

\newacronym[description={Distributed Media Versioning, the new distributed data
storage platform described and introduced in this paper}]
{DMV}{DMV}{Distributed Media Versioning}


% VCS architecture

\newglossaryentry{repository}{
    name={repository},
    plural={repositories},
    description={a location where data is stored in a version control system.
    Early systems would have a central repository that clients would check out
    from. In distributed version control, every client is a separate
    repository},
}

\newglossaryentry{objectstore}{
    name={object store},
    description={content-addressable storage for DAG objects},
    see={DAG},
}

\newglossaryentry{workdir}{
    name={working directory},
    description={a directory where files that are tracked by a version control
    system are actively worked on and edited},
}


% DAG

\newacronym[description={directed acyclic graph, the type of graph data
structure used to represent history in many distributed version control systems.
Directed meaning all the edges have a direction, from one node to another, and
acyclic meaning that there are no cycles, no paths that revisit any node}]
{DAG}{DAG}{directed acyclic graph}

\newacronym[
    description={binary large object, a sequence of unstructured binary data},
    first={blob (binary large object)}
]
{blob}{blob}{binary large object}

\newglossaryentry{chunkedblob}{
    name={chunked blob},
    description={In DMV, an index of blobs that make up a larger blob},
}

\newglossaryentry{tree}{
    name={tree},
    description={In Git and DMV, an object representing a particular state of a
    file hierarchy},
}

\newglossaryentry{commit}{
    name={commit},
    description={In version control, the operation for storing a particular
    version of the data. Also, the resulting object that represents that version
    in the history},
}

\newacronym[
    description={a reference to a commit},
]
{ref}{ref}{reference}

\newglossaryentry{packfile}{
    name={pack file},
    description={an object store file format that aggregates many objects in one
    file},
    see={objectstore},
}

\newglossaryentry{filelog}{
    name={filelog},
    description={Mercurial's file format that stores different versions of the
    same file as a base version followed by a series of delta},
}


% Rolling Hash

\newglossaryentry{rollinghash}{
    name={rolling hash},
    first={rolling hash algorithm},
    description={A hash checksum that operates over a moving window of data in a
    byte stream that can be used to find repeating patterns},
    see={windowsize,divisor},
}

\newglossaryentry{windowsize}{
    name={window size},
    symbol={\ensuremath{w}},
    description={In a rolling hash algorithm, the number of previous bytes used
    in the rolling sum},
    see={rollinghash,divisor},
}

\newglossaryentry{divisor}{
    name={divisor},
    symbol={\ensuremath{d}},
    description={In a rolling hash algorithm, the divisor in the modulus
    operation. A chunk boundary is created when the sum of the bytes in the
    window, modulo this divisor, is equal to zero},
    see={rollinghash,windowsize},
}


\iffalse
newterm{A}
newterm{ACID}
newterm{branches}
newterm{C}
newterm{CAP-theorem}
newterm{heads}
newterm{hg}
newterm{inode}
newterm{merge}
newterm{P}
newterm{refs}
newterm{repository}
\fi
