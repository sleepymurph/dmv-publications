\chapter{VCS Scaling Experiments}
\label{num-files-exp-desc}
\label{file-size-exp-desc}

\perottoinline{Otto does not like the term "test". These are \emph{experiments}.}

We performed experiments to probe the limits of existing \acrlongpl{VCS}, to see
how they would cope with file sizes and file numbers in ranges beyond what would
be expected in a source code tree. We wanted to see how long it would take for
each \gls{VCS} to store that amount of data, how much disk space it used, and
what CPU utilization was like during storage. And since \glspl{VCS} track
changes, so we also wanted to see what would happen when a small subset of that
data was modified and then committed again.

We conducted two major experiments. To measure the effect of file size, we would
\gls{commit} a single file of increasing size to each target \gls{VCS}. And to
measure the effect of file numbers, we would \gls{commit} increasing number of
small (\SI{1}{\kibi\byte}) files to each target \gls{VCS}.



\section{Version Control Systems Evaluated}

We ran each experiment using four different \glspl{VCS}: Git, Mercurial (also
known as \newterm{hg}), Bup, and the \gls{DMV} prototype (specific versions
listed in \autoref{vcs-versions}). As a control we also ran the experiments
using a dummy \gls{VCS} that simply made a copy of the files. We chose Git and
Mercurial because they are the two most widely used \glspl{DVCS} available. We
also included Bup because it uses Git's storage format but, like \gls{DMV}, it
also breaks files into chunks. For discussion on the similarities and
differences between \gls{DMV} and Bup, see \autoref{related_bup}.

\towrite{Include more background about Git, Mercurial, and Bup: Git and Hg
conceptually use same DAG but implemented differently. Bup uses Git's file
format directly but is rewritten.}

\begin{table}
    \caption{Version control systems tested and their versions}
    \label{vcs-versions}
    \centering
    \begin{tabular}{ l l }
        Git & 2.1.4 \\
        Mercurial (hg) & 3.1.2 \\
        Bup & debian/0.25-1 \\
        DMV prototype & exp\_prototype2x1mem (c9baf3a) \\
    \end{tabular}
\end{table}


\section{Procedure}

For each experiment, the procedure for a single trial was as follows:

\begin{tight_enumerate}

    \item Create an empty repository of the target \gls{VCS} in a temporary
        directory

    \item Generate target data to store, either a single file of the target
        size, or the target number of \SI{1}{\kibi\byte} files

    \item \Gls{commit} the target data to the repository, measuring wall clock
        time to \gls{commit}

    \item Verify that the first \gls{commit} exists in the repository, and if
        there was any kind of error, run the repository's integrity check
        operation

    \item Measure the total repository size

    \item Overwrite a fraction of each target file

    \item \Gls{commit} again, measuring wall clock time to \gls{commit}

    \item Verify that the second \gls{commit} exists in the repository, and if
        there was any kind of error, run the repository's integrity check
        operation

    \item Measure the total repository size again

    \item Delete temporary directory and all trial files

\end{tight_enumerate}

We increased file sizes exponentially by powers of two from \SI{1}{\byte} up to
\SI{128}{\gibi\byte}, adding an additional step at \num{1.5} times the base size
at each order of magnitude. For example, starting at \SI{1}{\mebi\byte}, we
would run trails with \SI{1}{\mebi\byte}, \SI{1.5}{\mebi\byte},
\SI{2}{\mebi\byte}, \SI{3}{\mebi\byte}, \SI{4}{\mebi\byte}, \SI{6}{\mebi\byte},
\SI{8}{\mebi\byte}, \SI{12}{\mebi\byte}, and so on.

We increased numbers of files exponentially by powers of ten from one file to
ten million files, adding additional steps at \num{2.5}, \num{5}, and \num{7.5}
times the base number at each order of magnitude. For example, starting at
\num{100} files we would run trials with \num{100}, \num{250}, \num{500},
\num{750}, \num{1000}, \num{2500}, \num{5000}, \num{7500}, \num{10000}, and so
on.

Test data files consisted simply of pseudo-random bytes taken from the operating
system's pseudo-random number generator (\lstinline{/dev/urandom} on Linux).

When updating data files for the second \gls{commit}, we would overwrite a
single contiguous section of each file with new pseudo-random bytes. We would
start one-quarter of the way into the file, and overwrite \num{1/1024}th of the
file's size (or 1 byte if the file was smaller than \SI{1024}{\kibi\byte}). So a
\SI{1}{\mebi\byte} file would have \SI{1}{\kibi\byte} overwritten, a
\SI{1}{\gibi\byte} file would have \SI{1}{\mebi\byte} overwritten, and so on.


\section{Automation and Measurement}

The trials were run via a Python script that would set up, run, and clean up
each trial in a loop, covering the full range of sizes or numbers for a given
\gls{VCS}. The script would measure the wall-clock time duration taken by each
\gls{commit} command and collect CPU utilization metrics. It would also
terminate any individual \gls{VCS} operation that ran longer than five hours.
After \gls{commit} and verification, the script would also measure repository
size.

The script measured the wall-clock time duration for each \gls{commit} by
checking the system time (\lstinline{time.time()}) just before and just after
using Python's \lstinline{subprocess} module to execute the necessary \gls{VCS}
command. CPU utilization was measured by sampling the CPU status lines provided
in Linux's \lstinline{/proc/stat} information. The status lines show a
cumulative count of time slices that the CPU has spent in user mode, system
mode, and waiting for I/O. Like with the time measurements, the script samples
CPU utilization before and after executing a \gls{VCS} command, and then
subtracts to get the number of time slices spent in each state during execution.
We then compare the relative number of time slices in each state to get an idea
of whether the operation is CPU-bound or I/O-bound.

The script measures repository size using the standard Unix disk usage command
(\lstinline{du}) and measures the size of the trial's entire temporary
directory, which includes the generated test data itself along with the
\gls{VCS}'s storage.


\section{Experiment Platform}

We ran the trials on four dedicated computers with no other
load. Each was a typical office desktop with a \SI{3.16}{\giga\hertz}
\num{64}-bit dual-core processor and \SI{8}{\gibi\byte} of RAM, running Debian
version 8.6 ("Jessie"). Each computer had one normal SATA hard disk (spinning
platter, not solid-state), and trials were conducted on a dedicated
\SI{197}{\gibi\byte} LVM partition formatted with the ext4 filesystem. All came
from the same manufacturer with the same specifications and were, for practical purposes,
identical.
Additional details can be found in \autoref{test-machine-specs}.

\begin{table}
    \caption{Test computer specifications}
    \label{test-machine-specs}
    \begin{tabular}{ l l }
        Vendor & Hewlett Packard \\
        CPU & Intel(R) Core(TM)2 Duo CPU     E8500  @ 3.16GHz \\
        RAM & \SI{8}{\gibi\byte} \\
        Hard disk & ATA model ST3250318AS \\
        \midrule
        Operating system & Debian 8.6 ("Jessie") \\
        Kernel & Linux 3.16.0 \\
        \midrule
        Test partition & \SI{197}{\gibi\byte} LVM partition \\
        Filesystem & ext4 \\
        I/O scheduler & cfq (unless otherwise noted) \\
    \end{tabular}
\end{table}

We ran every trial four times, once on each of the test computers, and took the
mean and standard deviation of each time and disk space measurement. However,
because the test machines are practically identical, there was little real variation.
