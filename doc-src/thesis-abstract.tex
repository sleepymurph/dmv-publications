\chapter{Abstract}

\glsunsetall
\glsreset{DMV}

%\perottoinline{1. What's wrong with the world?}

It is still strangely difficult to backup and synchronize data. Cloud computing
solves the problem by centralizing everything and letting someone else handle
the backups. But what about situations with low connectivity or sensitive data?

%\perottoinline{2. Motivation (why bother?)}

For this, software developers have an interesting distributed, decentralized,
and partition-tolerant data storage system right at their fingertips:
\glsdisp{DVCS}{distributed version control}.

%\perottoinline{3. How we did better: idea, architecture, design, implementation.}

Inspired by \glsdisp{DVCS}{distributed version control}, we have researched and
developed a prototype for a scalable high-availability system called \gls{DMV}.
\gls{DMV} expands Git's data model (the \acrshort{DAG}) to allow files to be
broken into more digestible chunks via a \gls{rollinghash}. \gls{DMV} will allow
data to be sharded according to data locality needs, slicing the data set in
space (subset of data with full history), time (subset of history for full data
set), or both. \gls{DMV} \glspl{repository} will be able to work with and update
whatever data they have locally, and synchronize with \glspl{repository} in an
ad-hoc network.

%\perottoinline{4. Experiments; results}

We have performed experiments to probe the scalability limits of existing
version control systems, specifically what happens as file sizes grow ever
larger or as the number of files grow. We found that processing files whole
limits maximum file size to what can fit in RAM, and that storing millions of
objects loose as files with hash-based names incurs severe write speed
penalties. We have observed a system needing \SI{24}{\second} to store a
\SI{6.8}{\kib} files.

%\perottoinline{5. Lessons learned}



%\perottoinline{6. Main conclusions}

We conclude that the key to storing large files is the break them into many
small chunks, and that the key to storing many chunks is to aggregate them into
pack files. And, if one uses a \gls{rollinghash} to split the files by content
and \gls{contentaddressablestorage} to store them, common file chunks are
naturally de-duplicated and many versions of a file can be stored efficiently.
\perotto{Does DMV do this? Do results back it up?}

%


\glsresetall
