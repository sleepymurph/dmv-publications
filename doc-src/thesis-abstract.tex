\chapter{Abstract}

\perottoinline{1-3 sentences for each:
    1. What's wrong with the world?
    2. Motivation (why bother?)
    3. How we did better: idea, architecture, design, implementation.
    4. Experiments; results
    5. Lessons learned
    6. Main conclusions
}

As we ponder the challenges of big data, we programmers and computer scientists
often make use of humble tools that elegantly store small data: version control
systems. The current wave of version control systems such as Git and Mercurial
are, in a sense, small-scale distributed systems. Interestingly, with
distributed version control, network partitions are not a failure mode or an
obstacle to be overcome but the default mode of operation. This paper argues
that the distributed version control concept has potential applications to large
scale distributed systems. And to that end, it explores the scalability limits
of existing version control systems and presents a design and prototype for a
new, more-scalable system that we call \gls{DMV}.
