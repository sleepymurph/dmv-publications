\chapter{Performance Tuning Experiments}

\section{Tuning Object Store Directory Layout}\label{dir-experiment}

During initial runs of the "many files" experiment
(\autoref{num-files-exp-desc}), we would often notice the disk being reported as
full even though the total bytes used was less than the capacity of the disk
partition. This had to do with the way we divided the test files into
directories, and how the version control systems stored their objects (see
\autoref{dir-impl} for DMV's object storage scheme). Since every directory
requires an inode on an ext4 filesystem, a storage scheme that creates too many
subdirectories will use up all the available inodes with directories rather than
data.

\todo{Note mysterious seek times}
\label{seek-times-results}

\paragraph{Procedure}

To test the effects of different object storage schemes, we created a new
\SI{100}{\mebi\byte} partition on a test computer, and then generated a series
of pseudo-random files of \SI{4}{\kibi\byte} each until the disk was reported
full. For each file, we would give it a pseudo-random name that resembled an
SHA-1 hash, and store it according to the object storage scheme under test. We
increased a counter each time we created a file, and another each time we
created a new directory.

We followed the basic scheme of taking leading hex digits of the SHA-1 hash to
form directories. We varied the number of directories taken (depth) and the
number of hex digits per directory (see
\autoref{sample-directory-scheme-variations} for examples). We tried depths from
\num{0} to \num{6} and digits per directory from \num{0} to \num{16}, discarding
combinations that did not make sense, such as combinations involving \num{0} and
another number (which would all simply be undivided), or those that required
more than the \num{40} hex digits of a \num{160}-byte SHA-1 hash.

\begin{table}[]
    \caption{Sample object store directory variations}
    \label{sample-directory-scheme-variations}
    \centering
    \begin{tabular}{l l l}
        Hex digits & Depth & Example \\
        \midrule
        0 & 0 & \lstinline{03d37679d1fab86e5286decd6cd2a94efcfe083f} \\
        1 & 1 & \lstinline{7/9332ca7ce9163f78e3c115a2173bd8fd853d286} \\
        1 & 3 & \lstinline{6/8/c/40e64f3e74e6ebefdcf2f5f30fb8da004792c} \\
        2 & 1 & \lstinline{9f/4ec22c3e0289b29eefefe4728dece14e67e6ac} \\
        2 & 2 & \lstinline{dd/52/bcccff156a179cdac0793ef049039372d8a1} \\
        3 & 1 & \lstinline{cc5/199084d70f7c5ba325a240e1927579ee24bb1} \\
        3 & 4 & \lstinline{472/e98/e88/0b1/c5905065c70cbe806361d32f6429} \\
        4 & 3 & \lstinline{1ed2/bd51/01fe/5b23763e8c76852739f59201280f} \\
    \end{tabular}
\end{table}

\paragraph{Environment}

Like the "many files" experiment, this was automated as a Python script and run
on one of the dedicated test computers used for that experiment (specs shown in
\autoref{test-machine-specs}). However, rather than spending hours to fill the
\SI{197}{\gibi\byte} partition used for the other experiments, this experiment
used a new \SI{100}{\mebi\byte} LVM partition.

\subsection{Results}

\begin{figure}[]
    \caption{Overwhelmed by subdirectories}
    \label{fig:plot-filesystem-limits--directory-takeover}
    \centering

    Files vs. directories created when organizing an object directory by the
    given scheme

    \includegraphics[]{plot-filesystem-limits--directory-takeover}
\end{figure}

\autoref{fig:plot-filesystem-limits--directory-takeover} shows directories
overtaking files as nesting goes deeper.


\begin{figure}[]
    \caption{Unusually high write times}
    \label{plot-seek-times}
    \centering

    \SI{4}{\kib} files that took longer than \SI{1}{\ms} to write, plotted
    according to the number of files and directories on the disk already, and
    colored by the time it took to write the file

    \includegraphics[]{plot-seek-times}
\end{figure}

\autoref{plot-seek-times} shows unusual write times by number of files and
directories already present on the filesystem.


\subsection{Observations}

\begin{itemize}

  \item In hindsight, it should have been easy to see that the number of
    directories would grow exponentially.

\end{itemize}

%



\section{Effect of Different Linux I/O Schedulers}

Since the anticipatory I/O scheduler was removed in version 2.6.33
\cite{as_removed_linux_release_notes}, the Linux kernel has included three
different I/O schedulers to choose from\cite{ioschedulers}:

\begin{description}

    \item[Completely Fair Queueing] The \lstinline{cfq} scheduler is the default
        I/O scheduler as of Linux 2.6.18 \cite{cfq_default_linux_release_notes}.
        It creates a separate queue for each process and handles requests in a
        round, preventing any one process from dominating I/O.

    \item[Deadline] The \lstinline{deadline} scheduler tries to set hard limits
        on wait time for scheduled I/O operations.

    \item[No-op] The \lstinline{noop} scheduler does as little as possible,
        passing requests directly to the device for it to manage.

\end{description}

We were curious if the choice of scheduler would have any effect on performance.
In particular, we hoped it might alleviate the high seek times we were seeing.
So we ran extra trials of our VCS scaling experiments using the DMV prototype
and different I/O schedulers.

\begin{figure}[]
    \caption{Time for DMV prototype to commit many 1KiB files to a fresh
    repository, by I/O scheduler}
    \label{fig:plot-iosched-num-files--c1-time}
    \centering

    The top left graph shows the whole data set on a logarithmic scale. The
    other two show an increasingly zoomed-in linear scale, to show how small the
    difference is.

    \includegraphics[]{plot-iosched-num-files--c1-time}
\end{figure}

The results of running the "many files" experiment with different schedulers are
shown in \autoref{fig:plot-iosched-num-files--c1-time}. In our case, the I/O
scheduler used made little difference. At \num{100000} files, the average
initial commit times were \SI{19.666}{\s} for CFQ, \SI{19.708}{\s} for deadline,
and \SI{19.598}{\s} for noop. The difference between each pair is less than any
of the standard deviations at that number of files: \SI{0.674}{\s},
\SI{0.3153}{\s}, and \SI{0.447}{\s}, respectively.

In retrospect, these results are not surprising. The I/O scheduler deals mainly
with juggling I/O access between different processes on the system, but the DMV
prototype is a single process. Perhaps a multi-threaded or multi-process version
of the prototype could give the scheduler something to work with, but a better
approach would be to combine objects into pack files that can be written
sequentially, as discussed in \autoref{chunk-then-recombine}.

%



\section{Tuning Chunk Size}\label{rolling-hash-exp}

The algorithm used to divide files into chunks (described in
\autoref{chunking-algoritm}) involves moving a window across the data, and
setting a chunk boundary where the sum of the bytes in that window is evenly
divisible by a given number. We ran an experiment to determine the effects of
these two parameters on chunk size.

\paragraph{Procedure}

For each combination of window size and divisor, we would run the rolling hash
algorithm on a stream of pseudo-random bytes until it had identified \num{100}
chunks. Then we would compute the mean and standard deviation of the chunk
sizes.

We used window sizes in powers of two from \SI{128}{\byte} ($2^7$) to
\SI{256}{\kibi\byte} ($2^{18}$), and divisors in powers of two from \num{256}
($2^8$) to \SI{128}{\kibi\relax} ($2^{17}$).

The pseudo-random number generator used was an xorshift RNG\cite{xorshift_rng}.
The experiment itself was automated as a unit test in the DMV prototype's Rust
code.

\paragraph{Environment}

Because this experiment measures only the output of calculations, the
environment in which it is run should make no difference in the outcome. In
fact, if the xorshift RNG is given the same initial seed value, the resulting
random byte stream will be identical, which will lead to an identical sequence
of chunks, which will lead to an identical average chunk size. This experiment
is deterministic.


\subsection{Results}

\begin{figure}[]
  \caption{Mean chunk size}
  \label{fig:plot-rolling-hash}
  \centering
    \includegraphics[]{plot-rolling-hash}
\end{figure}

\autoref{fig:plot-rolling-hash} shows the mean chunk size.

\subsection{Observations}

\begin{itemize}

  \item The mean chunk size is approximately the sum of the window size and
    match parameter.

  \item When the match parameter is less than the window size, the standard
    deviation is approximately the match parameter.

\end{itemize}


%
