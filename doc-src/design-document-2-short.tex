\documentclass[a4paper]{article}
% vim: set ts=2 sts=2 sw=2 :

\usepackage{graphicx}
\usepackage[utf8]{inputenc}

% Use a blank space between paragraphs instead of an indent.
\usepackage[parfill]{parskip}

% Source code listings
\usepackage{listings}
\lstset{basicstyle=\ttfamily\footnotesize,breaklines=false}

\usepackage{url}
% To actually link URLs in the PDF.
%
% The hyperref docs recommend declaring it after the other \usepackage
% declarations, because it has to redefine several commands to work properly,
% and other later redefinitions might interfere.
% Remove the hidelinks parameter to get (ugly) visual highlights around the
% links.
\usepackage[hidelinks]{hyperref}

% Create a short macro for newly-defined key terms.
\newcommand{\newterm}{\textit}


\begin{document}

\title{Master's Thesis Design Document \\ (Terse Version)}
\author{Michael Murphy}
\maketitle

\section{Idea}\label{idea}

A low-level distributed data storage platform that makes data location and
availability explicit, rather than trying to hide or abstract them away.

The target use for this system is to spread a data set across many compute
nodes, and to allow the sharding and replication of the data set to be
customizable according to the location and availability of the nodes, and what
data is needed where. The ultimate goal is to track the scattered data as a
coherent data whole, and to allow the end user or client application to
visualize and manage what data is stored on what nodes.

The system aims to be a low-level service, providing information about the data
and tools to manage it, but ultimately leaving decisions to the end user or
client application.

We assume that the data set and all nodes in the system are controlled by the
user.

The target data sets for this system include collections of media files (binary
files ranging in size from hundreds of kilobytes to several gigabytes) numbering
in the hundreds to millions. The system will not attempt to understand the
internals of different media formats and will treat all files as opaque blobs.
Such media data suggests a usage pattern where files are more often created or
moved than they are updated in-place. Also, we assume writes will be infrequent
compared to traditional online transactional databases (seconds to minutes, or
longer, between updates).

The system will be designed for high availability in the face of connections
that are intermittent, high-latency, expensive, or otherwise restricted. At each
node, the portion of data that is stored on that node will be available for
reading and writing by local applications. Metadata about the non-local portions
of the data set will be available as well, along with hints about the location
of that data. Clients can use that information to schedule transfers and
updates. Consistency will be resolved during updates. Conflicting versions will
be presented to the client application to resolve.


\section{Architecture}\label{architecture}

Data will be distributed across several \newterm{stores}, which will each hold a
portion of the data set. Stores can be diverse, from mobile devices to storage
clusters. Distribution of the data will be determined by the end user or client
application according to available resources and the needs of the user.

The data store network will be decentralized and unstructured (Figure
\ref{fig:dia_architecture}). Stores can be connected to other stores in an ad-hoc
manner, with a user- or application-defined topography. The number of
connections will not be strictly limited, but extreme scalability of individual
connections is not a priority. We are aiming for one, to tens, to perhaps
hundreds of connections.

\begin{figure}[h]
  \caption{Stores in an ad-hoc network}
  \label{fig:dia_architecture}
  \centering
    \includegraphics[width=0.95\textwidth]{dia_architecture}
\end{figure}

Each store will be autonomous, and local applications will be able to access and
perform processing on any data that is stored locally. Data not currently
available can be listed and requested. The listing will give hints about the
estimated availability of each piece of data, with metrics including connection
uptime, latency, and bandwidth to stores that have copies of the data. This
metadata will naturally fall out of date between syncs with connected nodes. The
age of that data can be part of the hint.

We are envisioning a filesystem with an \lstinline{ls} command that lists these
hints as part of its output (in this example presented as a single estimated
time to retrieve):

\begin{lstlisting}
    -rw-r--r-- 1 user user  121306 Oct 21 18:28 local   filex
    -rw-r--r-- 1 user user   25475 Oct 21 17:52 100ms   filey
    -rw-r--r-- 1 user user   32031 Oct 21 17:52 20min   filez
    -rw-r--r-- 1 user user   74968 Oct 18 17:12 missing filexx
    -rw-r--r-- 1 user user   83977 Sep 22 21:23 unknown fileyy
\end{lstlisting}

A primary goal for the system is to detect storage errors and never lose data
inadvertently. Local data stores will detect corruption, and data loss will be
prevented by replication between stores. However, data can be explicitly removed
from the data set. Data can be removed from the local store, relying on remote
stores to keep replicas, though it will be up the end user/application to ensure
that there are enough copies of the data to ensure its durability. Number of
known replicas will be one of the metrics tracked by the system.


\section{Design}\label{design}

The data set will be a traditional hierarchical file structure, and it will be
presented as a normal filesystem. We do not want to reinvent the filesystem, and
we do not want to have to modify local applications to work with the data,
though applications can be modified to be aware of and to work with the system.

The storage will be modeled on version control systems, specifically Git and
it's directed acyclic graph data structure (DAG). There will be a
content-addressable blob store to de-duplicate and store data in a DAG
structure, and a working tree where the files can be read and written by normal
applications.

Like in a version control system, previous versions of the data will be kept,
along with metadata about the history of changes. Snapshots of the data will be
explicitly \newterm{committed} to the system by the user/application. The chain
of commit history can diverge into branches to be merged later, and each
individual store is also naturally a branch. Syncs and merges with connected
stores will be explicitly initiated.

Individual stores are not required to store all blobs in the DAG. Some blobs may
not be available on a given store, and history may be deliberately pruned to
save space. Algorithms and client applications must work with the blobs they
have locally and the hints about the availability of remote blobs.


\section{Implementation}\label{implementation}

% TODO: DAG diagram

Files over a certain threshold will be broken into smaller chunks in the blob
store, using a content-based splitting technique such as Rabin fingerprinting.


\section{Possible extensions}

The goal of the system is to be a low-level platform for tracking a single data
set that is distributed across many nodes. Following the version-control
paradigm, most operations will be explicit. However, it will be possible to
write applications on top of this system that automate more operations.

\begin{itemize}

  \item A gossip protocol could spread information about availability of remote
    stores that are not directly connected.

  \item Daemons could auto-commit and auto-sync with other stores.

  \item To save storage space, the working directory can be a virtual filesystem
    that is a view into the object database.

\end{itemize}


% \nocite{*}  % Print all references even if they're not used
% \bibliographystyle{plain}
% \bibliography{research}

\end{document}
