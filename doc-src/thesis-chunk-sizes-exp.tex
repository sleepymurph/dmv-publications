\section{Tuning Chunk Size}\label{rolling-hash-exp}

The algorithm used to divide files into chunks (described in
\autoref{chunking-algorithm}) involves moving a window across the data, and
setting a chunk boundary where the sum of the bytes in that window is evenly
divisible by a given number. We ran an experiment to determine the effects of
these two parameters on chunk size.

\subsection{Procedure}

For each combination of window size and divisor, we would run the
\glsfirst{rollinghash} on a stream of pseudo-random bytes until it had
identified \num{100} chunks. Then we would compute the mean and standard
deviation of the chunk sizes.

We used window sizes in powers of two from \SI{128}{\byte} ($2^7$) to
\SI{256}{\kibi\byte} ($2^{18}$), and divisors in powers of two from \num{256}
($2^8$) to \SI{128}{\kibi\relax} ($2^{17}$).

The pseudo-random number generator used was an xorshift RNG\cite{xorshift_rng}.
The experiment itself was automated as a unit test in the \gls{DMV} prototype's
Rust code.

\subsection{Environment}

Because this experiment measures only the output of calculations, the
environment in which it is run should make no difference in the outcome. In
fact, if the xorshift RNG is given the same initial seed value, the resulting
random byte stream will be identical, which will lead to an identical sequence
of chunks, which will lead to an identical average chunk size. This experiment
is deterministic.

\subsection{Window Reset Bug}

During development, we noticed an error in the DMV \gls{rollinghash}
implementation: it would reset the window after every chunk. DMV's
\gls{rollinghash} implementation waits until the window is full before marking
any chunks, so resetting the window after each chunk would force it to fill
again before marking the next chunk. This effectively created a minimum chunk
size, the window size. We ran the experiment both with and without the bug to
see how it would affect chunk size.

\subsection{Results}
\label{results-rolling-hash}

\begin{figure}[p]
    \caption{Mean chunk size}
    \label{plot-rolling-hash}
    \centering
    \includegraphics[]{plot-rolling-hash}
\end{figure}

\begin{figure}[p]
    \caption{Mean chunk size, with reset bug}
    \label{plot-rolling-hash-with-reset}
    \centering
    \includegraphics[]{plot-rolling-hash-with-reset}
\end{figure}

% Analyze:
%
% paste -d'|' \
%       ../data/exp--rolling-hash/2017-05-09--rolling-hash--reuse-rand--no-reset.txt \
%       ../data/exp--rolling-hash/2017-05-09--rolling-hash--reuse-rand--reset.txt \
%   | tail -n +2 \
%   | awk -e '{printf "%6d   %6d | %8.1f   %8.1f   %8.1f | %7.1f   %7.1f   %7.1f \\\\\n", $1, $2, $8, $3, ($8+$1-$3), $9, $4, ($9-$4)}' | less

% Print for table:
%
% paste -d'|' \
%       ../data/exp--rolling-hash/2017-05-09--rolling-hash--reuse-rand--no-reset.txt \
%       ../data/exp--rolling-hash/2017-05-09--rolling-hash--reuse-rand--reset.txt \
%   | tail -n +2 \
%   | awk -e '{printf "%6d & %6d & %8.1f & %8.1f & %8.1f & %7.1f & %7.1f & %7.1f \\\\\n", $1, $2, $8, $3, ($8-$3), $9, $4, ($9-$4)}' | less

\begin{table}
    \caption{Chunk sizes for a window size of 4096}
    \label{chunk-size-table}
    \begin{tabular}{r | r r r | r r r}
        & \multicolumn{2}{c}{Mean} & & \multicolumn{2}{c}{Std.} & \\
        Divisor & As des. & w/reset & Diff & As des. & w/reset & Diff \\
        \midrule
    256 &    300.8 &   4363.7 &  -4062.9 &   476.3 &   218.1 &   258.2 \\
    512 &    583.9 &   4622.9 &  -4039.0 &   718.9 &   512.6 &   206.3 \\
   1024 &   1069.1 &   5185.6 &  -4116.5 &  1176.7 &  1003.7 &   173.0 \\
   2048 &   1932.0 &   6242.9 &  -4310.9 &  2072.1 &  2179.4 &  -107.3 \\
   4096 &   4255.8 &   8629.8 &  -4374.0 &  5499.7 &  5449.4 &    50.3 \\
   8192 &   8324.8 &  13735.5 &  -5410.7 &  9506.4 &  9793.4 &  -287.0 \\
  16384 &  13304.2 &  20639.0 &  -7334.8 & 13263.2 & 16862.1 & -3598.9 \\
  32768 &  13304.2 &  20639.0 &  -7334.8 & 13263.2 & 16862.1 & -3598.9 \\
    \end{tabular}
\end{table}

\autoref{plot-rolling-hash} shows the mean chunk sizes with the algorithm as
designed, and \autoref{plot-rolling-hash-with-reset} shows them with the window
reset bug. \autoref{chunk-size-table} shows specific values for a window size of
4096, both as designed ("as des.") and with the reset bug ("w/reset").

\paragraph{Increasing with divisor}

The overall trend is for chunk size to increase with both window size and
divisor. As designed, though, the effect of the window size is much smaller, and
the chunk size varies much more with the divisor. We did not perform a rigorous
mathematical analysis of the algorithm, but this makes intuitive sense. The
larger the divisor, the fewer numbers divide by it evenly, and so there is a
lower probability that a given byte will push the sum to an even multiple. With
a larger window, each bytes contribution to the sum is smaller, and that does
have some effect on the probability, but the effect is less obvious.

Also, the first chunk must always be at least the size of the window, because
the window has to fill before a chunk boundary can be triggered. Each value is
the mean of 100 chunk sizes, and the first must be larger than the window. After
the window is filled, the probability of triggering a chunk is the same, so the
first chunk is \glssymbol{windowsize} bytes larger than it would otherwise be.
This would in turn increase the mean by $\frac{\glssymbol{windowsize}}{100}$.

The standard deviation similarly tracks the divisor, with less effect from the
window size. Interestingly, the standard deviation is often larger than the
mean. So the chunk sizes are varying wildly. This also makes sense. Once the
window is full, any byte could potentially trigger a chunk.

The window size's greater effect is that it sets the limit for the smallest
pieces of identical content that can be identified. A common \SI{1}{\kib} string
will be lost inside a \SI{4}{\kib} window. Chunks may be smaller than the
window, but that indicates that the common data string started inside of the
previous chunk. So for more effective de-duplication, it is probably better to
set a smaller window size and then modify the divisor to tune the desired chunk
size.



\paragraph{Reset bug makes every chunk a first chunk}

With the reset bug, standard deviations are similar to those without it, but
mean chunk size is much larger. Interestingly, if one compares the mean chunk
size with the reset to the mean chunk size without it for the same window size
and divisor, it is often larger by an amount close to the window size itself.
This also makes sense. The reset makes every chunk a first chunk, which is
\glssymbol{windowsize} bytes larger than it would otherwise be. So the
window-reset bug's effect was to shift the mean chunk size up by the window
size.

\paragraph{Flat tops}

There is another interesting effect in cases with small window sizes and large
divisors. At some point, increasing the divisor has no more effect. A doubled
divisor will yield exactly the same chunk size. This phenomenon can be seen in
\autoref{chunk-size-table} where the values are identical for divisors of
\num{16384} and \num{32768}, and in \autoref{plot-rolling-hash} and
\autoref{plot-rolling-hash-with-reset}, where the left-most histogram clusters
flatten out on their right sides. This has to do with the fact that the divisors
are powers of two, and that the same pseudorandom stream is used for every
trial. When the divisor doubles, half of the values that would trigger chunk
boundaries disappear, but half of them are exactly the same. For those trials,
it just so happens that the first \num{100} chunk boundary trigger values that
the algorithm encounters in the byte stream are the ones that the divisors have
in common.

%


\subsubsection{Chunk Sizes in Action}

\begin{table}
    \caption{DMV versions examined with different rolling hash configurations}
    \label{chunk-size-versions}
    \centering

    \begin{tabular}{l r c l c c}
        Version & Window size & Divisor & Window reset & Mean & Std. \\
        \midrule
        c9baf3a & \SI{4}{\kib} & \SI{16}{\kibi\relax} & Yes & \SI{20.1}{\kib} & \SI{16.5}{\kib} \\
        b134cca & \SI{32}{\kib} & \SI{16}{\kibi\relax} & Yes & \SI{52.5}{\kib} & \SI{21.2}{\kib} \\
        a660730 & \SI{4}{\kib} & \SI{16}{\kibi\relax} & No & \SI{13.0}{\kib} & \SI{13.0}{\kib} \\
        3e599e3 & \SI{32}{\kib} & \SI{16}{\kibi\relax} & No & \SI{18.7}{\kib} & \SI{22.0}{\kib} \\ \end{tabular}

    \medskip

    The "Mean" and "Std." columns show the mean chunk size and its standard
    deviation for that configuration, as reported by the rolling-hash experiment
    (\autoref{plot-rolling-hash}).

\end{table}

\begin{figure}[]
    \caption{Chunk sizes in action}
    \label{plot-chunk-size-file-size--c1-time}
    \centering

    \explainlogsubfig

    \includegraphics[]{plot-chunk-size-file-size--c1-time}
\end{figure}

We also ran the file-size experiments (\autoref{file-size-exp-desc}) with DMV
versions that used four different \gls{rollinghash} configurations. The
reference DMV version used in other experiments (c9baf3a) had a window size of
\SI{4}{\kib} and a divisor of \SI{16}{\kibi\relax}, chosen arbitrarily. It also
has the window-reset bug. Later, in order to get a larger chunk size, we
increased the window size to \SI{32}{\kib} but left the divisor the same
(version b134cca). It was after that that we discovered the window-reset bug and
fixed it. We then re-ran the file-size experiments with both window sizes with
the bug fixed (\SI{4}{\kib} in version a660730, \SI{32}{\kib} in version
3e599e3). These versions and their mean chunk sizes are listed in
\autoref{chunk-size-versions}, and the results are shown in
\autoref{plot-chunk-size-file-size--c1-time}.

The file-size experiments are sensitive to chunk size because chunk size
determines the number of chunk objects, and the number of objects on the disk
affects write speed. Larger chunks means fewer objects which means less
filesystem overhead, as we found in the object-directory-layout experiment
(\autoref{dir-experiment}).

The version with a \SI{4}{\kib} window and no reset bug has the smallest mean
chunk size at \SI{13.0}{\kib} and it is clearly the slowest. Next, the
\SI{4}{\kib} window with reset and the \SI{32}{\kib} window without reset have
similar mean chunk sizes at \SI{20.1}{\kib} and \SI{18.7}{\kib} respectively,
and they have similar times. Finally, the \SI{32}{\kib} window with reset has
the largest mean chunk size at \SI{52.5}{\kib} and it is clearly the fastest.
However, we have to assume that as file sizes got even larger that it would also
succumb to the many-files problem.

This is further evidence of the importance of aggregating objects into
\glspl{packfile}. If files are packed, the number of objects becomes less of a
concern, and then so does chunk size. The effects of chunk size then would be
more subtle, smaller chunk sizes would let data be de-duplicated with a finer
granularity, better compressing the data. However, they would incur some
overhead by way of larger \gls{chunkedblob} objects because there would be more
chunks to index. We would have to perform additional experiments that used
real-world data to examine that trade-off.

%
