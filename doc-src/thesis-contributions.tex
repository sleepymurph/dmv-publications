\chapter{Summary of Contributions}

In this paper, we have examined the cryptographic directed acyclic graph (DAG)
as a data structure for data storage, and the ways that it can be sliced to
shard data across nodes in a distributed system, according to what data is
needed locally at each location.

We have run experiments to probe the scalability limits of existing DAG-based
version control systems and formulated ways of overcoming them, especially by
breaking large files into chunks and ensuring that algorithms do not assume that
whole files can always fit into RAM. We have also bumped up against and
experimented with some of the limitations of the ext4 filesystem for storing
large numbers of small files.

And finally, we have designed a distributed data storage system we call
\newterm{Distributed Media Versioning (DMV)} that expands on the distributed
version control concept to store larger and more diverse data sets, with a high
degree of control over data locality, and an availability to write updates for
any data held locally. Though time constraints prevented us from implementing
the network features we had planned, the DMV prototype has enough functionality
to be tested against existing version control systems and to demonstrate the
addition of new commits to a partial DAG.
