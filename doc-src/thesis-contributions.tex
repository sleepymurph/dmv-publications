\chapter{Contributions, Conclusions, and Future Work}

\glsreset{DMV}

\section{Summary of Contributions}

In this paper, we have examined the cryptographic \acrfull{DAG} as a data
structure for data storage, and the ways that it can be sliced to shard data
across nodes in a distributed system, according to what data is needed locally
at each location.

We have performed experiments to probe the scalability limits of existing
\gls{DAG}-based \acrlongpl{DVCS} and formulated ways of overcoming them,
especially by breaking large files into chunks and ensuring that algorithms do
not assume that whole files can always fit into RAM. We have also bumped up
against and experimented with some of the limitations of the ext4 filesystem for
storing large numbers of small files.

And finally, we have described the idea, architecture, design, and
implementation of a distributed data storage system we call \gls{DMV} that
expands on the \glsdisp{DVCS}{distributed version control} concept to store
larger and more diverse data sets, with a high degree of control over data
locality, and an availability to write updates for any data held locally. Though
time constraints prevented us from implementing the network features we had
planned, the \gls{DMV} prototype has enough functionality to be experimented on
against existing \acrlongpl{DVCS} and to demonstrate the addition of new
\glspl{commit} to a partial \gls{DAG}.
