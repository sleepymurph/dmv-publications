\section{Results: File Size}

\subsection{File Size Limits}
\label{file-size-limits-results}

Both Git and Mercurial had limits to the size of file they could store
successfully. With a \SI{2}{\gibi\byte} file, Mercurial would issue a warning,
saying "up to 6442 MB of RAM may be required to manage this file," but the
\gls{commit} would be stored successfully. With a file \SI{4}{\gibi\byte} or
larger, Mercurial's \gls{commit} operation would exit with an error message and
the \gls{commit} would not be stored. However, the repository would be left in a
consistent empty state. The atomicity of the \gls{commit} operation held.

Git's behavior was more erratic. Starting with a file \SI{12}{\gibi\byte} and
larger, Git's \gls{commit} operation would exit with an error code, reporting a
fatal out-of-memory error that \lstinline{malloc} failed to allocate
\SI{12}{\gibi\byte}. However, the \gls{commit} would be successfully stored with
no consistency errors in the repository reported by \lstinline{git fsck}.
Starting at \SI{24}{\gibi\byte}, the \gls{commit} operation would report the
same error, the \gls{commit} would still be written, but then the 
\lstinline{git fsck} integrity check would also exit with an error code.
However, the error it reported in its output was a similar "fatal" error as
Git's \lstinline{malloc} error as the \gls{commit} operation, and it did not
report any actual integrity errors in the repository.

So to test the \gls{commit}, we extracted the \SI{24}{\gibi\byte} file from the
repository and compared it. It was the same as the original. So the \gls{commit}
was intact. We also deliberately corrupted the Git pack file that stored the
\SI{24}{\gibi\byte} by overwriting one \SI{1}{\mebi\byte} block at an offset of
\SI{22}{\gibi\byte} with new pseudo-random data. When we ran the fsck command
again with the corrupted repository, it reported the integrity error, but it did
not report the \lstinline{malloc} error that it did before. So for the other
trials with larger files, we assumed that Git's errors were a false alarm and
allowed the trail to continue.

\towrite{After corruption, answer instantaneous rather than 7m before. Some
sorcery detected tampering, even though we fixed the mod time?}

Our DMV prototype was able to store a file up to \SI{64}{\gibi\byte} in size,
but time became a limiting factor as file size increased. At
\SI{96}{\gibi\byte}, our experiment script timed out and terminated the
\gls{commit} after five and a half hours.\perotto{Bottleneck was...?}

Our test environment itself limited the largest file stored by any VCS to
\SI{96}{\gibi\byte}. Any larger and it was simply impossible to store a second
copy of the file on our \SI{197}{\gibi\byte} test partition. Bup was able to
store a \SI{96}{\gibi\byte} file with no errors in just under two hours. Git
could also store such a large file, but one must ignore the false-alarm "fatal"
errors being reported by the user interface.

These findings are summarized in \autoref{file-sizes-table} and
\autoref{vcs-size-limits-table}.

\begin{table}
    \caption{Observations as file size increases}
    \label{file-sizes-table}
    \centering
    \begin{tabular}{r l}
        \SI{1.5}{\gibi\byte} & Largest successful commit with Mercurial \\
        \SI{2}{\gibi\byte} & Mercurial commit rejected \\
        \SI{8}{\gibi\byte} & Largest successful commit with Git \\
        \SI{12}{\gibi\byte} & Git false-alarm errors begin, but commit still intact \\
        \SI{16}{\gibi\byte} & Largest successful Git fsck command \\
        \SI{24}{\gibi\byte} & Git false-alarm errors begin during fsck, but commit still intact \\
        \SI{64}{\gibi\byte} & Largest successful DMV commit \\
        \SI{96}{\gibi\byte} & DMV timeout after \SI{5.5}{\hour} \\
                            & Last successful commit with Bup (and Git, ignoring false-alarm errors) \\
        \SI{128}{\gibi\byte} & All fail due to size of test partition \\
    \end{tabular}
\end{table}

\begin{table}
    \caption{Effective size limits for VCSs tested}
    \label{vcs-size-limits-table}
    \centering
    \begin{tabular}{l l}

        Git & Commit intact at all sizes, UI reports errors at \SI{12}{\gibi\byte} and larger \\

        Mercurial & Commit rejected at \SI{2}{\gibi\byte} and larger \\

        Bup & Successful commits at all sizes tested, up to \SI{96}{\gibi\byte} \\

        DMV & Successful commits up to \SI{64}{\gibi\byte}, timeout at
        \SI{5.5}{\hour} during \SI{96}{\gibi\byte} trial

    \end{tabular}
\end{table}

%


\subsection{Time for File-Size Initial Commit}

\autoref{fig:plot-file-size--c1-time} shows the time required for the initial
\gls{commit}, adding a single file of the given size to a fresh repository. Over
all, the trend is clear and unsurprising: \gls{commit} time increases with file
size.\perotto{Increases how? Linear? Exponential? Power of 2?}

At file sizes below around \SI{2}{\mebi\byte}, \gls{commit} times are dominated
by overhead-- around \SI{5}{\ms} for Git, \SI{100}{\ms} for Mercurial,
\SI{180}{\ms} for DMV, and \SI{900}{\ms} for Bup, vs only \SI{2}{\ms} for the
copy.

Bup, after starting with the highest overhead, goes on to have the fastest
inital \gls{commit} of all the systems tested for large files. It takes the lead
at \SI{2}{\gibi\byte}, where Mercurial drops out. To \gls{commit} the
\SI{2}{\gibi\byte} file, Git's average time is \SI{91.1}{\s}, Bup's is
\SI{89.1}{\s}, and DMV's is \SI{90.8}{\s}. All of these are a factor of around
ten times slower than the direct copy at \SI{9.1}{\s}. The differences get more
pronounced as the file sizes continue to increase. At \SI{64}{\gibi\byte}, Git's
average time is \SI{110}{\minute}, Bup's is \SI{72}{\minute}, DMV's is
\SI{298}{\minute}. The average \SI{64}{\gibi\byte} copy takes \SI{35}{\minute}.

\perottoinline{Multiplot graphs: Label a,b,c,d,e and describe in prose}
\perottoinline{Explain graph legends}

\begin{figure}[]
    \caption{Time to commit one large file to a fresh repository}
    \label{fig:plot-file-size--c1-time}
    \centering
    The top left graph shows the whole data set on a logarithmic scale. The
    others show certain interesting ranges on a linear scale.
    \includegraphics[]{plot-file-size--c1-time}
\end{figure}

%


\subsection{Time for File-Size Update Commit}

\autoref{fig:plot-file-size--c2-time} shows the time required for the second
\gls{commit}, after updating \num{1/1024}th of the file. Ideally this operation
should be faster than the first \gls{commit}, because the system should only be
storing the changed portion of the file. Indeed this is the case for Mercurial,
Bup, and DMV, which do store only the changed portion. Git, however, copies the
entire updated file into its repository as a new object, and so its \gls{commit}
time is virtually identical. The same is true of the copy control, though for
sizes smaller than \SI{8}{\gib} it is still faster than all the other systems.

\perottoinline{Describe results more thoroughly in words}

\begin{figure}[]
    \caption{Time to commit changes to one large file}
    \label{fig:plot-file-size--c2-time}
    \centering
    Sizes given are the total size of the file. The updated portion was
    \num{1/1024}th of the total file size.
    \includegraphics[]{plot-file-size--c2-time}
\end{figure}

%


\subsection{CPU Usage During File-Size Initial Commit}

\autoref{fig:plot-file-size--c1-cpu} shows the
CPU usage during the initial \gls{commit}.

\towrite{Prose description of results}
\perottoinline{Explain peaks in prototype CPU graph}
\perottoinline{Discuss somewhere "what if" iowait could be reduced
significantly. What would happen with memory, disk, remote}

\begin{figure}[]
  \caption{CPU utilization while committing one large file to a fresh repository}
  \label{fig:plot-file-size--c1-cpu}
  \centering
    \includegraphics[]{plot-file-size--c1-cpu}
\end{figure}


\begin{figure}[]
  \caption{CPU utilization while committing changes to one large file}
  \label{fig:plot-file-size--c2-cpu}
  \centering
    \includegraphics[]{plot-file-size--c2-cpu}
\end{figure}

%


\subsection{Repository Size after File-Size Update Commit}

\autoref{fig:plot-file-size--repo-size} shows the total repository size after
the update \gls{commit}, including the original file. This is after committing,
updating \num{1/1024}th of the file, and committing again.

\begin{figure}[]
  \caption{Total repository size after committing, editing, and committing again}
  \label{fig:plot-file-size--repo-size}
  \centering
    \includegraphics[]{plot-file-size--repo-size}
\end{figure}

The stored data overtakes the initial repository overhead after a file size of
around \SI{1}{\mib}, and the repository size for all systems converges to about
twice the size of the file. This is to be expected, since each measurement
includes the original file, the first copy of the file, and the updated
\num{1/1024}th. The exception is Git, which stores the entire updated file
during the update \gls{commit}, leading to a total disk space usage of three
times the file size. However, Git has a separate garbage collection stage where
it cleans up the repository and aggregates similar objects together in
\newterm{pack files}. The post-garbage collection size for Git is shown as a
separate line on the graph. This post-GC size converges to double the original
file size, but then jumps to three times at a file size of \SI{1.5}{\gib}. This
suggests that the pack step is failing silently at \SI{1.5}{\gib} and larger.
This is probably related to the way Mercurial's \glspl{commit} begin failing at
\SI{2}{\gib} and larger. Both operations are trying to load multiple versions of
the file into memory to calculate deltas for packing.

%



\section{Results: Number of Files}


\subsection{Time for Number-of-Files Initial Commit}

\autoref{fig:plot-num-files--c1-time} shows the time required for the initial
\gls{commit}, copying all files into a fresh empty repository.

\towrite{Prose description of results}

\begin{figure}[p]
    \caption{Time to commit many 1KiB files to a fresh repository}
    \label{fig:plot-num-files--c1-time}
    \centering
    The top left graph shows the whole data set on a logarithmic scale. The
    others show certain interesting ranges on a linear scale.
    \includegraphics[]{plot-num-files--c1-time}
\end{figure}

%


\subsection{Time for Number-of-Files Update Commit}

\towrite{Prose description of results}

\begin{figure}[p]
    \caption{Time to commit many 1KiB files after one of every \num{16} files
    has been updated}
    \label{fig:plot-num-files--c2-time}
    \centering
    The top left graph shows the whole data set on a logarithmic scale. The
    others show certain interesting ranges on a linear scale.
    \includegraphics[]{plot-num-files--c2-time}
\end{figure}

%


\subsection{CPU Usage During Number-of-Files Initial Commit}

\autoref{fig:plot-num-files--c1-cpu} shows CPU utilization during the
\gls{commit}.

\towrite{Prose description of results}

\begin{figure}[p]
  \caption{CPU utilization while committing many 1KiB files to a fresh
  repository}
  \label{fig:plot-num-files--c1-cpu}
  \centering
    \includegraphics[]{plot-num-files--c1-cpu}
\end{figure}

%


\subsection{CPU Usage During Number-of-Files Update Commit}

\towrite{Prose description of results}

\begin{figure}[p]
    \caption{CPU utilization while committing many 1KiB files after one of every
        \num{16} files has been updated}
  \label{fig:plot-num-files--c2-cpu}
  \centering
    \includegraphics[]{plot-num-files--c2-cpu}
\end{figure}

%


\subsection{Time for Number-of-Files Update Status Check}

\autoref{fig:plot-num-files--stat1-time} shows the time
required to check the changed status of all files just after committing.

\towrite{Prose description of results}

\begin{figure}[p]
  \caption{Real time required to check the status of many 1KiB files after
  initial commit}
  \label{fig:plot-num-files--stat1-time}
  \centering
    \includegraphics[]{plot-num-files--stat1-time}
\end{figure}

%


\subsection{Repository Size after Number-of-Files Update Commit}

\autoref{fig:plot-num-files--repo-size} shows the total
repository size, including the original files, after committing once, editing
1/1024th of every sixteenth file, and committing again.

\towrite{Prose description of results}

\begin{figure}[p]
  \caption{Total repository size after committing, editing, and committing again}
  \label{fig:plot-num-files--repo-size}
  \centering
    \includegraphics[]{plot-num-files--repo-size}
\end{figure}


\todo[inline]{Update these points and move them to appropriate prose
descriptions}

Unlike the experiment with a single large file, the numerous small files did not
quickly hit error-causing disk space or RAM limitations. The version control
systems happily crunched\perotto{to informal} the data as trial times grew into hours.

\begin{itemize}

    \item Again, after some initial overhead, commit and times increase
        linearly. However, Git's initial commit times actually decrease at
        certain points (128Ki, 1.5Mi, and 2Mi files). We are not sure how to
        explain this.

    \item Git in general is faster then Mercurial up to about half a million
        files. At 512Ki files is Mercurial and Git are about neck and neck. At
        768Ki and over, Mercurial is faster.

    \item Status check times are more erratic, though still increasing linearly
        overall. The variations may have to do with the output of the status
        commands and whether our terminal multiplexer was focused on the
        execution at the time. The status commands print one status line per
        file changed, which can be significant when hundreds of thousands of
        files involved. This output is significantly slower when the terminal
        multiplexer we used to monitor the experiments is connected, because it
        sends every line over the network to the monitoring machine.

    \item Mercurial update status is consistently slower than initial status,
        often by about 2-3x.

    \item Both Git and Mercurial converge to a little over 8x the space
        required. This probably has more to do with the filesystem block size
        than anything else. The underlying file system uses a 4KiB block size,
        so each 1KiB file will still use 4KiB of disk space. And since there are
        two copies of each file, that's 8KiB total for each 1KiB file, 8x the
        disk space.

    \item Mercurial converges towards the 8x limit faster though. We speculate
        this is because of lower repo overhead, and also because Git is creating
        \gls{tree} objects for each of the subdirectories in the file set. These
        files will be small, but each will take up another 4KiB block on the
        disk.

    \item Mercurial commits began to abort with disk space errors at 8Mi files,
        8GiB of data. This was surprising. Even at 8x disk usage, that should
        only be 64GiB of disk usage, well below the 192GiB free on the test
        disk.

\end{itemize}

%
