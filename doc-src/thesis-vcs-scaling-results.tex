\section{Results: File Size}

\subsection{File Size Limits}
\label{file-size-limits-results}

Both Git and Mercurial had limits to the size of file they could store
successfully. With a \SI{2}{\gibi\byte} file, Mercurial would issue a warning,
saying "up to 6442 MB of RAM may be required to manage this file," but the
commit would be stored successfully. With a file \SI{4}{\gibi\byte} or larger,
Mercurial's commit operation would exit with an error message and the commit
would not be stored. However, the repository would be left in a consistent empty
state. The atomicity of the commit operation held.

Git's behavior was more erratic. Starting with a file \SI{12}{\gibi\byte} and
larger, Git's commit operation would exit with an error code, reporting a fatal
out-of-memory error that \lstinline{malloc} failed to allocate
\SI{12}{\gibi\byte}. However, the commit would be successfully stored with no
consistency errors in the repository reported by \lstinline{git fsck}. Starting
at \SI{24}{\gibi\byte}, the commit operation would report the same error, the
commit would still be written, but then the \lstinline{git fsck} integrity check
would also exit with an error code. However, the error it reported in its output
was a similar "fatal" error as Git's \lstinline{malloc} error as the commit
operation, and it did not report any actual integrity errors in the repository.

So to test the commit, we extracted the \SI{24}{\gibi\byte} file from the
repository and compared it. It was the same as the original. So the commit was
intact. We also deliberately corrupted the Git pack file that stored the
\SI{24}{\gibi\byte} by overwriting one \SI{1}{\mebi\byte} block at an offset of
\SI{22}{\gibi\byte} with new pseudo-random data. When we ran the fsck command
again with the corrupted repository, it reported the integrity error, but it did
not report the \lstinline{malloc} error that it did before. So for the other
trials with larger files, we assumed that Git's errors were a false alarm and
allowed the trail to continue.

\towrite{After corruption, answer instantaneous rather than 7m before. Some
sorcery detected tampering, even though we fixed the mod time?}

Our DMV prototype was able to store a file up to \SI{64}{\gibi\byte} in size,
but time became a limiting factor as file size increased. At
\SI{96}{\gibi\byte}, our experiment script timed out and terminated the commit
after five and a half hours.

Our test environment itself limited the largest file stored by any VCS to
\SI{96}{\gibi\byte}. Any larger and it was simply impossible to store a second
copy of the file on our \SI{197}{\gibi\byte} test partition. Bup was able to
store a \SI{96}{\gibi\byte} file with no errors in just under two hours. Git
could also store such a large file, but one must ignore the false-alarm "fatal"
errors being reported by the user interface.

These findings are summarized in \autoref{file-sizes-table} and
\autoref{vcs-size-limits-table}.

\begin{table}
    \caption{Effects as file size increases}
    \label{file-sizes-table}
    \centering
    \begin{tabular}{r l}
        \SI{1.5}{\gibi\byte} & Largest successful commit with Mercurial \\
        \SI{2}{\gibi\byte} & Mercurial commit rejected \\
        \SI{8}{\gibi\byte} & Largest successful commit with Git \\
        \SI{12}{\gibi\byte} & Git false-alarm errors begin, but commit still intact \\
        \SI{16}{\gibi\byte} & Largest successful Git fsck command \\
        \SI{24}{\gibi\byte} & Git false-alarm errors begin during fsck, but commit still intact \\
        \SI{64}{\gibi\byte} & Largest successful DMV commit \\
        \SI{96}{\gibi\byte} & DMV timeout after \SI{5.5}{\hour} \\
                            & Last successful commit with Bup (and Git, ignoring false-alarm errors) \\
        \SI{128}{\gibi\byte} & All fail due to size of test partition \\
    \end{tabular}
\end{table}

\begin{table}
    \caption{Effective size limits for VCSs tested}
    \label{vcs-size-limits-table}
    \centering
    \begin{tabular}{l l}

        Git & Commit intact at all sizes, UI reports errors at \SI{12}{\gibi\byte} and larger \\

        Mercurial & Commit rejected at \SI{2}{\gibi\byte} and larger \\

        Bup & Successful commits at all sizes tested, up to \SI{96}{\gibi\byte} \\

        DMV & Successful commits up to \SI{64}{\gibi\byte}, timeout at
        \SI{5.5}{\hour} during \SI{96}{\gibi\byte} trial

    \end{tabular}
\end{table}


\subsection{Time for File-Size Initial Commit}

\autoref{fig:plot-file-size--c1-time} shows the time required for the initial
commit, adding a single file of the given size to a fresh repository. Over all,
the trend is clear and unsurprising: commit time increases with file size.

At file sizes below around \SI{2}{\mebi\byte}, commit times are dominated by
overhead-- around \SI{5}{\ms} for Git, \SI{100}{\ms} for Mercurial,
\SI{180}{\ms} for DMV, and \SI{900}{\ms} for Bup, vs only \SI{2}{\ms} for the
copy.

Bup, after starting with the highest overhead, goes on to have the fastest
inital commit of all the systems tested for large files. It takes the lead at
\SI{2}{\gibi\byte}, where Mercurial drops out. To commit the \SI{2}{\gibi\byte}
file, Git's average time is \SI{91.1}{\s}, Bup's is \SI{89.1}{\s}, and DMV's is
\SI{90.8}{\s}. All of these are a factor of around ten times slower than the
direct copy at \SI{9.1}{\s}. The differences get more pronounced as the file
sizes continue to increase. At \SI{64}{\gibi\byte}, Git's average time is
\SI{110}{\minute}, Bup's is \SI{72}{\minute}, DMV's is \SI{298}{\minute}. The
average \SI{64}{\gibi\byte} copy takes \SI{35}{\minute}.


\begin{figure}[]
    \caption{Time to commit one large file to a fresh repository}
    \label{fig:plot-file-size--c1-time}
    \centering
    The top left graph shows the whole data set on a logarithmic scale. The
    others show certain interesting ranges on a linear scale.
    \includegraphics[]{plot-file-size--c1-time}
\end{figure}


\subsection{Time for File-Size Update Commit}

\autoref{fig:plot-file-size--c2-time} shows the time required for the second
commit, after updating \num{1/1024}th of the file. Ideally this operation should
be faster than the first commit, because the system should only be storing the
changed portion of the file.


\begin{figure}[]
    \caption{Time to commit changes to one large file}
    \label{fig:plot-file-size--c2-time}
    \centering
    Sizes given are the total size of the file. The updated portion was
    \num{1/1024}th of the total file size.
    \includegraphics[]{plot-file-size--c2-time}
\end{figure}


\subsection{CPU Usage During File-Size Initial Commit}

\autoref{fig:plot-file-size--c1-cpu} shows the
CPU usage during the initial commit.

\begin{figure}[]
  \caption{CPU utilization while committing one large file to a fresh repository}
  \label{fig:plot-file-size--c1-cpu}
  \centering
    \includegraphics[]{plot-file-size--c1-cpu}
\end{figure}


\begin{figure}[]
  \caption{CPU utilization while committing changes to one large file}
  \label{fig:plot-file-size--c2-cpu}
  \centering
    \includegraphics[]{plot-file-size--c2-cpu}
\end{figure}


\subsection{Repository Size after File-Size Update Commit}

\autoref{fig:plot-file-size--repo-size} shows the total
repository size, including the original file, after committing once, editing
1/1024th of the file, and committing again.


\begin{figure}[]
  \caption{Total repository size after committing, editing, and committing again}
  \label{fig:plot-file-size--repo-size}
  \centering
    \includegraphics[]{plot-file-size--repo-size}
\end{figure}

\iffalse

\subsubsection{Observations}

Performance observations:

\begin{itemize}

    \item After some initial overhead, performance increases linearly with size.
        This is to be expected, since the operations are IO bound, copying all
        data to the repository.

    \item Times for the Mercurial update are faster than for Git's update,
        because Mercurial's archive format only saves deltas to the file.

    \item Mercurial has more initial time overhead, this is probably due to the
        fact that it is written in Python and requires starting the Python
        interpreter each time. This overhead is only 50 or 100ms, and quickly
        becomes insignificant compared to the IO operation time.

    \item Mercurial has less initial space overhead than Git.

        \begin{itemize}
            \setlength{\itemsep}{0pt}
            \setlength{\parskip}{0pt}
            \setlength{\parsep}{0pt}
            \item Minimum Mercurial repository size: 80KiB.
            \item Minimum Git repository size: 16KiB.
        \end{itemize}

        This can be seen in the disk space graph in the way the Mercurial usage
        converges towards 2x faster than Git usage does. But again, this quickly
        becomes insignificant compared to the size of the file.

\end{itemize}

\fi

\section{Results: Number of Files}

\subsubsection{Results}

\begin{figure}[p]
    \caption{Time to commit many 1KiB files to a fresh repository}
    \label{fig:plot-num-files--c1-time}
    \centering
    The top left graph shows the whole data set on a logarithmic scale. The
    others show certain interesting ranges on a linear scale.
    \includegraphics[]{plot-num-files--c1-time}
\end{figure}

\begin{figure}[p]
    \caption{Time to commit many 1KiB files after one of every \num{16} files
    has been updated}
    \label{fig:plot-num-files--c2-time}
    \centering
    The top left graph shows the whole data set on a logarithmic scale. The
    others show certain interesting ranges on a linear scale.
    \includegraphics[]{plot-num-files--c2-time}
\end{figure}

\begin{figure}[p]
  \caption{CPU utilization while committing many 1KiB files to a fresh
  repository}
  \label{fig:plot-num-files--c1-cpu}
  \centering
    \includegraphics[]{plot-num-files--c1-cpu}
\end{figure}

\begin{figure}[p]
    \caption{CPU utilization while committing many 1KiB files after one of every
        \num{16} files has been updated}
  \label{fig:plot-num-files--c2-cpu}
  \centering
    \includegraphics[]{plot-num-files--c2-cpu}
\end{figure}

\begin{figure}[p]
  \caption{Real time required to check the status of many 1KiB files after
  initial commit}
  \label{fig:plot-num-files--stat1-time}
  \centering
    \includegraphics[]{plot-num-files--stat1-time}
\end{figure}

\begin{figure}[p]
  \caption{Total repository size after committing, editing, and committing again}
  \label{fig:plot-num-files--repo-size}
  \centering
    \includegraphics[]{plot-num-files--repo-size}
\end{figure}

\autoref{fig:plot-num-files--c1-time} shows the time
required for the initial commit, copying all files into a fresh empty
repository.

\autoref{fig:plot-num-files--c1-cpu} shows CPU utilization
during the commit.

\autoref{fig:plot-num-files--stat1-time} shows the time
required to check the changed status of all files just after committing.

\autoref{fig:plot-num-files--repo-size} shows the total
repository size, including the original files, after committing once, editing
1/1024th of every sixteenth file, and committing again.


\iffalse

We performed a test where increasingly large sets of files were committed to the
different version control repositories. The procedure was as follows:

\begin{enumerate}
    \setlength{\itemsep}{0pt}
    \setlength{\parskip}{0pt}
    \setlength{\parsep}{0pt}
    \item Initialize an empty repository
    \item Generate a test file set of the given size. Each file is 1KiB of
        random binary data
    \item Commit the file set
    \item Check the status of the files
    \item Overwrite a small part of some of the files (1/1024th of the data in
        1/16 files)
    \item Check the status of the files again
    \item Commit the file set again
\end{enumerate}

Unlike the test with a single large file, the numerous small files did not
quickly hit error-causing disk space or RAM limitations. The version control
systems happily crunched the data as test times grew into hours.

Observations:

\begin{itemize}

    \item Again, after some initial overhead, commit and times increase
        linearly. However, Git's initial commit times actually decrease at
        certain points (128Ki, 1.5Mi, and 2Mi files). We are not sure how to
        explain this.

    \item Git in general is faster then Mercurial up to about half a million
        files. At 512Ki files is Mercurial and Git are about neck and neck. At
        768Ki and over, Mercurial is faster.

    \item Status check times are more erratic, though still increasing linearly
        overall. The variations may have to do with the output of the status
        commands and whether our terminal multiplexer was focused on the
        execution at the time. The status commands print one status line per
        file changed, which can be significant when hundreds of thousands of
        files involved. This output is significantly slower when the terminal
        multiplexer we used to monitor the experiments is connected, because it
        sends every line over the network to the monitoring machine.

    \item Mercurial update status is consistently slower than initial status,
        often by about 2-3x.

    \item Both Git and Mercurial converge to a little over 8x the space
        required. This probably has more to do with the filesystem block size
        than anything else. The underlying file system uses a 4KiB block size,
        so each 1KiB file will still use 4KiB of disk space. And since there are
        two copies of each file, that's 8KiB total for each 1KiB file, 8x the
        disk space.

    \item Mercurial converges towards the 8x limit faster though. We guess this
        is because of lower repo overhead, and also because Git is creating tree
        objects for each of the subdirectories in the file set. These files will
        be small, but each will take up another 4KiB block on the disk.

    \item Mercurial commits began to abort with disk space errors at 8Mi files,
        8GiB of data. This was surprising. Even at 8x disk usage, that should
        only be 64GiB of disk usage, well below the 192GiB free on the test
        disk.

\end{itemize}

\fi
