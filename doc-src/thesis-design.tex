\chapter{Design}

The data set will be a traditional hierarchical file structure, and it will be
presented as a normal filesystem. We do not want to reinvent the filesystem, and
we do not want to have to modify local applications to work with the data,
though applications can be modified to be aware of and to work with the system.

The storage will be modeled on version control systems, specifically Git and
it's directed acyclic graph data structure (DAG). There will be a
content-addressable blob store to de-duplicate and store data in a DAG
structure, and a working tree where the files can be read and written by normal
applications.

Like in a version control system, previous versions of the data will be kept,
along with metadata about the history of changes. Snapshots of the data will be
explicitly \newterm{committed} to the system by the user/application. The chain
of commit history can diverge into branches to be merged later, and each
individual store is also naturally a branch. Syncs and merges with connected
stores will be explicitly initiated.

Individual stores are not required to store all blobs in the DAG. Some blobs may
not be available on a given store, and history may be deliberately pruned to
save space. Algorithms and client applications must work with the blobs they
have locally and the hints about the availability of remote blobs.
