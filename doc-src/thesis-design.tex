\chapter{Design}

TODO: Low-level

TODO: Wide variety of platforms

TODO: Neutral
    - Normal hierarchy FS
    - No prescribed file formats
    - Normal files locally
    - Files editable with normal applications

TODO: Prototype copies working directory and object store. Disadvantage: disk
space. Could use FUSE

TODO: Breaking files into chunks, modified DAG

\begin{figure}[h!]
    \caption{Modified Git DAG}
    \label{fig:dia_new_dag}
    \centering
        \includegraphics[width=0.5\textwidth]{dia_new_dag}
\end{figure}



\subsection{What the system should not do}

We want to focus on the problem of storing file history and synchronizing files
between replicas.
We should be careful not to expand across the wrong abstraction boundaries or to
try to do too much.
In particular:

\begin{itemize}

  \item We do not want to reinvent the filesystem. The system should place and
    update files on the filesystem (or offer a filesystem view, such as with
    FUSE) for applications to use normally. Applications such as editors should
    not have to be rewritten to use our system.

  \item We do not want to create new exotic file formats. We believe that the
    classic tree of files is our best chance for long-term storage.

  \item We hope this system could eventually be used as a piece of
    infrastructure on which to build useful applications. It should not
    incorporate functionality that would better be left to an application.

  \item We do not want to deal with media metadata and categorization. Metadata
    and categorization is best left to the applications that produce and consume
    those media formats. We will merely provide the storage.

  \item However, knowledge of media formats might be used for behind-the-scenes
    optimization such as more efficient compression. E.g. recognizing that only
    tag data has changed in an audio file.

\end{itemize}
