\chapter{DMV Design}


\section{DMV's DAG}

\gls{DMV}'s data set and its history are represented as a \gls{DAG}, and the
rest of the design flows from that. \gls{DMV}'s \gls{DAG} is based on Git's, but
it adds a new object type, the \gls{chunkedblob}, which represents a \gls{blob}
that has been broken into several smaller chunks. An example \gls{DMV} \gls{DAG}
is shown in \autoref{dia_dmv_dag_example_three_commits}.

\begin{figure}[]
    \centering
    \includegraphics[width=0.9\textwidth]{dia_dmv_dag_example_three_commits}
    \caption{A simple DMV DAG with three commits}
    \label{dia_dmv_dag_example_three_commits}
\end{figure}

The object types that make up the \gls{DMV} \gls{DAG} are as follows:

\begin{description}

    \item[Blob] As with Git, a \gls{blob} simply holds binary data.

    \item[Chunked blob] Unlike with Git, larger \glspl{blob} in \gls{DMV} are
        broken into chunks. Each \gls{chunkedblob} is simply an index of
        \glspl{blob} (or other \glspl{chunkedblob}) that make up the larger
        \gls{blob}. A file that is stored in the system may be represented by a
        \gls{blob} or a \gls{chunkedblob}.

    \item[Tree] As with Git, a directory that is stored in the system is
        represented by a \gls{tree} object. The \gls{tree} refers to the
        \gls{blob}, \gls{chunkedblob}, or \gls{tree} that represents each file
        or subdirectory, along with metadata such as the filename.

    \item[Commit] As with Git, a \gls{commit} represents a given state of the
        data set. It refers to the \gls{tree} that represents the top-level
        directory of the data set at that state, along with metadata such as
        author, date, and description. It also refers to the previous
        \gls{commit} (or \glspl{commit}) that represented the previous state of
        the data set.

    \item[Ref] As with Git, a \gls{ref} is a reference to a particular
        \gls{commit}. A \gls{ref} might represent the current state of a branch of
        development, or a particular state to save for later (a tag).

\end{description}

\perrolfinline{These description lists are great. The look good. They're
concise. And they let people who already know what you're talking about skip the
explanations. Perhaps employ this in other places as well. Maybe the description
of Git objects? Maybe for node/object store/working directory?}

The \gls{DAG} begins with \glspl{ref}. Each repository has a list of \glspl{ref}
that lead into the \gls{DAG}. Each \gls{ref} refers to a \gls{commit}. Each
\gls{commit} refers to a \gls{tree} that represents the state at the time of
committing, plus one or more parent \glspl{commit}. A \gls{tree} refers to the
\gls{blob}, \gls{chunkedblob}, or \gls{tree} that makes up each entry in the
directory. A \gls{chunkedblob} can refer to other \glspl{chunkedblob} or to
\glspl{blob}. And finally, the \glspl{blob} are the leaves of the graph. This
relationship is illustrated in \autoref{fig:dia_new_dag}.

\begin{figure}[]
    \centering
        \includegraphics[width=0.5\textwidth]{dia_new_dag}
    \caption{DMV DAG Object Types}
    \label{fig:dia_new_dag}
\end{figure}


Files are split into chunks using a \gls{rollinghash} such as Rabin-Karp
fingerprinting\cite{rabin_karp_fingerprinting}. This splits the files into
chunks by content rather than position, so that identical chunks within files
(and especially different versions of the same file) will be found and stored as
identical objects, regardless of their position within the file. This way,
identical chunks will be naturally de-duplicated by the \gls{DAG}, and only the
changed portions of files need to be stored as new objects.

%


\section{Working with an Incomplete DAG}

The \gls{DAG} stores the full history of a data set, and it can be sliced to
partition the data in space and time.

\begin{description}

    \item[Partial history of full data set] Keep a subset of \glspl{commit},
        plus all \glspl{tree}, \glspl{blob}, and \glspl{chunkedblob} reachable
        from them (\autoref{dia_dmv_dag_slice_partial_history}).

    \item[Full history of part of data set] Keep all \glspl{commit}, but keep
        only the \glspl{tree}, \glspl{blob}, and \glspl{chunkedblob} for certain
        paths (\autoref{dia_dmv_dag_slice_history_of_subset}).

    \item[Full history of metadata] Keep all \glspl{commit} and \glspl{tree},
        but omit \glspl{blob} and \glspl{chunkedblob}
        (\autoref{dia_dmv_dag_slice_history_of_metadata}).

\end{description}

\newcommand{\slicediagramwidth}{0.45\textwidth}

\begin{figure}

    \centering

    \begin{subfigure}[]{\slicediagramwidth}
        \includegraphics[width=\textwidth]{dia_dmv_dag_slice_partial_history}
        \caption{Partial history of full data set}
        \label{dia_dmv_dag_slice_partial_history}
    \end{subfigure}
    ~
    \begin{subfigure}[]{\slicediagramwidth}
        \includegraphics[width=\textwidth]{dia_dmv_dag_slice_history_of_subset}
        \caption{Full history of part of data set}
        \label{dia_dmv_dag_slice_history_of_subset}
    \end{subfigure}
    ~
    \begin{subfigure}[]{\slicediagramwidth}
        \includegraphics[width=\textwidth]{dia_dmv_dag_slice_history_of_metadata}
        \caption{Full history of metadata}
        \label{dia_dmv_dag_slice_history_of_metadata}
    \end{subfigure}

    \caption{A DMV DAG, sliced in different dimensions}
\end{figure}

A \gls{DMV} node can use any combination of these slicing techniques to keep
only those objects needed at that location. New \glspl{commit} can still be made
to a partial \gls{DAG}. A node that stores only part of the data set can create
new \glspl{commit} that represent changes to that portion of the data set by
assuming that everything else remains unchanged. Even a metadata-only node can
create new \glspl{commit} that represent renames and reorganization of the same
files.

So the \gls{DAG} can be spread across many nodes, sliced according to what is
needed at each location. Nodes can keep a record of what objects exist at their
neighbor nodes, and statistics about latency to those nodes, which allow them to
estimate how long it would take to transfer a file that is not currently present
on the system.

Though we did not have time to implement this feature, we are envisioning an
enhanced \lstinline{ls} command output that shows these estimates, as
illustrated in \autoref{example-ls-output}.

\begin{figure}[]
    \caption{Speculative DMV ls output showing remote files}
    \label{example-ls-output}
    \lstinputlisting[nolol]{lst-example-ls-output.txt}
\end{figure}


%
