\chapter{Discussion}



\section{Possible Applications}

\begin{itemize}

  \item Individual users might use it to maintain a collection of important
    documents, photos, and media, making it easier to keep up-to-date backups
    and to synchronize between computers, mobile devices, and removable drives.

  \item Professional users that work with files too large for traditional
    version control, such as graphic designers, audio engineers, or maybe even
    video editors, might finally be able to adopt a version-control workflow.

  \item Corporate or government users might use it to maintain large archives of
    data with full history.

  \item Far-flung networks with high-latency or rare connectivity, such as
    remote wildlife sensors or Mars rovers, could use it to manage and
    synchronize data.

\end{itemize}



\section{As an Abstraction of Data Space and Time}

We are thinking about data across a number of dimensions:

\begin{description}

  \item[Coverage of data set] How much of the data set is available locally or
    in neighboring nodes?

  \item[Coverage of data history] How much of the data set's history is
    available locally or in neighboring nodes?

  \item[Divergence of versions] How many different branches has this data been
    forked into, and how different are they?

  \item[Number of replicas] How many times is the data replicated across
    neighboring nodes? Is any data in danger of being permanently lost?

  \item[Availability of or distance to replicas] Of the replicas available, how
    available are they? What is the bandwidth of the connection to the
    neighboring nodes? What is the latency?

\end{description}


\section{Limitations}

TODO: non-programmers (and even some programmers) cannot handle Git
    - Key to usability would be to make as linear a history as possible with
    auto-updates, but that is the job of a separate app
