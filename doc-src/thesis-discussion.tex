\chapter{Discussion}


\section{Data Granularity and Storage Schemes}
\label{chunk-then-recombine}

All four of the systems we examined in detail-- Git, Mercurial, Bup, and
\gls{DMV}-- model data and its history with a similar directed acyclic graph.
The major difference is the granularity at which they work with data, and how
they store it.

Both Git and Mercurial take the file as the basic unit of data granularity,
though they approach storage differently. Git stores files whole as \glspl{blob}
during \gls{commit}, storing them and other objects as files in object
directories (as tested in \autoref{dir-experiment}). Later, an optional packing
phase will compact objects together into \glspl{packfile}, where similar
objects are stored as deltas against a base revision \cite[Section
10.4]{git_book}. Mercurial stores each file's different revisions as deltas
   against a base revision in a \newterm{filelog} structure \cite[Chapter
   4]{hgbook}. This is Mercurial's primary storage format, and it is constructed
   during \gls{commit}.

By using files as the basic unit of storage, and storing files as deltas against
a base revision, both Git and Mercurial will at some point load an entire file
into memory in order to compare it to another version. This ties the maximum
file size that the system can work with to the amount of available RAM. In
Mercurial's case, the error message that appears when attempting to \gls{commit} a
\SI{2}{\gib} file warns that \SI{6}{\gib} will be required to manage it. And
because it has to calculate deltas in order to store a file at all, Mercurial
simply cannot work with any file that it can't fit into memory three times over.
This is why Mercurial could not store files larger than \SI{1.5}{\gib} in the
file-size experiments (\autoref{file-size-limits-results}).

Because Git's delta calculation happens in a secondary behind-the-scenes phase,
it can still manage to \gls{commit} files larger than available RAM, but it
prints errors as the other operations fail. The two-phase approach also requires
extra disk space and processing power. If a large file is changed, the at first
both revisions will be written in full, taking twice the disk space. Then a
separate operation will have to reread both \glspl{blob} in full to calculate
deltas and pack the objects.

Both \gls{DMV} and Bup avoid these pitfalls by operating with a finer
granularity, using a rolling hash to divide files into chunks by their content.
It is the chunks and their indexes that must fit into memory, not the entire
file. And then since chunks are only a few kilobytes and chunk indexes are
hierarchical, file size becomes theoretically unlimited. Dividing into chunks by
rolling hash also makes delta compression unnecessary, because identical chunks
in different files or file revisions will naturally de-duplicate. At this point,
it is the method of object storage that becomes the bottleneck.

The current \gls{DMV} prototype stores objects loose as files on the filesystem.
This proves to be wasteful of disk space, taking up whole filesystem blocks with
tiny objects and with directories. It also causes dramatic slowdown as the disk
gets full and the filesystem has to work harder to find free blocks (as seen in
\autoref{seek-times-results}). Dividing into chunks solves the problem of
storing large files by turning it into the problem of storing many files.

Bup's storage strategy is the best of both worlds. It first divides files into
chunks, but then re-packs objects together into \glspl{packfile}. In fact, Bup
uses Git's \gls{packfile} format\footnotemark, but it writes it directly without
the separate compacting phase, and without bothering to calculate deltas
\cite{bup_design}. This makes efficient use of disk space, and allows the
\glspl{packfile} to be written sequentially, minimizing disk seeks. This is why
Bup was clearly the fastest of the systems tested in both the file-size and
number-of-files experiments (\autoref{num-files-exp-desc}).

\footnotetext{Git has no notion of chunks, but Bup reuses Git's \gls{tree}
objects as chunk indices. Git can read a repository written by Bup, but it will
see a large file as a directory full of small chunk files.}

So we see that the key to handling large files is to break them into many
smaller files, and the key to storing many small files is to combine them into
larger files. The magic is in the combination, where files and revisions of
files are broken into chunks by content, so that identical chunks are naturally
de-duplicated in storage. That is what gives significant disk space savings over
simply zipping up snapshots of the data.

%


\section{Subtleties of the Rolling Hash}

The rolling hash is the key to providing smaller granularity, because it is what
identifies common byte strings within files.

\towrite{The key to dealing with many small chunks is to re-aggregate them. Then
    what are other implications of chunk size?
    - De-duplication granularity vs metadata overhead.
    - Need investigation on real-world data.
    - And what is up with high standard deviation?
    - Big window size probably not what we're looking for.
    - Other ways to correct deviation?
    - Deal with runs of zeros?}

\towrite{Rolling hash improvements:
    - more thought to window size. Larger than avg chunk size and you get
    subsequent chunks affecting each other.
    - thought to what happens at chunk boundaries. Run of zeros at chunk
    boundary will create a run of 1-byte chunks. How likely is that?
    - What about long runs of zeros anyway?
}

\section{DMV Prototype development}

\towrite{Mistakes along the way: storing whole status tree in memory}

\towrite{Graph walk abstraction}


%


\section{Limitations}

\towrite{Did not get to implementing networking features, but did pull off
subtree commit}

\towrite{non-programmers (and even some programmers) cannot handle Git
    - Key to usability would be to make as linear a history as possible with
        auto-updates, but that is the job of a separate app
    - Cite redesign of Git\cite{redesign_of_git}
}

%


\section{Potential Applications}

As a general distributed storage platform, \gls{DMV} could have a wide range of
potential applications:

\begin{description}

    \item[Private data storage] Individual users might use \gls{DMV} to maintain
        a collection of important documents, photos, and media on their own
        devices, making it easier to keep up-to-date backups and to synchronize
        between computers, mobile devices, and removable drives without giving
        their data to a third-party cloud service.

    \item[Large-file version control] Professional users that work with files
        too large for traditional version control, such as graphic designers,
        audio engineers, or video editors, might finally be able to adopt a
        version-control workflow.

    \item[Long-term data archiving] Corporate or government users might use it
        to maintain large archives of data with full revision history.

    \item[Low-connectivity networks] Far-flung networks with high-latency or
        rare connectivity, such as remote wildlife sensors or Mars rovers, could
        use it to manage and synchronize data.

\end{description}

%


\section{Aggregating Data about a Sharded DAG}

Though not implemented in the \gls{DMV} prototype, we would like a \gls{DMV}
node to be able to aggregate data about what \gls{DAG} objects are available at
its neighbors and throughout the network. The data could be analyzed to give
metrics about the data set in several dimensions:

\begin{description}

  \item[Coverage of data set] How much of the data set is available locally or
    in neighboring nodes?

  \item[Coverage of data history] How much of the data set's history is
    available locally or in neighboring nodes?

  \item[Divergence of versions] How many different branches has this data been
    forked into, and how different are they?

  \item[Number of replicas] How many times is the data replicated across
    neighboring nodes? Is any data in danger of being permanently lost?

  \item[Availability of or distance to replicas] Of the replicas available, how
    available are they? What is the bandwidth of the connection to the
    neighboring nodes? What is the latency?

\end{description}

Such data could be useful for monitoring the health of the data set, alerting
the user to shards of data that risk being lost without further replication.

\todo[inline]{Move this section to design chapter? It's not in the actual
prototype, but it's part of the design we wanted to build. -- Otto: Ok to have
it here.}

%


\section{What the system should not do}

\todo{Update tense}
We want to focus on the problem of storing file history and synchronizing files
between replicas.
We should be careful not to expand across the wrong abstraction boundaries or to
try to do too much.
In particular:

\begin{itemize}

  \item We do not want to reinvent the filesystem. The system should place and
    update files on the filesystem (or offer a filesystem view, such as with
    FUSE) for applications to use normally. Applications such as editors should
    not have to be rewritten to use our system.

  \item We do not want to create new exotic file formats. We believe that the
    classic tree of files is our best chance for long-term storage.

  \item We hope this system could eventually be used as a piece of
    infrastructure on which to build useful applications. It should not
    incorporate functionality that would better be left to an application.

  \item We do not want to deal with media metadata and categorization. Metadata
    and categorization is best left to the applications that produce and consume
    those media formats. We will merely provide the storage.

  \item However, knowledge of media formats might be used for behind-the-scenes
    optimization such as more efficient compression. E.g. recognizing that only
    tag data has changed in an audio file.

\end{itemize}
