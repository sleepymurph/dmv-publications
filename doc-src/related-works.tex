\documentclass[a4paper]{article}
% vim: set ts=2 sts=2 sw=2 :

\usepackage{graphicx}
\usepackage[utf8]{inputenc}

% Use a blank space between paragraphs instead of an indent.
\usepackage[parfill]{parskip}

% Source code listings
\usepackage{listings}
\lstset{basicstyle=\ttfamily\footnotesize,breaklines=false}

\usepackage{url}
% To actually link URLs in the PDF.
%
% The hyperref docs recommend declaring it after the other \usepackage
% declarations, because it has to redefine several commands to work properly,
% and other later redefinitions might interfere.
% Remove the hidelinks parameter to get (ugly) visual highlights around the
% links.
\usepackage[hidelinks]{hyperref}

% Create a short macro for newly-defined key terms.
\newcommand{\newterm}{\textit}


\begin{document}

\title{Master's Thesis Related Works}
\author{Michael Murphy}
\maketitle

\section{Related Works}

\subsection{Distributed storage and synchronization systems}

\subsubsection{Camlistore}

Camlistore \cite{camlistore_homepage} is an open-source project to create a
private long-term data storage system for personal users. It allows storage of
diverse types of data and it synchronizes between multiple replicas of the data
store. However, it eschews normal filesystems and creates its own schemas to
store various media.


\subsubsection{Dat Data}

Dat \cite{dat_homepage} is an open-source project for publishing and sharing
scientific data sets for research. This project has a lot of overlap with ours,
and several of the core ideas are similar, including breaking files into smaller
chunks, and tracking changes via a Git-like DAG. However, their focus is
different. The Dat team is concentrating on publishing research data, and making
that specific task as simple as possible for non-technical researchers who might
not be familiar with version control. By contrast, our project operates at a
lower level of abstraction, offering the full power of version control in a very
general way, exposing and illuminating the complexities rather than trying to
hide them or automate them away.

Where Dat focuses on publishing on the open internet, we focus on ad-hoc
networks and data that may be private. Where Dat has components for automating
peer discovery and consensus, we work at a lower level, trying to perfect and
generalize the storage aspect first. Dat seems to assume that data sets will be
small enough to fit on a typical disk on a workstation, while we want to scale
even larger.

We hope that our system could be used as a base to build something like Dat, but
we intend to create something even more general than the Dat core.


\subsubsection{Eyo}

Eyo \cite{Strauss:2011:EDP:2002181.2002216} is system for storing personal media
and synchronizing it between devices. It utilizes a Git-like content-addressed
object database behind the scenes, but it works more like a networked filesystem
than version control. It focuses on organizing media by metadata, which requires
agreement on metadata formats, and it requires applications to be rewritten to
access files via Eyo rather than the filesystem, both of which are thorny and
ambitious problems. We prefer to focus purely on storage and synchronization.


\subsubsection{git-annex and git-media}

Git-annex \cite{git_annex_homepage} and git-media \cite{git_media_github} are
open-source projects that extend Git with special handling for larger files.
Both store the metadata of larger files in the normal Git repository and then
store the files themselves in a separate location. Git-media stores all the
larger files in a separate data store which may be remote. Git-annex is more
flexible. Annex files may be spread across several different remote repository
clones or data stores, and git-annex has features for tracking the locations of
annex files in different remote repositories and moving them from one repository
to another. These tracking and distribution features are very similar to our
goals. However, git-annex is not quite as flexible as we aim for in our system.
It considers the large files atomic units, and it does not break them into
smaller chunks for de-duplication. Also, because metadata is processed by Git,
it has the same limitations that Git does. All repositories must have all
metadata, and performance suffers when metadata is too large to fit into RAM.


\subsubsection{IPFS: The Interplanetary Filesystem}

IPFS \cite{ipfs_github_main} is an open-source project to create a global
content-addressed filesystem. By its global nature, all files are stored
together, publicly, in a global network of nodes with global addressing. IPFS
should be an excellent resource for storing published information, but we wanted
to work on a smaller, more private scale with discrete data sets. We want
individuals and organizations to be able manage their own data stores privately
on their own hardware.

It should be noted that IPFS does have support for storing private objects by
way of object-level encryption. However, this seems wasteful of disk space,
since small changes in the plain text of a file would completely change the
ciphertext, leaving no way to compress the redundancy.


\subsubsection{Kademlia}

Kademlia \cite{Maymounkov2002} is an advanced distributed hash table system that
updates its network topology information as part of normal lookups. It is an
advanced piece of infrastructure, but like other distributed hash tables, it
focuses on system-wide consistency, rather than the version-control paradigm we
are trying to achieve.


\subsection{Content-Addressed Storage and Backup}

\subsubsection{Boar}

Boar \cite{boar_homepage} is an open-source project to create a version control
system for large binary files. It is one of the main inspirations for our
project. It stores file versions in a content-addressed way, and provides
de-duplication for large files that only change in small pieces, and it can
truncate history to reclaim disk space. However, Boar retreats to a centralized
version control paradigm, with a central repository that working directories
must connect to to check files in or out. We want to provide the advantages of
Boar in a flexible distributed version control model. Boar also has practical
limitations on repository size and number of files. Repositories are assumed to
fit on one disk volume, and file metadata is assumed to fit into Ram. We aim to
overcome both of those limitations.


\subsubsection{Bup}

Bup \cite{bup_homepage} is an open-source file backup system that is based on
Git's repository format. A Bup backup is a valid Git repository and it can be
read by Git, but Bup is a separate program written from scratch to read and
write files to Git's pack file format directly, skipping Git's separate store
and pack steps that use double the disk space. It has many features that we want
for our low-level storage of the object database. It breaks files into chunks by
rolling checksum, and it has considerations for metadata that is larger than
RAM. However, it is locked into a backup-based workflow. History is linear and
based on clock time of backup. And it assumes that the whole data set and the
whole repository can fit onto one filesystem.



% \nocite{*}  % Print all references even if they're not used
\bibliographystyle{plain}
\bibliography{research}

\end{document}
