\chapter{Introduction}

It is still surprisingly difficult to keep backups and synchronize data. Many of
us have several computers, perhaps a laptop, a phone, and a work computer, and
we would like to synchronize data between them. We want to keep a Word document
synchronized between home and work. We want to put new music on our phones, and
pull photos off of camera SD cards. We have backups on removable drives, but we
don't remember when it was that we last did a backup, or what is new since then.
We have these sets of files that tend to fragment themselves across our devices,
and we lose track of what is where.

Cloud computing offers to centralize and safeguard our data, keeping it all in
one place and taking care of the backups for us. Google Drive gives us a shared
document that many people can edit in real time. Spotify offers endless music
streams. Instagram lets us save and share photos. DropBox gives us a folder that
syncs. But many of these solutions are specialized applications for specific
media, which can limit their general usefulness; most require constant network
connectivity, making them ill-suited for intermittent or high-latency
connections; and all require entrusting your data to a third-party service,
which raises concerns about privacy and storage longevity. Why can't we simply
track the files we have across the devices we have?

% TODO: Cite "folder that syncs"
% https://www.quora.com/Why-is-Dropbox-more-popular-than-other-programs-with-similar-functionality

Software programmers have an excellent system for backup and sync right at their
fingertips: \newterm{distributed version control systems (DVCSs)}, such as Git
and Mercurial. Version control systems track all changes to a collection of
files, allowing collaborators to work independently and then synchronize and
share their work. Additionally, in a distributed version control system, every
collaborator has a full copy of the project's history. That redundancy not only
allows collaborators to work offline, but it also functions as a backup. Linus
Torvalds, the creator of Linux and Git, once famously joked that he doesn't keep
backups, he simply publishes his work on the internet and lets others copy it
\cite{linus_no_backups}.

The major limitation distributed version control systems is that they are
designed to store program source code: plain text files in the range of tens of
kilobytes. They often have trouble with larger media files, becoming sluggish or
wasteful of disk space. Many a web-design team has come to regret checking their
graphical assets into version control. In addition, many will have trouble
scaling up as the number of files increases or the history grows increasingly
long.

What if we could generalize the distributed version control concept to store a
wide variety of file sizes, from kilobyte text files to multi-gigabyte videos?
In addition, what if we relaxed the assumption that every replica contain the
complete history, and allowed each replica to choose what subsets of the files
and the history to store, according to the replica's capacity and need? The
answer could be a new abstraction for tracking a data set and its history as a
cohesive whole, even when it is physically spread over many different nodes.


\section{CAP Theorem and the Importance of Availability}

Distributed systems are ruled by the \newterm{CAP-theorem} \cite{cap_origin},
which states that a system cannot be completely consistent (\newterm{C}),
available (\newterm{A}), and tolerant of network partitions (\newterm{P}) all at
the same time. One area must always suffer, and since network partitions are
always a possibility, a distributed system must make trade-offs between
availability and consistency.

TODO: Traditional ACID

TODO: Modern NoSQL/BASE

TODO: New thinking (CAP years later)
   - partitions rare
   - partitions are just latency
   - "the partition decision": cancel and decrease availability, proceed and
   risk inconsistency
   - general problem of resolving conflicts is not solvable
   \cite{cap_years_later}

TODO: DVCS is always available
    - Branches: No global concept of recent version


\section{Git and the DAG}

TODO: Explain Git and DAG
    - Append only
    - Easy to sync
    - Trick is merging

TODO: Diagram of Git DAG
