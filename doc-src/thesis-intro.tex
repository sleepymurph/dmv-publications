\chapter{Introduction}

% TODO: Cloud neat but requires connectivity, privacy concerns
% TODO: Individual users: photos, music, documents
% TODO: Corporate or government data to keep on own servers
% TODO: programmers have DVCS but it's for source code
% TODO: "It's a folder that syncs" -- https://www.quora.com/Why-is-Dropbox-more-popular-than-other-programs-with-similar-functionality
% TODO: Linus: no backups, just upload -- http://www.webcitation.org/6P8EBZqQX

It is still surprisingly difficult to keep backups and synchronize data. Many of
us have several computers, perhaps a laptop, a phone, and a work computer, and
we would like to synchronize data between them. We want to keep a Word document
synchronized between home and work. We want to put new music on our phones, and
pull photos off of camera SD cards. We have these data sets that tend to
fragment themselves across our devices, and we lose track of what is where.

Cloud computing offers to centralize and safeguard our data. Google Drive gives
us a shared document that many people can edit in real time. Spotify offers
endless music streams. Instagram lets us save and share photos. DropBox gives us
a folder that syncs. But many of these solutions are specialized for specific
media, most require constant network connectivity, and all require entrusting
your data to a third-party. Why can't we just track the files we have across the
devices we have?

Those of us who are programmers have an excellent system for backup and sync
right at our fingertips: \newterm{distributed version control systems (DVCSs)},
such as Git and Mercurial. Version control systems track all changes to a
collection of files, allowing collaborators to work independently and then
synchronize and share their work. In a distributed version control system, every
collaborator has a full copy of the project's history, and that redundancy
functions as a backup as well. Linus Torvalds, the creator of Linux and Git,
once famously said that he doesn't keep backups, he simply publishes his work
and lets others copy it \cite{linus_no_backups}.
