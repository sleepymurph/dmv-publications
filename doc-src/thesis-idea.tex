\chapter{Idea}

A low-level distributed data storage platform that makes data location and
availability explicit, rather than trying to hide or abstract them away.

The target use for this system is to spread a data set across many compute
nodes, and to allow the sharding and replication of the data set to be
customizable according to the location and availability of the nodes, and what
data is needed where. The ultimate goal is to track the scattered data as a
coherent data whole, and to allow the end user or client application to
visualize and manage what data is stored on what nodes.

The system aims to be a low-level service, providing information about the data
and tools to manage it, but ultimately leaving decisions to the end user or
client application.

We assume that the data set and all nodes in the system are controlled by the
user.

The target data sets for this system include collections of media files (binary
files ranging in size from hundreds of kilobytes to several gigabytes) numbering
in the hundreds to millions. The system will not attempt to understand the
internals of different media formats and will treat all files as opaque blobs.
Such media data suggests a usage pattern where files are more often created or
moved than they are updated in-place. Also, we assume writes will be infrequent
compared to traditional online transactional databases (seconds to minutes, or
longer, between updates).

The system will be designed for high availability in the face of connections
that are intermittent, high-latency, expensive, or otherwise restricted. At each
node, the portion of data that is stored on that node will be available for
reading and writing by local applications. Metadata about the non-local portions
of the data set will be available as well, along with hints about the location
of that data. Clients can use that information to schedule transfers and
updates. Consistency will be resolved during updates. Conflicting versions will
be presented to the client application to resolve.
