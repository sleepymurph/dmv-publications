\chapter{Idea: Distributed Media Versioning (DMV)}

\glsreset{DMV}

\gls{DMV} is our new distributed data storage platform. The core idea is
relatively simple --- store data in a Git-like \gls{DAG}, but make the following
changes:

\begin{tight_enumerate}

    \item{Store data at a finer granularity than the file}

    \item{Allow nodes to store only a portion of the \gls{DAG} as a whole}

\end{tight_enumerate}

Doing so allows a data set to be replicated or sharded across many nodes
according to the capacity of nodes and the needs of local users. The focus is on
data locality: tracking what data is where, presenting that information to the
user, and making it easy to transfer data to other nodes as desired. The
ultimate goal is to create a new abstraction, of \emph{many devices, one data
item} in varying states of synchronization.

%


\section{Characteristics}
\label{dmv-characteristics}


\paragraph{General storage}

\gls{DMV} is a generalized storage platform that places no restriction on file
types. Its data model is a classic hierarchical filesystem, but with history.
Applications on each node can read and write to the files via the filesystem as
normal. \gls{DMV} is dogmatic about the
\gls{endtoendargument}~\cite{endtoendargument}, realizing that it cannot
anticipate all the needs of end users and applications. So it aims to be as
general and neutral as possible, focusing on the core task of storage and
tracking, and providing a platform for other applications to build on.


\paragraph{Inspired by version control}

\gls{DMV} is inspired by \acrlongpl{DVCS}. Its core data structure is
a \gls{DAG}, based on Git's but modified to store a wider range of file sizes.
The key difference is that large files are broken into smaller chunks (around
\SI{52}{\kibi\byte}), which is what ensures that no single object is too large
for memory or disk. Breaking files into chunks also allows the data structure to
naturally de-duplicate parts of files that do not change. For example, if a
large media file has its metadata block updated, only the chunk containing the
updated metadata is new. The other chunks will simply be reused.


\paragraph{Scalability}

\gls{DMV} can store a wide variety of file sizes and file quantities. Current
\glspl{VCS} load entire files into memory to calculate deltas, which limits the
maximum size of files they can store to what can fit into RAM. Additionally,
they create separate files for each input file, leading to disk-space overhead
and write-speed slowdowns as the number of files on one filesystem increases.
\gls{DMV} successfully avoids the file-size limitation, but as currently
described and implemented, it does succumb to the number-of-files limitations.
However, we know of an effective method to avoid the number-of-files limitations
in the future. These limitations are explored experimentally in
\autoref{file-size-exp-desc}.

Additionally the stored data set can also be sharded, so that the data set as a
whole can scale to sizes too large to fit on any one computer in the network.

\gls{DMV} will be able to run on a wide range of hardware. The current prototype
runs on Linux PCs, but it is designed with an eye towards porting it to
low-powered and mobile devices such as mobile phones or sensor networks.


\paragraph{Configurable sharding}

The \gls{DAG} structure tracks the data set in three dimensions:

\begin{tight_enumerate}
    \item The set of files themselves
    \item The history of the files
    \item The parallel branches of development in the history
\end{tight_enumerate}

Traditional \glspl{DVCS} tend to assume that each repository has the full
history of all files, though not every branch of development. In contrast, each
\gls{DMV} node will store a subset of the data, sharded along any of those
dimensions, configurable by the user. A node could keep the full history of only
a small subset of files, or only the most recent few versions of the full set of
files, or only a few branches, or any combination.


\paragraph{Availability}

Like in a \gls{DVCS}, the \gls{DAG} structure records all history of the data
set and allows many different branches of development to exist in parallel. This
allows high availability. Any node can always make updates autonomously, without
coordinating with other nodes. Reconciliation of conflicting writes happens
later --- via explicit, user-driven merging. \gls{DMV} will only require a
connection to another node during explicit synchronize operations, and so it
should be well-suited for applications with intermittent or high-latency
connectivity.


\paragraph{Data integrity}

Because the \gls{DAG} is append-only, and \gls{DAG} objects are addressed by a
cryptographic hash of their contents, this allows \gls{DMV} to verify stored
objects and detect corrupted data. Corrupted objects can be replaced by
retrieving an intact copy from another replica.

\gls{DMV} should never lose data accidentally. However, because \gls{DMV}
tolerates an incomplete \gls{DAG}, objects can be deliberately deleted from all
nodes to save space or to redact sensitive information.


\section{Security Model}
\label{security-model}

The \gls{DAG}'s append-only nature and cryptographic content addressing provide
protection from tampering. As long as the complete \gls{DAG} is available, its
integrity can be verified. Allowing an incomplete \gls{DAG} does introduce an
opening for tampering, because not all objects are available to verify, but we
ignore such a possibility for now. Because \gls{DMV} is primarily meant to allow
individuals or organizations to manage their own data on their own hardware, we
assume that all nodes will be under the user's control, and that users will only
accept additions to their \gls{DAG} from trusted collaborators. This defers
questions of security to the systems and networks that are running the \gls{DMV}
nodes. Data can be kept private by keeping all \gls{DMV} nodes on a private
network. Unfortunately, though \gls{DMV}'s checksums can be checked to detect
tampering, \gls{DMV} itself has no way to detect unauthorized reads. We also do
not consider byzantine failures or guard against malicious nodes at this time.

%


\section{What's in a Name?}

We chose the name \emph{\acrlong{DMV}} because it is a concise way to describe
the system, emphasizing its distributed nature, its roots in version control,
and its goal of storing a wide range of media rather than just source code. The
acronym \gls{DMV} makes for a short and easy-to-type base command for command
line control, in the grand tradition of \lstinline{cvs}, \lstinline{svn},
\lstinline{git}, and \lstinline{hg}. And though in many places the acronym is
associated with a Department of Motor Vehicles, it does not seem to have any
prior conflicting uses in the computing domain.\footnote{Possibly because of
negative associations with the Department of Motor Vehicles} It is also a nod to
the author's home town of Washington DC, where the Washington metropolitan area
is sometimes referred to as "The DMV" as it spills out of the District of
Columbia and into Maryland and Virgina.
