\chapter{Idea: Distributed Media Versioning (DMV)}

\newterm{Distributed Media Versioning (DMV)} is our new low-level distributed
data storage platform. The core idea is relatively simple-- store data in a
Git-like DAG, but make the following changes:

\begin{tight_enumerate}

    \item{Store data at a finer granularity than the file}

    \item{Allow notes to store only a portion of the DAG as a whole}

\end{tight_enumerate}

Doing so allows a data set to be replicated or sharded across many nodes
according to the capacity of nodes and the needs of local users. The focus is on
data locality: tracking what data is where, presenting that information to the
user, and making it easy to transfer data to other nodes as desired. The
ultimate goal is to create a new abstraction, of \newterm{many devices, one data
item} in varying states of synchronization.


\paragraph{General storage}

DMV is a generalized storage platform that places no restriction on file types.
Its data model is a classic hierarchical filesystem, but with history.
Applications on each node can read and write to the files via the filesystem as
normal. DMV is dogmatic about the end-to-end argument \cite{endtoendargument},
that it cannot anticipate all the needs of end users and applications. So it
aims to be as general and neutral as possible, focusing on the core task of
storage and tracking, and providing a platform for other applications to build
on. DMV stores files as binary large objects (BLOBs), and it can handle a wide
variety of file sizes, from empty files to files tens of gigabytes in size. It
also stores a wide range of file quantities, from one to hundreds to millions.
Where the data set is too large to fit on one node, it can be spread over many
nodes, divided with the user's guidance according to data locality needs.


\paragraph{Based on version control}

DMV is inspired by distributed version control. Its core data structure is a
DAG, based on Git's but modified to store a wider range of file sizes. The key
difference is that large files are broken into smaller chunks (around
\SI{52}{\kibi\byte}) for easier handling. Breaking files into chunks also allows
the data structure to naturally de-duplicate parts of files that do not change.
For example, if a large media file has its metadata block updated, only the
chunk containing the updated metadata is new. The other chunks will simply be
reused.


\paragraph{Always writable}

Like in version control, the DAG structure records all history of the data set
and allows many different branches of development to exist in parallel. This
allows high availability. Any node can always make updates autonomously, without
coordinating with other nodes. Reconciliation of conflicting writes happens
later via merging. DMV only requires a connection to another node during
explicit synchronize operations, and so it is well-suited for applications with
intermittent or high-latency connectivity.


\paragraph{Configurable sharding}

The DAG structure tracks the data set in three dimensions:

\begin{tight_enumerate}
    \item The set of files themselves
    \item The history of the files
    \item The parallel branches of development in the history
\end{tight_enumerate}

Traditional distributed version control systems tend to assume that each replica
has the full history of all files, though not every branch of development. In
contrast, each DMV node can store a subset of the data along any of those
dimensions, configurable by the user. A node could keep the full history of only
a small subset of files, or only the most recent few versions of the full set of
files, or only a few branches, or any combination.
\todo{Be consistent about "node" vs "replica"}


\paragraph{Data integrity}

Because the DAG is append-only, and DAG objects are addressed by a cryptographic
hash of their contents, it is easy to verify data integrity and detect corrupted
objects. A corrupted object can easily be replaced by an intact copy from
another replica. DMV should never lose data accidentally. However, because DMV
tolerates an incomplete DAG, objects can be deliberately deleted from all nodes
to save space or to redact sensitive information.


\paragraph{Security Model}

The DAG's append-only nature and cryptographic content-addressing provide
protection from tampering. As long as the complete DAG is available, its
integrity can be verified. Allowing an incomplete DAG does introduce an opening
for tampering, because not all objects are available to verify, but we ignore
such a possibility for now. Because DMV is primarily meant to allow individuals
or organizations to manage their own data on their own hardware, we assume that
all nodes will be under the user's control, and that users will only accept
additions to the DAG from trusted collaborators. This makes the aforementioned
tampering less of a concern. It is also why we do not consider byzantine
failures or guard against malicious nodes. Data can be kept private by keeping
all DMV nodes on a private network.

%


\section{What's in a Name?}

We chose the name Distributed Media Versioning because it is a concise way to
describe the system, emphasizing its distributed nature, its roots in version
control, and its goal of storing a wide range of media rather than just source
code. The acronym DMV makes for a short and easy-to-type base command for
command line control, in the grand tradition of \lstinline{cvs},
\lstinline{svn}, \lstinline{git}, and \lstinline{hg}. And though in many places
the acronym is associated with a Department of Motor Vehicles, it does not seem
to have any prior conflicting uses in the computing domain.\footnote{Possibly
because of negative associations with the Department of Motor Vehicles} It is
also a nod to Michael's home town of Washington DC, where the Washington
metropolitan area is sometimes referred to as "The DMV" as it spills out of the
District of Columbia and into Maryland and Virgina.
