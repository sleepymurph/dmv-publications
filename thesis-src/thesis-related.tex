\chapter{Related Works}

\perrolfinline{Include systems you tested against as well: Git, Mercurial, Bup}

\section{Distributed storage and synchronization systems}

\paragraph{Camlistore}

Camlistore~\cite{camlistore_homepage} is an open-source project to create a
private long-term data storage system for personal users. It allows storage of
diverse types of data and it synchronizes between multiple replicas of the data
store. However, it eschews normal filesystems and creates its own schemas to
store various media.


\paragraph{Dat Data}

Dat~\cite{dat_homepage} is an open-source project for publishing and sharing
scientific data sets for research. This project has a lot of overlap with ours,
and several of the core ideas are similar, including breaking files into smaller
chunks, and tracking changes via a Git-like \gls{DAG}. However, their focus is
different. The Dat team is concentrating on publishing research data, and making
that specific task as simple as possible for non-technical researchers who might
not be familiar with version control. By contrast, our project operates at a
lower level of abstraction, offering the full power of version control in a very
general way, exposing and illuminating the complexities rather than trying to
hide them or automate them away.

Where Dat focuses on publishing on the open internet, we focus on ad-hoc
networks and data that may be private. Where Dat has components for automating
peer discovery and consensus, we work at a lower level, trying to perfect and
generalize the storage aspect first. Dat seems to assume that data sets will be
small enough to fit on a typical disk on a workstation, while we want to scale
even larger.

We hope that our system could be used as a base to build something like Dat, but
we intend for \gls{DMV} to be more general than the Dat core.


\paragraph{Eyo}

Eyo~\cite{Strauss:2011:EDP:2002181.2002216} is system for storing personal media
and synchronizing it between devices. It utilizes a Git-like content-addressed
\gls{objectstore} behind the scenes, but it works more like a networked filesystem
than version control. It focuses on organizing media by metadata, which requires
agreement on metadata formats, and it requires applications to be rewritten to
access files via Eyo rather than the filesystem, both of which are thorny and
ambitious problems. We focus purely on storage and synchronization.


\paragraph{Git-Annex and Git-Media}

Git-annex~\cite{git_annex_homepage} and Git-media~\cite{git_media_github} are
open-source projects that extend Git with special handling for larger files.
Both store information about the larger files in the normal Git \gls{repository}
and then store the files themselves in a separate location. Git-media stores all
the larger files in a separate data store which may be remote. Git-annex is more
flexible. Annex files may be spread across several different remote
\gls{repository} clones or data stores, and Git-annex has features for tracking
the locations of annex files in different remote \glspl{repository} and moving
them from one \gls{repository} to another. These tracking and distribution
features are very similar to our goals. However, Git-annex is not quite as
flexible as we aim for with \gls{DMV}. It considers the large files atomic
units, and it does not break them into smaller chunks for de-duplication. Also,
because metadata is processed by Git, it has the same limitations that Git does.
All \glspl{repository} must have all metadata, and performance suffers when
metadata is too large to fit into RAM.


\paragraph{IPFS: The Interplanetary Filesystem}

IPFS~\cite{ipfs_github_main} is an open-source pro\-ject to create a global
content-addressed filesystem. By its global nature, all files are stored
publicly, in a global network of nodes with global addressing. IPFS should be an
excellent resource for storing published information, but we want \gls{DMV} to
work with private data sets. We want individuals and organizations to be able
manage their own data stores privately on their own hardware.

It should be noted that IPFS does have support for storing private objects by
way of object-level encryption. However, this seems wasteful of disk space,
since small changes in the plain text of a file would completely change the
ciphertext, leaving no way to compress the redundancy.


\paragraph{Kademlia}

Kademlia~\cite{Maymounkov2002} is an advanced distributed hash table system that
updates its network topology information as part of normal lookups. Its focus is
on efficient routing between nodes, while we are focusing first on storage.


\paragraph{Rsync}
\label{related-rsync}

Rsync~\cite{rsynctechreport} is a utility that synchronizes files across a
low-band\-width network link by using a rolling hash to find similar parts of the
file, and then transfer only those portions of the file that have changed. Its
\glsfirst{rollinghash} is the inspiration for the chunk-splitting algorithms
used by Rsyncrypto~\cite{rsyncrypto_algorithm}, Bup~\cite{bup_design}, and
\gls{DMV} (\autoref{chunking-algorithm}). As a utility, Rsync focuses
exclusively on synchronizing files. It does not track history or the relation
between files.

Rsync coauthor Andrew Tridgell is also indirectly responsible for the
development of Git. His work to reverse-engineer the BitKeeper protocol is what
caused BitMover to revoke the special free BitKeeper license for Linux kernel
developers, which is what led Linus Torvalds to develop his own
\gls{SCM}~\cite{git_10_years_interview}.


\paragraph{Rsyncrypto}
\label{related-rsyncrypto}

Rsyncrypto~\cite{rsyncrypto_algorithm} is an encryption system that uses a
roll\-ing hash to encrypt files by chunks, so that the encrypted files are still
"rsync-able." Normally, a goal of an encryption algorithm is to have small
changes in the input lead to a completely different ciphertext. This is good for
security, but it negates the advantages of rsync --- if the entire file is
different, there are no advantages to be had in transferring only the changed
parts when synchronizing files. Rsync uses a rolling hash to break the file into
chunks and encrypt the chunks separately, so that a small change in the input
will only change the ciphertext for the chunk that contains it, and so rsync can
transfer only the chunk that changes.

Rsyncrypto's uses the same \glsfirst{rollinghash} as
Rsync~\cite{rsyncrypto_algorithm,rsynctechreport}, setting chunk boundaries where
the checksum is zero. \gls{DMV} uses this same chunking algorithm (see
\autoref{chunking-algorithm}).


\section{De-duplicating Storage and Backup}

\paragraph{Boar}

Boar~\cite{boar_homepage} is an open-source project to create a version control
system for large binary files. It is one of the main inspirations for our
project. It stores file versions in a content-addressed way, and provides
de-duplication for large files that only change in small pieces, and it can
truncate history to reclaim disk space. However, Boar retreats to the
centralized version control paradigm, with a central \gls{repository} that
working directories must connect to in order to check files in or out. We want
to provide the advantages of Boar in a flexible distributed version control
model. Boar also has practical limitations on \gls{repository} size and number
of files. \glspl{repository} are assumed to fit on one disk volume, and file
metadata is assumed to fit into Ram. \gls{DMV} tries to avoid both of those
limitations.


\paragraph{Bup}\label{related_bup}

Bup~\cite{bup_homepage} is an open-source file backup system that is based on
Git's \gls{repository} format. It does many of the things that \gls{DMV} does:
it breaks files into chunks by \gls{rollinghash}, and it has considerations for
metadata that is larger than RAM. Bup is one of the systems we evaluated along
with Git and Mercurial (\autoref{file-size-exp-desc}), and we determined that
Bup's manner of storing chunks is ideal for the next step for \gls{DMV}
(\autoref{chunk-then-recombine}). However, though Bup is built on
\glsdisp{VCS}{version control}, it is locked into a backup-based workflow.
History is linear and based on clock time of backup. And it assumes that the
whole data set and the whole \gls{repository} can fit onto one filesystem.
\gls{DMV} is built to be more general and versatile.


\paragraph{Time Machine}

Time Machine~\cite{timemachine_patent} is a backup utility from Apple, included
in Mac OS X Leopard, that makes frequent automated file hierarchy backups, using
filesystem hard links to de-duplicate unchanged files~\cite{timemachine_magic}.
This de-duplication allows it to store many different backup versions. Time
Machine's functionality can be mimicked by using Rsync with the
\lstinline{--link-dest} option, which also hard-links unchanged files during a
recursive sync~\cite{timemachine_foreveryunix}.
