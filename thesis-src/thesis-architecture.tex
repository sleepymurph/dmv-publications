\chapter{DMV Architecture}

\glsreset{repository}

Each \gls{DMV} node may have one or more \glspl{repository}, each consisting of
a \glsdisp{contentaddressablestorage}{content-addressable} \gls{objectstore} for
immutable \gls{DAG} objects and a \gls{workdir} for active file editing
(\autoref{fig:dia_obj_db_and_wd}). \Glspl{repository} that are used only for
storage may omit the \gls{workdir}, similar to a Git bare \gls{repository}.

\begin{figure}[h]
    \centering
    \includegraphics[width=0.5\textwidth]{dia_obj_db_and_wd}
    \caption{Repositories, object stores, and working directories}
    \label{fig:dia_obj_db_and_wd}
\end{figure}

Each \gls{repository} is autonomous, storing a portion of the \gls{DAG}, and
able to add to it at any time. However, it can transfer \gls{DAG} objects to and
from other \glspl{repository}, and it can cache data about what \gls{DAG}
objects are available at a remote \gls{repository}. Thus, \gls{DMV} forms an
ad-hoc, unstructured network of \glspl{repository}, and each \gls{repository}
can inform the user about what data is available where. \Glspl{repository} may
exist on separate computers or coexist on the same computer. Some
\glspl{repository} may be on removable storage and accessed only when that
storage is connected.

Together, the \glspl{repository} hold the entire data set --- or the portion of
the data set that has not been intentionally deleted --- in varying states of
sharding and replication. An example ad-hoc network of repositories is shown in
\autoref{fig:dia_architecture}.

\begin{figure}[h]
    \centering
    \includegraphics[width=0.95\textwidth]{dia_architecture}
    \caption{Repositories in an ad-hoc network}
    \label{fig:dia_architecture}
\end{figure}

\gls{DMV} assumes that each \gls{repository} will connect to a human-scale
number of other \glspl{repository}, maybe tens or hundreds. \gls{DMV} does not
dictate network structure. The user or a higher-level application may determine
the network topology and workflow according to their needs.

\gls{DMV} operates as a command-line utility, performing disk operations and
acting as a client to remote repositories according to explicit user commands.
The command-line utility is built as a thin shell around a library, so that
applications can use it as well. An optional server component will listen for
incoming connections from the command-line utility.
