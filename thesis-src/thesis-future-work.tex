\chapter{Future Work}

The primary goal of \gls{DMV} is to be a general low-level storage platform for
storing and tracking a data set across many nodes. The next step (after
\glsdisp{packfile}{packing} objects) would be to complete the networking
features of the prototype so that it can actually be used in real-life
applications.

Applications could potentially layer additional technologies on top of \gls{DMV}
to create interesting systems. For example, a gossip protocol could spread
information about object availability on remote data stores that are not
directly connected, allowing data to be spread and transfered across far-flung
networks. To focus on usability, daemons could automatically \gls{commit}
changes and sync with other nodes, shielding users from the complexity of
branching where possible and trying to present a coherent current state of the
data set. An important optimization would be to create a virtual filesystem that
is a view into a \gls{tree} in the \gls{DMV} repository. A virtual filesystem
could be used as the working directory, eliminating the wasted disk space of
having a second writable copy of all files, and it would eliminate the copying
of those files back into the immutable data store on \gls{commit}. Such a
virtual filesystem would also make it much easier to write an auto-\gls{commit}
daemon, since writes would have to go through the filesystem.

We look forward to continuing to work on and expand the system. The \gls{DMV}
project is open source, and development continues online at \dmvurl~.
