\documentclass[usenglish]{nik}

% File input and fonts
\usepackage[utf8]{inputenc}
\usepackage{mathptm}                % Times New Roman (text and math)

% Let us check for draft/final conditions
\usepackage{ifdraft}

% Figures
\usepackage{graphicx}
\usepackage[margin=1cm,justification=centering]{caption}
\usepackage{subcaption} % sub-figures

\ifoptionfinal{}{
\usepackage{endfloat}   % Move figures out of text for now
}


% Listings
\usepackage{listings}
\lstset{basicstyle=\ttfamily\footnotesize,breaklines=false}


% Numbers and units of measurement
\usepackage[binary-units]{siunitx}

\newcommand{\gib}{\gibi\byte}
\newcommand{\mib}{\mebi\byte}
\newcommand{\kib}{\kibi\byte}


%%% Typographical tweaks

\usepackage{booktabs}               % Nicer hlines (\midrule) in tables
\usepackage[inline]{enumitem}       % Compact inline lists with enumerate*

% An enumerated list with less spacing between items
\newenvironment{tight_enumerate}{
\begin{enumerate}
  \setlength{\itemsep}{0pt}
  \setlength{\parskip}{0pt}
}{\end{enumerate}}



%%% Bibliography

\usepackage[
  backend=bibtex,
  urldate=long, % use date like "Apr. 7, 2017" instead of American "04/07/2017"
  firstinits=true,    % use only initials of given names
]{biblatex}

\addbibresource{dmv-nik.bib}
% Bib names last-first
\DeclareNameAlias{sortname}{last-first}
\DeclareNameAlias{default}{last-first}
% Bib names small caps
\renewcommand{\mkbibnamelast}[1]{\textsc{#1}}



%%% TODO notes

\usepackage[obeyFinal]{todonotes}
\ifoptionfinal{}{
    \setlength{\marginparwidth}{2.5cm}
}

% towrite / written
% Write points that need to be made in a "towrite", then change to "written"
% when its done.
% Add `final` to the document to disable all.
\newcommand\towrite[1]{
    \todo[inline,color=white,bordercolor=white]{\textcolor{blue}{Write: #1}}
}
\newcommand\written[1]{%
    \ifoptionfinal{}{
        {
            \setlength{\parindent}{0em}
            \par
            \vspace{.5em}
            \textcolor{gray}{Written: #1}
            \vspace{.5em}
            \par
        }
    }
}

% Feedback from specific people
\newcommand{\perotto}[1]{\todo[color=blue!40]{Otto: #1}}
\newcommand{\perottoinline}[1]{\todo[color=blue!40,inline]{Otto: #1}}
\newcommand{\askotto}[1]{\todo[color=violet!40]{Ask Otto: #1}}
\newcommand{\askottoinline}[1]{\todo[color=violet!40,inline]{Ask Otto: #1}}



%%% ↑↑↑ Hyperref unaware packages above this line ↑↑↑ %%%%

% The hyperref docs recommend declaring it after the other \usepackage
% declarations, because it has to redefine several commands to work properly,
% and other later redefinitions might interfere.
%
% However, other packages are aware of hyperref and ask to be declared AFTER it.

% Link URLs in the PDF, and link references within the PDF itself
\usepackage[
  hidelinks,    % Do not style links. I think this is classier.
  pdfusetitle,  % Use doc title metadata for PDF title metadata
  pdfdisplaydoctitle,           % Display document title instead of filename in title bar
  bookmarksnumbered,            % use section numbers in PDF index
  pdfpagemode={UseOutlines},    % Show bookmarks
  pdfstartview={FitV},          % Fit height of page to PDF viewer window
]{hyperref}

%%% ↓↓↓ Hyperref aware packages below this line ↓↓↓ %%%%



% Combined references (chapters 1, 2 and 3)
\usepackage{cleveref}

% Glossaries
\usepackage[toc]{glossaries}
\makeglossaries
% Glossary entries
% Style: Capitalize first letter of all descriptions,
%   but don't capitalize all initialze in acronym unless it's a proper name
%   or if it's not obvious where the acronym is from.


% Dist sys

\newacronym[
    description={Atomicity, consistency, durability, and isolation, the
    guarantees of a traditional database commit},
]{ACID}{ACID}{atomicity, consistency, durability, and isolation}

\newglossaryentry{CAP-theorem}{
    name={CAP-theorem},
    description={The fundamental theorem of distributed systems, that no system
    can simultaneously be consistent (C), available (A), and tolerant of network
    partitions (P)},
    see={partitiondecision},
}

\newglossaryentry{partitiondecision}{
    name={partition decision},
    description={The dilemma faced by a distributed system during a network
    partition: to decrease availability or risk inconsistency},
    see={CAP-theorem},
}

\newglossaryentry{endtoendargument}{
    name={end-to-end argument},
    description={When designing a communications system, the idea that there is
    certain functionality that can only be implemented correctly by the
    higher-level application at the endpoints of the communication, and so it is
    futile for the communication system to try to provide that functionality
    itself},
}

% General Version Control

\newacronym[
    description={Distributed version control system, such as Git, where
    individual repositories can operate independently without having to connect
    to a central repository},
    see={VCS},
]
{DVCS}{DVCS}{distributed version control system}

\newacronym[
    description={Version control system, a program that stores many versions of
    a file or set of files, commonly used to track changes to source code},
    see={SCM},
]
{VCS}{VCS}{version control system}

\newacronym[
    description={Source code manager, a version control system that is designed
    primarily to store source code},
    see={VCS},
]
{SCM}{SCM}{source code manager}


% Specific systems

\newacronym[
    description={Distributed Media Versioning, the new distributed data storage
    platform described and introduced in this dissertation}
]
{DMV}{DMV}{Distributed Media Versioning}


% VCS architecture

\newglossaryentry{repository}{
    name={repository},
    plural={repositories},
    description={A location where data is stored in a version control system.
    Early systems would have a central repository that clients would check out
    from. In distributed version control, every client is a separate
    repository},
}

\newglossaryentry{objectstore}{
    name={object store},
    description={Content-addressable storage for DAG objects},
    see={DAG},
}

\newglossaryentry{workdir}{
    name={working directory},
    description={A directory where files that are tracked by a version control
    system are actively worked on and edited},
}

\newglossaryentry{branch}{
    name={branch},
    plural={branches},
    description={In a version control system, separate concurrent lines of
    update history},
    see={head,merge},
}

\newglossaryentry{head}{
    name={head},
    description={In a version control system, the most recent revision in a
    branch},
    see={branch},
}

\newglossaryentry{merge}{
    name={merge},
    description={In a version control system, an operation that combines two
    branches and reconciles conflicting changes},
    see={branch},
}


% DAG

\newglossaryentry{contentaddressablestorage}{
    name={content addressable storage},
    description={Storage that stores immutable objects named by a hash of their
    content, which naturally de-duplicates identical objects},
}

\newacronym[
    description={Directed acyclic graph, the type of graph data structure used
    to represent history in many distributed version control systems. Directed
    meaning all the edges have a direction, from one node to another, and
    acyclic meaning that there are no cycles, no paths that revisit any node},
    see={blob,chunkedblob,tree,commit,ref},
]
{DAG}{DAG}{directed acyclic graph}

\newacronym[
    description={Binary large object, a sequence of unstructured binary data. In
    Git and DMV, a DAG object holding file data},
    first={blob (binary large object)},
    see={DAG},
]
{blob}{blob}{binary large object}

\newglossaryentry{chunkedblob}{
    name={chunked blob},
    description={In DMV, a DAG object that is an index of blobs that make up a
    larger blob},
    see={DAG,blob},
}

\newglossaryentry{tree}{
    name={tree},
    description={In Git and DMV, a DAG object representing a particular state of
    a file hierarchy},
    see={DAG},
}

\newglossaryentry{commit}{
    name={commit},
    description={In version control, the operation for storing a particular
    version of the data. Also, the resulting DAG object that represents that
    version in the history},
    see={DAG},
}

\newacronym[
    description={A reference to a commit object in the DAG},
    see={DAG},
]
{ref}{ref}{reference}

\newglossaryentry{packfile}{
    name={pack file},
    description={An object store file format that aggregates many objects in one
    file},
    see={objectstore},
}

\newglossaryentry{filelog}{
    name={filelog},
    description={Mercurial's file format that stores different versions of the
    same file as a base version followed by a series of delta},
}


% Rolling Hash

\newglossaryentry{rollinghash}{
    name={rolling hash},
    first={rolling hash algorithm},
    description={A hash checksum that operates over a moving window of data in a
    byte stream that can be used to find repeating patterns},
    see={windowsize,divisor},
}

\newglossaryentry{windowsize}{
    name={window size},
    symbol={\ensuremath{w}},
    description={In a rolling hash algorithm, the number of previous bytes used
    in the rolling sum},
    see={rollinghash,divisor},
}

\newglossaryentry{divisor}{
    name={divisor},
    symbol={\ensuremath{d}},
    description={In a rolling hash algorithm, the divisor in the modulus
    operation. A chunk boundary is created when the sum of the bytes in the
    window, modulo this divisor, is equal to zero},
    see={rollinghash,windowsize},
}


% low-level

\newglossaryentry{inode}{
    name={inode},
    description={A data structure in a Unix filesystem that stores file
    metadata. Each filesystem has a fixed number of inodes, which limits the
    total number of files and directories the filesystem can hold},
}

\newacronym[
    description={Application binary interface, the public interface between a
    system library and a client application}
]{ABI}{ABI}{application binary interface}

\newacronym[
    description={random number generator}
]{RNG}{RNG}{random number generator}

% Emphasize first use of glossary term
\defglsentryfmt[main]{\ifglsused{\glslabel}{\glsgenentryfmt}{\emph{\glsgenentryfmt}}}
% Disable glossary hyperlinks because we will not print a glossary in this paper
\glsdisablehyper



% Repeated text snippets

\newcommand{\explainlogsubfig}{

    Subfigure (a) shows the full range on a logarithmic scale, while the others
    are linear-scale for specific ranges and include error bars.

}

\newcommand{\explaindiskspaceplot}{

    Subfigure (a) shows repository size on a logarithmic scale, while subfigure
    (b) shows the ratio of total repository size to input data size.

}

\newcommand{\muninurl}{\url{http://munin.uit.no}}
\newcommand{\dmvurl}{\url{http://dmv.sleepymurph.com/}}

% towrite / written
% Write points that need to be made in a "towrite", then change to "written"
% when its done.
% Add `final` to the document to disable all.
\newcommand\towrite[1]{
    \todo[inline,color=white,bordercolor=white]{\textcolor{blue}{Write: #1}}
}
\newcommand\written[1]{%
    \ifoptionfinal{}{
        ~\\
        \textcolor{gray}{Write: #1}
        ~\\
    }
}


% An enumerated list with less spacing between items
\newenvironment{tight_enumerate}{
\begin{enumerate}
  \setlength{\itemsep}{0pt}
  \setlength{\parskip}{0pt}
}{\end{enumerate}}


\title{DMV: Distributed Media Versioning across devices}
\author{Michael J. Murphy \and Otto J. Anshus \and John Markus Bjørndalen}
\date{November 2017}

\begin{document}
\maketitle

\begin{abstract}
\input{dmv-nik-abstract.txt}
\end{abstract}

\section*{OUTLINE}

\begin{verbatim}

- Problem: many devices, more data, difficult to follow what is where

- Cloud not solution. Relies on third party. Connection, privacy, etc.

- DVCS: An alternate approach

    - Extreme availability
    - Version history gives chain of causality for later reconciliation

- Interesting properties of DAG

    - DAG gives de-duplication
    - Content addressing gives tampering/bitrot protection
    - DAG also gives convenient ways to shard data

- Problem 1: dealing with larger files: chunking

    - Bup has chunking but locked into backup workflow

- Problem 2: dealing with many files: packing

    - Git has packing but in separate step that fails for large files

- Problem 3: increasing data: sharding

- In-between: de-duplication

- DMV prototype

- Experiments

    - File size and number of files
    - Random writes

- Results

- Conclusion

    - chunk, content-address, re-pack
    - DMV not yet viable, but it's a start

\end{verbatim}

\section{Introduction}

\written{Distributed version control is an interesting form of distributed system because it takes eventual consistency to the extreme.}

Distributed systems are ruled by the \gls{CAP-theorem}~\cite{cap_origin}, which states that a system cannot be completely consistent (C), available (A), and tolerant of network partitions (P) all at the same time.
When communication between replicas breaks down and they cannot all acknowledge an operation, the system is faced with "the \gls{partitiondecision}: block the operation and thus decrease availability, or proceed and thus risk inconsistency."~\cite{cap_years_later}

Much research is aimed at improving consistency.
Vector clocks~\cite{lamport_ordering} and consensus algorithms such as Paxos~\cite{paxos_made_simple,paxos_made_moderately_complex} make sure the same updates are applied in the same order on all replicas even, if a minority of nodes cannot respond.
There are also data types are cleverly designed to be commutative, so that the resulting data will be the same regardless of the order in which updates are applied~\cite{crdt_orig}.
But in general, when systems cannot communicate, the CAP theorem cannot be avoided~\cite{cap_proof}, and the system is still faced with the \gls{partitiondecision}.

\written{Every replica of a repository contains the full history in an append-only data structure, any replica may add new commits, and conflicting updates are reconciled later in a merge operation.}

Though maybe not designed with the CAP theorem explicitly in mind, a \gls{DVCS} is in fact a small-scale distributed system that takes the availability-first approach to the extreme.
Rather than a set of connected nodes that may occasionally lose contact in a network partition, a \gls{DVCS}'s \glspl{repository} are self-contained and offline by default.
They allow writes to local data at any time, and only connect to other \glspl{repository} intermittently by user command to exchange updates.
Concurrent updates are not only allowed but embraced as different \glspl{branch} of development.
A \gls{DVCS} can track many different \glspl{branch} at the same time, and conflicting \glspl{branch} can be combined and resolved by the user in a \gls{merge} operation.

The \glsdisp{DVCS}{distributed version control} concept may have something to
teach larger-scale systems about availability.

\written{These systems are popular, but their use is generally limited to the small text files of source code.}

\Glspl{DVCS} are designed primarily to store program source code: plain text files in the range of tens of kilobytes.
Checking in larger binary files such as images, sound, or video affects performance.
Actions that require copying data in and out of the system slow from hundredths of a second to full seconds or minutes.
And since a \gls{DVCS} keeps every version of every file in every \gls{repository}, forever, the disk space needs compound.

This has lead to a conventional wisdom that binary files should never be stored in version control, inspiring blog posts with titles such as
"Don't ever commit binary files to Git! Or what to do if you do"~\cite{dont_ever_commit_binaries_to_version_control},
even as the modern software development practice of continuous delivery was commanding teams to "keep absolutely everything in version control."~\cite[p.33]{continuousdeliverybook}

\towrite{This paper explores the challenges of using version control to store larger binary files, with the goal of building a scalable, highly-available, distributed storage system for media files such as images, audio, and video.}

\subsection{The Power of the DAG}

Git stores data in a directed acyclic graph (DAG) data structure~\cite{git_initial_readme}.
Each version of each file is hashed with the cryptographic SHA-1 digest, becoming a blob object, which is stored in an object store with the SHA-1 hash as its ID.
Directory states are stored by creating a list of hash IDs for each file in the directory, a tree object, and also storing it by SHA-1 hash ID.
Tree objects can also refer to other trees, representing subdirectories.
Commit objects contain the hash ID of the tree object representing the directory state at the time of commit, plus metadata such as the author and commit message.
The resulting graph is directed because the links between objects are directional.
It is acyclic because objects are content-addressed.
An object can only refer to another object by hash, so it must refer to an existing object whose hash is known.
And objects cannot be updated without changing their hash.
Therefore, it is impossible to create a circular reference.

This DAG data structure has several interesting properties for distributed data storage.
The content-addressing naturally de-duplicates identical objects, since identical objects will have the same hash ID.
This results in a natural compression of redundant objects.
The append-only nature of the DAG allows replicas to make independent updates without disturbing the existing history.
Then, when transferring updates from one replica to another, only new objects need to be transferred.
Concurrent updates will result in multiple branches of history, but references from child commit to parent commit establish a happens-before relationship and give a chain of causality.
Data integrity can also be verified by re-hashing each object and comparing to its ID, protecting against tampering and bit rot.
Updates can also be made atomic by waiting to update branch head references until after all new objects are written to the object store.

The efficiency of de-duplication depends on how well identical pieces of data map to identical objects.
In Git, the redundant objects are the files and directories that do not change between commits.
De-duplication of redundant data within files is accomplished by aggregating objects together into pack files and compressing them with zlib \cite[Section 10.4]{git_book}.

Calculating deltas during this \glsdisp{packfile}{packing} phase requires loading the objects into memory, and so it can cause an out-of-memory error if an object is too large to fit into available RAM.
Because Git stores files whole in \glspl{blob}, it cannot \glsdisp{packfile}{pack} objects that are larger than available RAM.

If the \gls{DAG} operated at a granularity smaller than the file, it could become even more powerful.
It could naturally de-duplicate chunks of files the way that Git already de-duplicates whole files, and it could ensure that all objects fit into RAM for \glsdisp{packfile}{packing} or other operations.

This sub-file granularity and de-duplication is the core idea behind our new data storage system, \acrlong{DMV}.

%

\section{Architecture}

\section{Design}

\begin{figure}[]
    \centering
    \includegraphics[width=0.9\textwidth]{dia_dmv_dag_example_three_commits}
    \caption{A simple DMV DAG with three commits}
    \label{dia_dmv_dag_example_three_commits}
\end{figure}



\newcommand{\slicediagramwidth}{0.45\textwidth}

\begin{figure}[]

    \centering

    \begin{subfigure}[]{\slicediagramwidth}
        \includegraphics[width=\textwidth]{dia_dmv_dag_slice_partial_history}
        \caption{Partial history of full data set}
        \label{dia_dmv_dag_slice_partial_history}
    \end{subfigure}
    ~
    \begin{subfigure}[]{\slicediagramwidth}
        \includegraphics[width=\textwidth]{dia_dmv_dag_slice_history_of_subset}
        \caption{Full history of part of data set}
        \label{dia_dmv_dag_slice_history_of_subset}
    \end{subfigure}
    ~
    \begin{subfigure}[]{\slicediagramwidth}
        \includegraphics[width=\textwidth]{dia_dmv_dag_slice_history_of_metadata}
        \caption{Full history of metadata}
        \label{dia_dmv_dag_slice_history_of_metadata}
    \end{subfigure}

    \caption{A DMV DAG, sliced in different dimensions}
\end{figure}

\section{Implementation}

\towrite{Talk about prototype implementation}

\section{Experiments}

\subsection{Methodology}

We conducted two major experiments.
In order to measure the effect of file size, we committed a single file of increasing size to a each target \gls{VCS}.
\perotto{Effect on what?}
And to measure the effect of numbers of files, we committed increasing number of small (\SI{1}{\kibi\byte}) files to each target \gls{VCS}.
\perotto{Effect on what?}

We ran each experiment with four different \glspl{VCS}: Git, Mercurial, Bup, and the DMV prototype.
We chose Git because it is the most popular \gls{DVCS} in use today~\cite{what_are_devs_talking_about} and the main inspiration for DMV.
\askotto{You write "Refs ref? [1?]" here. I'm not sure what you mean. Can you clarify?}
We chose the Mercurial and Bup because they are both related to Git but each store data differently.
Git and DMV both store objects in an \gls{objectstore} directory as a file named for its hash ID.
Git has a separate garbage collection step that takes object files and aggregates them into \glspl{packfile}~\cite[Section 10.7]{git_book}.
Mercurial stores revisions of each file as a base revision followed by a series of deltas~\cite[Chapter 4]{hgbook}, much like previous \glspl{VCS} such as RCS, CVS, and Subversion~\cite{history_of_version_control}.
Bup uses Git's exact data model and \gls{packfile} format, but Bup breaks files into chunks using a \gls{rollinghash}, reusing Git's \gls{tree} object as a \gls{chunkedblob} index\footnotemark.
Unlike Git, Bup writes to the \gls{packfile} format directly, without Git's separate commit and pack steps, and without bothering to calculate deltas~\cite{bup_design}.
As a control, we also ran the experiments with a dummy \gls{VCS} that simply copied the files to a hidden directory.
\perjmb{Do we have a name for the dummy VCS? Mike: No, I just refer to it as "copy"}

\footnotetext{Git can read a repository written by Bup, but it will see
the large file as a directory full of smaller chunk files.}

For each experiment, the procedure for a single trial was as follows:
\begin{tight_enumerate}
    \item Create an empty \gls{repository} of the target \gls{VCS} in a temporary directory
    \item Generate target data to store, either a single file of the target size, or the target number of \SI{1}{\kibi\byte} files
    \item \Gls{commit} the target data to the \gls{repository}, measuring wall-clock time to \gls{commit}
    \item Verify that the first \gls{commit} exists in the \gls{repository}, and if there was any kind of error, run the \gls{repository}'s integrity check operation
    \item Measure the total \gls{repository} size
    \item Overwrite a fraction (\num{1/1024}) of each target file
    \item (Number-of-files experiment only) Run the \gls{VCS}'s status command that lists what files have changed, and measure the wall-clock time that it takes to complete
    \item \Gls{commit} again, measuring wall-clock time to \gls{commit}
    \item Verify that the second \gls{commit} exists in the \gls{repository}, and if there was any kind of error, run the \gls{repository}'s integrity check operation
    \item Measure the total \gls{repository} size again
    \item (File-size experiment only, Git only) Run Git's garbage collector (\lstinline{git fsck}) to pack objects, then measure total \gls{repository} size again
    \item Delete temporary directory and all trial files
\end{tight_enumerate}

We increased file sizes exponentially by powers of two from \SI{1}{\byte} up to \SI{128}{\gibi\byte}, adding an additional step at \num{1.5} times the base size at each order of magnitude.
For example, on the megabyte scale, the file sizes are \SI{1}{\mebi\byte}, \SI{1.5}{\mebi\byte}, \SI{2}{\mebi\byte}, \SI{3}{\mebi\byte}, \SI{4}{\mebi\byte}, \SI{6}{\mebi\byte}, \SI{8}{\mebi\byte}, \SI{12}{\mebi\byte}, and so on.

We increased numbers of files exponentially by powers of ten from one file to ten million files, adding additional steps at \num{2.5}, \num{5}, and \num{7.5} times the base number at each order of magnitude.
For example, at the hundreds and thousands scales, the file quantities are \num{100}, \num{250}, \num{500}, \num{750}, \num{1000}, \num{2500}, \num{5000}, \num{7500}, \num{10000}, and so on.

Input data files consisted of pseudorandom bytes taken from the operating system's pseudorandom number generator (\lstinline{/dev/urandom} on Linux).

%

\subsection{Experiment Platform}

We ran the trials on four dedicated computers with no other load.
Each was a typical office desktop with a \SI{3.16}{\giga\hertz} \num{64}-bit dual-core processor and \SI{8}{\gibi\byte} of RAM, running Debian version 8.6 ("Jessie").
Each computer had one normal SATA hard disk (spinning platter, not solid-state), and trials were conducted on a dedicated \SI{197}{\gibi\byte} LVM partition formatted with the ext4 filesystem.
All came from the same manufacturer with the same specifications and were, for practical purposes, identical.
%Additional details can be found in \autoref{test-machine-specs}.
\todo{Include platform table?}

We ran every trial four times, once on each of the experiment computers, and took the mean and standard deviation of each time and disk space measurement.
However, because the experiment computers are practically identical, there was little variation.

%

\section{Results}

\subsection{File Size}

\subsubsection{File Size Limits: RAM, Time, Disk Space}

In the experiments, both Git and Mercurial had file size limits that were related to the size of RAM.
Mercurial refused to commit a file \SI{2}{\gib} or larger.
It exited with an error code and print an error message saying "up to 6442 MB of RAM may be required to manage this file."
This is because Mercurial stores file revisions as deltas against a base revision, so it has to do its delta calculation up front.
It loads each revision of the file into memory to do the calculations, plus it allocates memory to write the output.
As a result, Mercurial needs to be able to fit the file into memory three times over in order to commit it.
We saw that in each case, the commit was not stored, and the repository was left unchanged.
Mercurial commits are atomic.

Git's commit operation appeared to fail with files \SI{12}{\gib} and larger.
It exited with an error code and print an error message saying "fatal: Out of memory, malloc failed (tried to allocate 12884901889 bytes)."
However, the commit was be written to the repository, and git's \lstinline{fsck} operation reported no errors.
So the commit operation completes successfully, even though an error is reported.
\perotto{Is this A, D, or I}

With files \SI{24}{\gib} and larger, Git's \lstinline{fsck} operation itself failed.
In each case, the \lstinline{fsck} command exited with an error code and give a similar "fatal ... malloc" error.
However, the \SI{24}{\gib} file could still be checked out from the repository without error.
So we continued the trials assuming that these were also false alarms.

Git's delta compression takes place in a separate garbage collection step.
For Git, we ran the garbage collector at the end of each trial and measured repository size before and after garbage collection.
With file sizes up to and including \SI{1}{\gib}, the garbage collection resulted in a reduction in repository size from approximately three times the input data size (the input file, and two separately stored revisions) to approximately twice the input data size (the input file, and compressed stored revisions).
At \SI{1.5}{\gib} and above, the repository size remained approximately three times the input data size after garbage collection.
This indicates that Git's delta compression also requires that the file be able to fit into disk space three times over.

The \gls{DMV} prototype was able to store a file up to \SI{64}{\gibi\byte} in size, but time became a limiting factor as file size increased.
We set an arbitrary five and a half hour timeout for commits in our experiment script.
At \SI{96}{\gib}, the DMV commit operation hit this limit and was terminated.

The largest file size committed in the trials was \SI{96}{\gib}.
This was a limitation of the experiment environment, not a limit of the systems under test.
The experiments were performed on a \SI{197}{\gib} partition.
The next trial size \SI{128}{\gib} is too large to fit two copies on the partition.
So no version control system was able to commit it, since they all save a copy of the file.

Bup was able to store a \SI{96}{\gibi\byte} file with no errors in just under two hours.
Git could also store such a large file, but one must ignore the false-alarm "fatal" errors being reported by the user interface.

These findings are summarized in \autoref{file-sizes-table}.

\begin{table}[]
    \caption{Observations as file size increases}
    \label{file-sizes-table}
    \centering
    \begin{tabular}{r l}
        Size & Observation \\
        \midrule
        \SI{1.5}{\gibi\byte} & Largest successful commit with Mercurial \\
        \SI{1.5}{\gibi\byte} & Git commit successful, but garbage collection fails to compress \\
        \SI{2}{\gibi\byte} & Mercurial commit rejected \\
        \SI{8}{\gibi\byte} & Largest successful commit with Git \\
        \SI{12}{\gibi\byte} & Git false-alarm errors begin, but commit still intact \\
        \SI{16}{\gibi\byte} & Largest successful Git fsck command \\
        \SI{24}{\gibi\byte} & Git false-alarm errors begin during fsck, but commit still intact \\
        \SI{64}{\gibi\byte} & Largest successful DMV commit \\
        \SI{96}{\gibi\byte} & DMV timeout after \SI{5.5}{\hour} \\
        \SI{96}{\gibi\byte} & Last successful commit with Bup (and Git, ignoring false-alarm errors) \\
        \SI{128}{\gibi\byte} & All fail due to size of test partition \\
    \end{tabular}
\end{table}

%

\subsubsection{Commit Times for Increasing File Sizes}

\autoref{fig:plot-file-size--c1-time} shows the wall-clock time required for the initial \gls{commit}, adding a single file of the given size to a fresh \gls{repository}.
Over all, the trend is clear and unsurprising: \gls{commit} time increases with file size.
It increases linearly for Git, Mercurial, and Bup.
DMV's commit times increase in a more parabolic fashion,
similar to how it and Git respond to increasing numbers of files.
This is because DMV breaks the large files into many smaller objects, trading the large file problem for the many file problem.


%



\subsection{Number of Files}

\subsubsection{File Quantity Limits: inodes}

Git, Mercurial, DMV, and the copy operation all failed when trying to store \num{7.5} million files or more, reporting that the disk was full.
However, the disk was not actually out of space.
It was out of \glspl{inode}.

Unix filesystems, ext4 included, store file and directory metadata in a data structure called an \gls{inode}, which reside in a fixed-length table~\cite{unix_timesharing_system}.
When all of the \glspl{inode} in the table are allocated, the filesystem cannot store any more files or directories.
The number of inodes is tunable at filesystem creation by passing a bytes-per-inode parameter (\lstinline{-i}) to \lstinline{mke2fs}, but our experiment partitions used the default setting.
The \SI{197}{\gib} partitions had approximately \num{13} million inodes.

All systems tested except for Bup store a new copy of the input data, with one stored file per input file.
So committing \num{7.5} million input files would create an additional \num{7.5} million stored files, for a total of \num{15} million inodes, \num{2} million more than the \num{13} million on the filesystem.

Bup avoided the \gls{inode} limit because it writes directly into Git's pack file format.
It aggregates objects and conserves inodes.
Bup trials could continue until the input data itself exhausted the system's \glspl{inode} attempting to generate \num{25} million input files.

%


\subsubsection{Commit Times for Increasing Numbers of Files}

\autoref{fig:plot-num-files--c1-time} shows the time required for the initial \gls{commit}, storing all files into a fresh empty \gls{repository}.
Here we see the commit times for Git and DMV increasing quadratically with the number of files, while Mercurial, Bup, and the copy increase linearly.

\begin{figure}
    \centering
    \begin{minipage}{.5\textwidth}

        \caption{Wall-clock time to commit one large file to a fresh repository}
        \label{fig:plot-file-size--c1-time}
        \centering
        \includegraphics[width=\textwidth]{plot-file-size--c1-time}

    \end{minipage}%
    \begin{minipage}{.5\textwidth}

        \caption{Wall-clock time to commit many 1KiB files to a fresh repository}
        \label{fig:plot-num-files--c1-time}
        \centering
        \includegraphics[width=\textwidth]{plot-num-files--c1-time}

    \end{minipage}
\end{figure}

\section{Conclusion}

\towrite{We conclude that the key to storing large files is to break them into smaller chunks, and that the key to storing many small chunks is to aggregate them into larger files.}

\towrite{We intend to incorporate these insights into future versions of DMV.}


\printbibliography[]

\listoftodos

\end{document}
