\documentclass[
    usenglish,
    %final,
]{nik}

% File input and fonts
\usepackage[utf8]{inputenc}
\usepackage{mathptm}                % Times New Roman (text and math)

% Let us check for draft/final conditions
\usepackage{ifdraft}

% Figures
\usepackage{graphicx}
\usepackage[margin=1cm,justification=centering]{caption}
\usepackage{subcaption} % sub-figures

\ifoptionfinal{}{
\usepackage{endfloat}   % Move figures out of text for now
}


% Listings
\usepackage{listings}
\lstset{basicstyle=\ttfamily\footnotesize,breaklines=false}


% Numbers and units of measurement
\usepackage[binary-units]{siunitx}

\newcommand{\gib}{\gibi\byte}
\newcommand{\mib}{\mebi\byte}
\newcommand{\kib}{\kibi\byte}


%%% Typographical tweaks

\usepackage{booktabs}               % Nicer hlines (\midrule) in tables
\usepackage[inline]{enumitem}       % Compact inline lists with enumerate*

% An enumerated list with less spacing between items
\newenvironment{tight_enumerate}{
\begin{enumerate}
  \setlength{\itemsep}{0pt}
  \setlength{\parskip}{0pt}
}{\end{enumerate}}



%%% Bibliography

\usepackage[
  backend=bibtex,
  urldate=long, % use date like "Apr. 7, 2017" instead of American "04/07/2017"
  firstinits=true,    % use only initials of given names
]{biblatex}

\addbibresource{dmv-nik.bib}
% Bib names last-first
\DeclareNameAlias{sortname}{last-first}
\DeclareNameAlias{default}{last-first}
% Bib names small caps
\renewcommand{\mkbibnamelast}[1]{\textsc{#1}}



%%% TODO notes

\usepackage[obeyFinal]{todonotes}
\ifoptionfinal{}{
    \setlength{\marginparwidth}{2.5cm}
}

% towrite / written
% Write points that need to be made in a "towrite", then change to "written"
% when its done.
% Add `final` to the document to disable all.
\newcommand\towrite[1]{
    \todo[inline,color=white,bordercolor=white]{\textcolor{blue}{Write: #1}}
}
\newcommand\written[1]{%
    \ifoptionfinal{}{
        {
            \setlength{\parindent}{0em}
            \par
            \vspace{.5em}
            \textcolor{gray}{Written: #1}
            \vspace{.5em}
            \par
        }
    }
}

% Feedback from specific people
\newcommand{\perotto}[1]{\todo[color=blue!40]{Otto: #1}}
\newcommand{\perottoinline}[1]{\todo[color=blue!40,inline]{Otto: #1}}
\newcommand{\askotto}[1]{\todo[color=violet!40]{Ask Otto: #1}}
\newcommand{\askottoinline}[1]{\todo[color=violet!40,inline]{Ask Otto: #1}}



%%% ↑↑↑ Hyperref unaware packages above this line ↑↑↑ %%%%

% The hyperref docs recommend declaring it after the other \usepackage
% declarations, because it has to redefine several commands to work properly,
% and other later redefinitions might interfere.
%
% However, other packages are aware of hyperref and ask to be declared AFTER it.

% Link URLs in the PDF, and link references within the PDF itself
\usepackage[
  hidelinks,    % Do not style links. I think this is classier.
  pdfusetitle,  % Use doc title metadata for PDF title metadata
  pdfdisplaydoctitle,           % Display document title instead of filename in title bar
  bookmarksnumbered,            % use section numbers in PDF index
  pdfpagemode={UseOutlines},    % Show bookmarks
  pdfstartview={FitV},          % Fit height of page to PDF viewer window
]{hyperref}

%%% ↓↓↓ Hyperref aware packages below this line ↓↓↓ %%%%



% Combined references (chapters 1, 2 and 3)
\usepackage{cleveref}

% Glossaries
\usepackage[toc]{glossaries}
\makeglossaries
% Glossary entries
% Style: Capitalize first letter of all descriptions,
%   but don't capitalize all initialze in acronym unless it's a proper name
%   or if it's not obvious where the acronym is from.


% Dist sys

\newacronym[
    description={Atomicity, consistency, durability, and isolation, the
    guarantees of a traditional database commit},
]{ACID}{ACID}{atomicity, consistency, durability, and isolation}

\newglossaryentry{CAP-theorem}{
    name={CAP-theorem},
    description={The fundamental theorem of distributed systems, that no system
    can simultaneously be consistent (C), available (A), and tolerant of network
    partitions (P)},
    see={partitiondecision},
}

\newglossaryentry{partitiondecision}{
    name={partition decision},
    description={The dilemma faced by a distributed system during a network
    partition: to decrease availability or risk inconsistency},
    see={CAP-theorem},
}

\newglossaryentry{endtoendargument}{
    name={end-to-end argument},
    description={When designing a communications system, the idea that there is
    certain functionality that can only be implemented correctly by the
    higher-level application at the endpoints of the communication, and so it is
    futile for the communication system to try to provide that functionality
    itself},
}

% General Version Control

\newacronym[
    description={Distributed version control system, such as Git, where
    individual repositories can operate independently without having to connect
    to a central repository},
    see={VCS},
]
{DVCS}{DVCS}{distributed version control system}

\newacronym[
    description={Version control system, a program that stores many versions of
    a file or set of files, commonly used to track changes to source code},
    see={SCM},
]
{VCS}{VCS}{version control system}

\newacronym[
    description={Source code manager, a version control system that is designed
    primarily to store source code},
    see={VCS},
]
{SCM}{SCM}{source code manager}


% Specific systems

\newacronym[
    description={Distributed Media Versioning, the new distributed data storage
    platform described and introduced in this dissertation}
]
{DMV}{DMV}{Distributed Media Versioning}


% VCS architecture

\newglossaryentry{repository}{
    name={repository},
    plural={repositories},
    description={A location where data is stored in a version control system.
    Early systems would have a central repository that clients would check out
    from. In distributed version control, every client is a separate
    repository},
}

\newglossaryentry{objectstore}{
    name={object store},
    description={Content-addressable storage for DAG objects},
    see={DAG},
}

\newglossaryentry{workdir}{
    name={working directory},
    description={A directory where files that are tracked by a version control
    system are actively worked on and edited},
}

\newglossaryentry{branch}{
    name={branch},
    plural={branches},
    description={In a version control system, separate concurrent lines of
    update history},
    see={head,merge},
}

\newglossaryentry{head}{
    name={head},
    description={In a version control system, the most recent revision in a
    branch},
    see={branch},
}

\newglossaryentry{merge}{
    name={merge},
    description={In a version control system, an operation that combines two
    branches and reconciles conflicting changes},
    see={branch},
}


% DAG

\newglossaryentry{contentaddressablestorage}{
    name={content addressable storage},
    description={Storage that stores immutable objects named by a hash of their
    content, which naturally de-duplicates identical objects},
}

\newacronym[
    description={Directed acyclic graph, the type of graph data structure used
    to represent history in many distributed version control systems. Directed
    meaning all the edges have a direction, from one node to another, and
    acyclic meaning that there are no cycles, no paths that revisit any node},
    see={blob,chunkedblob,tree,commit,ref},
]
{DAG}{DAG}{directed acyclic graph}

\newacronym[
    description={Binary large object, a sequence of unstructured binary data. In
    Git and DMV, a DAG object holding file data},
    first={blob (binary large object)},
    see={DAG},
]
{blob}{blob}{binary large object}

\newglossaryentry{chunkedblob}{
    name={chunked blob},
    description={In DMV, a DAG object that is an index of blobs that make up a
    larger blob},
    see={DAG,blob},
}

\newglossaryentry{tree}{
    name={tree},
    description={In Git and DMV, a DAG object representing a particular state of
    a file hierarchy},
    see={DAG},
}

\newglossaryentry{commit}{
    name={commit},
    description={In version control, the operation for storing a particular
    version of the data. Also, the resulting DAG object that represents that
    version in the history},
    see={DAG},
}

\newacronym[
    description={A reference to a commit object in the DAG},
    see={DAG},
]
{ref}{ref}{reference}

\newglossaryentry{packfile}{
    name={pack file},
    description={An object store file format that aggregates many objects in one
    file},
    see={objectstore},
}

\newglossaryentry{filelog}{
    name={filelog},
    description={Mercurial's file format that stores different versions of the
    same file as a base version followed by a series of delta},
}


% Rolling Hash

\newglossaryentry{rollinghash}{
    name={rolling hash},
    first={rolling hash algorithm},
    description={A hash checksum that operates over a moving window of data in a
    byte stream that can be used to find repeating patterns},
    see={windowsize,divisor},
}

\newglossaryentry{windowsize}{
    name={window size},
    symbol={\ensuremath{w}},
    description={In a rolling hash algorithm, the number of previous bytes used
    in the rolling sum},
    see={rollinghash,divisor},
}

\newglossaryentry{divisor}{
    name={divisor},
    symbol={\ensuremath{d}},
    description={In a rolling hash algorithm, the divisor in the modulus
    operation. A chunk boundary is created when the sum of the bytes in the
    window, modulo this divisor, is equal to zero},
    see={rollinghash,windowsize},
}


% low-level

\newglossaryentry{inode}{
    name={inode},
    description={A data structure in a Unix filesystem that stores file
    metadata. Each filesystem has a fixed number of inodes, which limits the
    total number of files and directories the filesystem can hold},
}

\newacronym[
    description={Application binary interface, the public interface between a
    system library and a client application}
]{ABI}{ABI}{application binary interface}

\newacronym[
    description={random number generator}
]{RNG}{RNG}{random number generator}

% Emphasize first use of glossary term
\defglsentryfmt[main]{\ifglsused{\glslabel}{\glsgenentryfmt}{\emph{\glsgenentryfmt}}}
% Disable glossary hyperlinks because we will not print a glossary in this paper
\glsdisablehyper



% Repeated text snippets

\newcommand{\explainlogsubfig}{

    Subfigure (a) shows the full range on a logarithmic scale, while the others
    are linear-scale for specific ranges and include error bars.

}

\newcommand{\explaindiskspaceplot}{

    Subfigure (a) shows repository size on a logarithmic scale, while subfigure
    (b) shows the ratio of total repository size to input data size.

}

\newcommand{\muninurl}{\url{http://munin.uit.no}}
\newcommand{\dmvurl}{\url{http://dmv.sleepymurph.com/}}

% towrite / written
% Write points that need to be made in a "towrite", then change to "written"
% when its done.
% Add `final` to the document to disable all.
\newcommand\towrite[1]{
    \todo[inline,color=white,bordercolor=white]{\textcolor{blue}{Write: #1}}
}
\newcommand\written[1]{%
    \ifoptionfinal{}{
        ~\\
        \textcolor{gray}{Write: #1}
        ~\\
    }
}


% An enumerated list with less spacing between items
\newenvironment{tight_enumerate}{
\begin{enumerate}
  \setlength{\itemsep}{0pt}
  \setlength{\parskip}{0pt}
}{\end{enumerate}}


\title{Exploring the Scalability of Distributed Version Control Systems}
\author{Michael J. Murphy \and Otto J. Anshus \and John Markus Bjørndalen}
\date{November 2017}

\begin{document}
\maketitle

\begin{abstract}
\input{dmv-nik-abstract.txt}
\end{abstract}

\section{Introduction}

\towrite{Distributed version control is an interesting form of distributed system because it takes eventual consistency to the extreme.}

\towrite{Every replica of a repository contains the full history in an append-only data structure, any replica may add new commits, and conflicting updates are reconciled later in a merge operation.}


\written{These systems are popular, but their use is generally limited to the small text files of source code.}

\Glspl{DVCS} are designed primarily to store program source code: plain text files in the range of tens of kilobytes.
Checking in larger binary files such as images, sound, or video affects performance.
Actions that require copying data in and out of the system slow from hundredths of a second to full seconds or minutes.
And since a \gls{DVCS} keeps every version of every file in every \gls{repository}, forever, the disk space needs compound.

This has lead to a conventional wisdom that binary files should never be stored in version control, inspiring blog posts with titles such as
"Don't ever commit binary files to Git! Or what to do if you do"~\cite{dont_ever_commit_binaries_to_version_control},
even as the modern software development practice of continuous delivery was commanding teams to "keep absolutely everything in version control."~\cite[p.33]{continuousdeliverybook}

\towrite{This paper explores the challenges of using version control to store larger binary files, with the goal of building a scalable, highly-available, distributed storage system for media files such as images, audio, and video.}

\section{Distributed Media Versioning}

\towrite{We developed an early prototype of such a system, which we call Distributed Media Versioning (DMV).}

\section{Experiments}

\towrite{We perform experiments with the popular version control systems Git and Mercurial, the Git-based backup tool Bup, and our DMV prototype.}

\towrite{We measured commit times and repository sizes when storing single files of increasing size, and when storing increasing numbers of single-kilobyte files.}

We conducted two major experiments.
In order to measure the effect of file size, we would \gls{commit} a single file of increasing size to each target \gls{VCS}.
And to measure the effect of numbers of files, we would \gls{commit} increasing number of small (\SI{1}{\kibi\byte}) files to each target \gls{VCS}.

For each experiment, the procedure for a single trial was as follows:

\begin{tight_enumerate}
    \item Create an empty \gls{repository} of the target \gls{VCS} in a temporary directory
    \item Generate target data to store, either a single file of the target size, or the target number of \SI{1}{\kibi\byte} files
    \item \Gls{commit} the target data to the \gls{repository}, measuring wall-clock time to \gls{commit}
    \item Verify that the first \gls{commit} exists in the \gls{repository}, and if there was any kind of error, run the \gls{repository}'s integrity check operation
    \item Measure the total \gls{repository} size
    \item Overwrite a fraction of each target file
    \item (Number-of-files experiment only) Run the \gls{VCS}'s status command that lists what files have changed, and measure the wall-clock time that it takes to complete
    \item \Gls{commit} again, measuring wall-clock time to \gls{commit}
    \item Verify that the second \gls{commit} exists in the \gls{repository}, and if there was any kind of error, run the \gls{repository}'s integrity check operation
    \item Measure the total \gls{repository} size again
    \item (File-size experiment only) Run Git's garbage collector (\lstinline{git fsck}) to pack objects, then measure total \gls{repository} size again
    \item Delete temporary directory and all trial files
\end{tight_enumerate}

We increased file sizes exponentially by powers of two from \SI{1}{\byte} up to \SI{128}{\gibi\byte}, adding an additional step at \num{1.5} times the base size at each order of magnitude.
For example, starting at \SI{1}{\mebi\byte}, we would run trails with \SI{1}{\mebi\byte}, \SI{1.5}{\mebi\byte}, \SI{2}{\mebi\byte}, \SI{3}{\mebi\byte}, \SI{4}{\mebi\byte}, \SI{6}{\mebi\byte}, \SI{8}{\mebi\byte}, \SI{12}{\mebi\byte}, and so on.

We increased numbers of files exponentially by powers of ten from one file to ten million files, adding additional steps at \num{2.5}, \num{5}, and \num{7.5} times the base number at each order of magnitude.
For example, starting at \num{100} files we would run trials with \num{100}, \num{250}, \num{500}, \num{750}, \num{1000}, \num{2500}, \num{5000}, \num{7500}, \num{10000}, and so on.

Input data files consisted of pseudorandom bytes taken from the operating system's pseudorandom number generator (\lstinline{/dev/urandom} on Linux).

When updating data files for the second \gls{commit}, we would overwrite a single contiguous section of each file with new pseudorandom bytes.
We would start one-quarter of the way into the file, and overwrite \num{1/1024}th of the file's size (or 1 byte if the file was smaller than \SI{1024}{\kibi\byte}).
So a \SI{1}{\mebi\byte} file would have \SI{1}{\kibi\byte} overwritten, a \SI{1}{\gibi\byte} file would have \SI{1}{\mebi\byte} overwritten, and so on.

%

\section{Results}

\towrite{We find that processing files whole will limit maximum file size to what can fit in RAM.}

\towrite{And we find that storing millions of objects loose as files with hash-based names will result in inefficient write speeds and use of disk space.}

\section{Conclusion}

\towrite{We conclude that the key to storing large files is to break them into smaller chunks, and that the key to storing many small chunks is to aggregate them into larger files.}

\towrite{We intend to incorporate these insights into future versions of DMV.}


\printbibliography[]

\listoftodos

\ifoptionfinal{}{
    \section*{Scratch Pad}

This section contains text that I have written, but decided to set aside for now.
It should disappear if the \lstinline{final} option is applied to the document.

\paragraph{CRDTS}
There are also data types are cleverly designed to be commutative, so that the resulting data will be the same regardless of the order in which updates are applied~\cite{crdt_orig}.

}

\end{document}
