\documentclass[usenglish]{nik}

\usepackage[utf8]{inputenc}
\usepackage{mathptm}                % Times New Roman

% towrite / written
% Write points that need to be made in a "towrite", then change to "written"
% when its done.
% Add `final` to the document to disable all.
\newcommand\towrite[1]{
    \todo[inline,color=white,bordercolor=white]{\textcolor{blue}{Write: #1}}
}
\newcommand\written[1]{%
    \ifoptionfinal{}{
        {
            \setlength{\parindent}{0em}
            \par
            \vspace{.5em}
            \textcolor{gray}{Written: #1}
            \vspace{.5em}
            \par
        }
    }
}

% Feedback from specific people
\newcommand{\perotto}[1]{\todo[color=blue!40]{Otto: #1}}
\newcommand{\perottoinline}[1]{\todo[color=blue!40,inline]{Otto: #1}}
\newcommand{\askotto}[1]{\todo[color=violet!40]{Ask Otto: #1}}
\newcommand{\askottoinline}[1]{\todo[color=violet!40,inline]{Ask Otto: #1}}


% An enumerated list with less spacing between items
\newenvironment{tight_enumerate}{
\begin{enumerate}
  \setlength{\itemsep}{0pt}
  \setlength{\parskip}{0pt}
}{\end{enumerate}}



% Repeated text snippets

\newcommand{\explainlogsubfig}{

    Subfigure (a) shows the full range on a logarithmic scale, while the others
    are linear-scale for specific ranges and include error bars.

}

\newcommand{\explaindiskspaceplot}{

    Subfigure (a) shows repository size on a logarithmic scale, while subfigure
    (b) shows the ratio of total repository size to input data size.

}

\newcommand{\muninurl}{\url{http://munin.uit.no}}
\newcommand{\dmvurl}{\url{http://dmv.sleepymurph.com/}}


\title{DMV: Distributed Media Versioning across devices}
\author{Michael J. Murphy \and Otto J. Anshus \and John Markus Bjørndalen}
\date{November 2017}

\begin{document}
\maketitle

\begin{abstract}
\input{dmv-nik-abstract.txt}
\end{abstract}

\section*{OUTLINE}

\begin{verbatim}

- Problem: many devices, more data, difficult to follow what is where

- Cloud not solution. Relies on third party. Connection, privacy, etc.

- DVCS: An alternate approach

    - Extreme availability
    - Version history gives chain of causality for later reconciliation

- Interesting properties of DAG

    - DAG gives de-duplication
    - Content addressing gives tampering/bitrot protection
    - DAG also gives convenient ways to shard data

- Problem 1: dealing with larger files: chunking

    - Bup has chunking but locked into backup workflow

- Problem 2: dealing with many files: packing

    - Git has packing but in separate step that fails for large files

- Problem 3: increasing data: sharding

- In-between: de-duplication

- DMV prototype

- Experiments

    - File size and number of files
    - Random writes

- Results

- Conclusion

    - chunk, content-address, re-pack
    - DMV not yet viable, but it's a start

\end{verbatim}

\section{Introduction}

The \gls{CAP-theorem}~\cite{cap_origin} states that a distributed system cannot be completely consistent (C), available (A), and tolerant of network partitions (P) all at the same time.
When communication between nodes breaks down and they cannot all acknowledge an operation, the system is faced with "the \gls{partitiondecision}: block the operation and thus decrease availability, or proceed and thus risk inconsistency."~\cite{cap_years_later}

Much research is focused on consistency.
However distributed version control systems focus on availability.

A \gls{DVCS} is a small-scale distributed system, where nodes are autonomous.
Rather than a set of nodes that is connected by default, a \gls{DVCS}'s \glspl{repository} are self-contained and offline by default.
A DVCS allows writes to local data at any time, and only connects to other \glspl{repository} intermittently to exchange updates at the user's command.
Concurrent updates are not only allowed but embraced as different \glspl{branch} of development.
A \gls{DVCS} can track many different \glspl{branch} at the same time, and conflicting \glspl{branch} can be combined and resolved by the user in a \gls{merge} operation.

Version control systems are historically designed primarily to store program source code~\cite{history_of_version_control}, plain text files in the range of tens of kilobytes.
Checking in larger binary files such as images, sound, or video affects performance.
Actions that require copying data in and out of the system slow from hundredths of a second to full seconds or minutes.
And since a \gls{DVCS} keeps every version of every file in every \gls{repository} forever, the disk space needs only increase.

This has lead to a conventional wisdom that binary files should never be stored in version control, inspiring blog posts with titles such as
"Don't ever commit binary files to Git! Or what to do if you do"~\cite{dont_ever_commit_binaries_to_version_control},
even as the modern software development practice of continuous delivery was commanding teams to "keep absolutely everything in version control."~\cite[p.33]{continuousdeliverybook}

This paper evaluates the behavior of current version control systems when storing larger binary files, with the goal of building a scalable, highly-available, distributed storage system with versioning for media files such as images, audio, and video.

\section{Architecture}

\begin{figure}[]
    \centering
    \includegraphics[width=0.95\textwidth]{dia_architecture}
    \caption{Repositories in an ad-hoc network}
    \label{fig:dia_architecture}
\end{figure}

\towrite{Explain ad-hoc architecture}

\section{Design}

\begin{figure}[]
    \centering
    \includegraphics[width=0.9\textwidth]{dia_dmv_dag_example_three_commits}
    \caption{A simple DMV DAG with three commits}
    \label{dia_dmv_dag_example_three_commits}
\end{figure}

\section{Implementation}

\towrite{Talk about prototype implementation}

\section{Experiments}

\written{We perform experiments with the popular version control systems Git and Mercurial, the Git-based backup tool Bup, and our DMV prototype.}

\written{We measured commit times and repository sizes when storing single files of increasing size, and when storing increasing numbers of single-kilobyte files.}

\subsection{Methodology}

We conducted two major experiments.
In order to measure the effect of file size, we committed a single file of increasing size to a each target \gls{VCS}.
And to measure the effect of numbers of files, we committed increasing number of small (\SI{1}{\kibi\byte}) files to each target \gls{VCS}.

We ran each experiment with four different \glspl{VCS}: Git, Mercurial, Bup, the DMV prototype.
We chose Git because it is the most popular \gls{DVCS} in use today~\cite{what_are_devs_talking_about}.
We chose the Mercurial and Bup because they are both related to Git but each store data differently.
Git and DMV both store objects in an \gls{objectstore} directory as a file named for its hash ID.
Git has a separate garbage collection step that takes object files and aggregates them into \glspl{packfile}~\cite[Section 10.7]{git_book}.
Mercurial stores revisions of each file as a base revision followed by a series of deltas~\cite[Chapter 4]{hgbook}, much like previous \glspl{VCS} such as RCS, CVS, and Subversion~\cite{history_of_version_control}.
Bup uses Git's exact data model and \gls{packfile} format, but Bup breaks files into chunks using a \gls{rollinghash}, reusing Git's \gls{tree} object as a \gls{chunkedblob} index\footnotemark.
Unlike Git, Bup writes to the \gls{packfile} format directly, without Git's separate commit and pack steps, and without bothering to calculate deltas~\cite{bup_design}.
As a control, we also ran the experiments with a dummy \gls{VCS} that simply copied the files to a hidden directory.

\footnotetext{Git can read a repository written by Bup, but it will see
the large file as a directory full of smaller chunk files.}

For each experiment, the procedure for a single trial was as follows:
\begin{tight_enumerate}
    \item Create an empty \gls{repository} of the target \gls{VCS} in a temporary directory
    \item Generate target data to store, either a single file of the target size, or the target number of \SI{1}{\kibi\byte} files
    \item \Gls{commit} the target data to the \gls{repository}, measuring wall-clock time to \gls{commit}
    \item Verify that the first \gls{commit} exists in the \gls{repository}, and if there was any kind of error, run the \gls{repository}'s integrity check operation
    \item Measure the total \gls{repository} size
    \item Overwrite a fraction (\num{1/1024}) of each target file
    \item (Number-of-files experiment only) Run the \gls{VCS}'s status command that lists what files have changed, and measure the wall-clock time that it takes to complete
    \item \Gls{commit} again, measuring wall-clock time to \gls{commit}
    \item Verify that the second \gls{commit} exists in the \gls{repository}, and if there was any kind of error, run the \gls{repository}'s integrity check operation
    \item Measure the total \gls{repository} size again
    \item (File-size experiment only) Run Git's garbage collector (\lstinline{git fsck}) to pack objects, then measure total \gls{repository} size again
    \item Delete temporary directory and all trial files
\end{tight_enumerate}

We increased file sizes exponentially by powers of two from \SI{1}{\byte} up to \SI{128}{\gibi\byte}, adding an additional step at \num{1.5} times the base size at each order of magnitude.
For example, on the megabyte scale, the file sizes are \SI{1}{\mebi\byte}, \SI{1.5}{\mebi\byte}, \SI{2}{\mebi\byte}, \SI{3}{\mebi\byte}, \SI{4}{\mebi\byte}, \SI{6}{\mebi\byte}, \SI{8}{\mebi\byte}, \SI{12}{\mebi\byte}, and so on.

We increased numbers of files exponentially by powers of ten from one file to ten million files, adding additional steps at \num{2.5}, \num{5}, and \num{7.5} times the base number at each order of magnitude.
For example, at the hundreds and thousands scales, the file quantities are \num{100}, \num{250}, \num{500}, \num{750}, \num{1000}, \num{2500}, \num{5000}, \num{7500}, \num{10000}, and so on.

Input data files consisted of pseudorandom bytes taken from the operating system's pseudorandom number generator (\lstinline{/dev/urandom} on Linux).

%

\subsection{Experiment Platform}

We ran the trials on four dedicated computers with no other load.
Each was a typical office desktop with a \SI{3.16}{\giga\hertz} \num{64}-bit dual-core processor and \SI{8}{\gibi\byte} of RAM, running Debian version 8.6 ("Jessie").
Each computer had one normal SATA hard disk (spinning platter, not solid-state), and trials were conducted on a dedicated \SI{197}{\gibi\byte} LVM partition formatted with the ext4 filesystem.
All came from the same manufacturer with the same specifications and were, for practical purposes, identical.
%Additional details can be found in \autoref{test-machine-specs}.
\todo{Include platform table?}

We ran every trial four times, once on each of the experiment computers, and took the mean and standard deviation of each time and disk space measurement.
However, because the experiment computers are practically identical, there was little variation.

%

\section{Results}

\written{We find that processing files whole will limit maximum file size to what can fit in RAM.}

In our experiments, both Git and Mercurial had file size limits that were related to RAM.
Mercurial would refuse to commit a file \SI{2}{\gib} or larger.
It would exit with an error code and print an error message saying "up to 6442 MB of RAM may be required to manage this file."
The commit would not be stored, and the repository would be left unchanged.
This suggests that Mercurial needs to be able to fit the file into memory three times over in order to commit it.

Git's commit operation would appear to fail with files \SI{12}{\gib} and larger.
It would exit with an error code and print an error message saying "fatal: Out of memory, malloc failed (tried to allocate 12884901889 bytes)."
However, the commit would be written to the repository, and git's \lstinline{fsck} operation would report no errors.
So the commit operation completes successfully, even though an error is reported.

With files \SI{24}{\gib} and larger, Git's \lstinline{fsck} operation itself would fail.
The \lstinline{fsck} command would exit with an error code and give a similar "fatal ... malloc" error.
However, the file could still be checked out from the repository without error.
So we continued the trials assuming that these were also false alarms.

\towrite{Give file size result timing}

\towrite{And we find that storing millions of objects loose as files with hash-based names will result in inefficient write speeds and use of disk space.}

\section{Conclusion}

We have performed experiments to probe the scalability limits of existing \gls{DAG}-based \acrlongpl{DVCS}.
We have shown that the maximum size of file that Git and Mercurial can store is limited by the amount of available memory in the system.
We conclude that this is because those systems calculate deltas of files to de-duplicate data, and they load the entire file into memory in order to do so.
\perjmbinline{memory limitations is probably not a common issue for systems intended for small files, otherwise they would probably have used algorithms that don't require the entire file to be loaded into memory.}
\perjmbinline{The point is: we should be careful when sounding like we say that file-size diffs is causing memory issues as there are two ways to handle this:\\
- better algorithms\\
- larger swap (slow)}

We have also rediscovered the limits of the Unix filesystem for storing many small files.
We saw that writing files smaller than the filesystem block size incurs storage overhead, that splitting files among too many subdirectories takes \glspl{inode} that are needed to store files, and that jumping between directories when writing files incurs write-speed penalties.

We have shown that any \gls{VCS} that stores objects as individual files on the filesystem will encounter these filesystem limitations as they try to scale in terms of number of files.
\perjmb{it could be interesting to see how other file systems behave with the same experiments.}
A \gls{VCS} that also breaks files into chunks will turn the problem of storing large files into the problem of storing many files, again encountering these limitations.
However, the limitations can be avoided by aggregating objects into \glspl{packfile} as Bup does.


\printbibliography[]

\listoftodos

\end{document}
