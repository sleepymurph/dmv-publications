\documentclass[usenglish]{nik}

% File input and fonts
\usepackage[utf8]{inputenc}
\usepackage{mathptm}                % Times New Roman (text and math)

% Let us check for draft/final conditions
\usepackage{ifdraft}

% Figures
\usepackage{graphicx}
\usepackage[margin=1cm,justification=centering]{caption}
\usepackage{subcaption} % sub-figures

\ifoptionfinal{}{
\usepackage{endfloat}   % Move figures out of text for now
}


% Listings
\usepackage{listings}
\lstset{basicstyle=\ttfamily\footnotesize,breaklines=false}


% Numbers and units of measurement
\usepackage[binary-units]{siunitx}

\newcommand{\gib}{\gibi\byte}
\newcommand{\mib}{\mebi\byte}
\newcommand{\kib}{\kibi\byte}


%%% Typographical tweaks

\usepackage{booktabs}               % Nicer hlines (\midrule) in tables
\usepackage[inline]{enumitem}       % Compact inline lists with enumerate*

% An enumerated list with less spacing between items
\newenvironment{tight_enumerate}{
\begin{enumerate}
  \setlength{\itemsep}{0pt}
  \setlength{\parskip}{0pt}
}{\end{enumerate}}



%%% Bibliography

\usepackage[
  backend=bibtex,
  urldate=long, % use date like "Apr. 7, 2017" instead of American "04/07/2017"
  firstinits=true,    % use only initials of given names
]{biblatex}

\addbibresource{dmv-nik.bib}
% Bib names last-first
\DeclareNameAlias{sortname}{last-first}
\DeclareNameAlias{default}{last-first}
% Bib names small caps
\renewcommand{\mkbibnamelast}[1]{\textsc{#1}}



%%% TODO notes

\usepackage[obeyFinal]{todonotes}
\ifoptionfinal{}{
    \setlength{\marginparwidth}{2.5cm}
}

% towrite / written
% Write points that need to be made in a "towrite", then change to "written"
% when its done.
% Add `final` to the document to disable all.
\newcommand\towrite[1]{
    \todo[inline,color=white,bordercolor=white]{\textcolor{blue}{Write: #1}}
}
\newcommand\written[1]{%
    \ifoptionfinal{}{
        {
            \setlength{\parindent}{0em}
            \par
            \vspace{.5em}
            \textcolor{gray}{Written: #1}
            \vspace{.5em}
            \par
        }
    }
}

% Feedback from specific people
\newcommand{\perotto}[1]{\todo[color=blue!40]{Otto: #1}}
\newcommand{\perottoinline}[1]{\todo[color=blue!40,inline]{Otto: #1}}
\newcommand{\askotto}[1]{\todo[color=violet!40]{Ask Otto: #1}}
\newcommand{\askottoinline}[1]{\todo[color=violet!40,inline]{Ask Otto: #1}}



%%% ↑↑↑ Hyperref unaware packages above this line ↑↑↑ %%%%

% The hyperref docs recommend declaring it after the other \usepackage
% declarations, because it has to redefine several commands to work properly,
% and other later redefinitions might interfere.
%
% However, other packages are aware of hyperref and ask to be declared AFTER it.

% Link URLs in the PDF, and link references within the PDF itself
\usepackage[
  hidelinks,    % Do not style links. I think this is classier.
  pdfusetitle,  % Use doc title metadata for PDF title metadata
  pdfdisplaydoctitle,           % Display document title instead of filename in title bar
  bookmarksnumbered,            % use section numbers in PDF index
  pdfpagemode={UseOutlines},    % Show bookmarks
  pdfstartview={FitV},          % Fit height of page to PDF viewer window
]{hyperref}

%%% ↓↓↓ Hyperref aware packages below this line ↓↓↓ %%%%



% Combined references (chapters 1, 2 and 3)
\usepackage{cleveref}

% Glossaries
\usepackage[toc]{glossaries}
\makeglossaries
% Glossary entries
% Style: Capitalize first letter of all descriptions,
%   but don't capitalize all initialze in acronym unless it's a proper name
%   or if it's not obvious where the acronym is from.


% Dist sys

\newacronym[
    description={Atomicity, consistency, durability, and isolation, the
    guarantees of a traditional database commit},
]{ACID}{ACID}{atomicity, consistency, durability, and isolation}

\newglossaryentry{CAP-theorem}{
    name={CAP-theorem},
    description={The fundamental theorem of distributed systems, that no system
    can simultaneously be consistent (C), available (A), and tolerant of network
    partitions (P)},
    see={partitiondecision},
}

\newglossaryentry{partitiondecision}{
    name={partition decision},
    description={The dilemma faced by a distributed system during a network
    partition: to decrease availability or risk inconsistency},
    see={CAP-theorem},
}

\newglossaryentry{endtoendargument}{
    name={end-to-end argument},
    description={When designing a communications system, the idea that there is
    certain functionality that can only be implemented correctly by the
    higher-level application at the endpoints of the communication, and so it is
    futile for the communication system to try to provide that functionality
    itself},
}

% General Version Control

\newacronym[
    description={Distributed version control system, such as Git, where
    individual repositories can operate independently without having to connect
    to a central repository},
    see={VCS},
]
{DVCS}{DVCS}{distributed version control system}

\newacronym[
    description={Version control system, a program that stores many versions of
    a file or set of files, commonly used to track changes to source code},
    see={SCM},
]
{VCS}{VCS}{version control system}

\newacronym[
    description={Source code manager, a version control system that is designed
    primarily to store source code},
    see={VCS},
]
{SCM}{SCM}{source code manager}


% Specific systems

\newacronym[
    description={Distributed Media Versioning, the new distributed data storage
    platform described and introduced in this dissertation}
]
{DMV}{DMV}{Distributed Media Versioning}


% VCS architecture

\newglossaryentry{repository}{
    name={repository},
    plural={repositories},
    description={A location where data is stored in a version control system.
    Early systems would have a central repository that clients would check out
    from. In distributed version control, every client is a separate
    repository},
}

\newglossaryentry{objectstore}{
    name={object store},
    description={Content-addressable storage for DAG objects},
    see={DAG},
}

\newglossaryentry{workdir}{
    name={working directory},
    description={A directory where files that are tracked by a version control
    system are actively worked on and edited},
}

\newglossaryentry{branch}{
    name={branch},
    plural={branches},
    description={In a version control system, separate concurrent lines of
    update history},
    see={head,merge},
}

\newglossaryentry{head}{
    name={head},
    description={In a version control system, the most recent revision in a
    branch},
    see={branch},
}

\newglossaryentry{merge}{
    name={merge},
    description={In a version control system, an operation that combines two
    branches and reconciles conflicting changes},
    see={branch},
}


% DAG

\newglossaryentry{contentaddressablestorage}{
    name={content addressable storage},
    description={Storage that stores immutable objects named by a hash of their
    content, which naturally de-duplicates identical objects},
}

\newacronym[
    description={Directed acyclic graph, the type of graph data structure used
    to represent history in many distributed version control systems. Directed
    meaning all the edges have a direction, from one node to another, and
    acyclic meaning that there are no cycles, no paths that revisit any node},
    see={blob,chunkedblob,tree,commit,ref},
]
{DAG}{DAG}{directed acyclic graph}

\newacronym[
    description={Binary large object, a sequence of unstructured binary data. In
    Git and DMV, a DAG object holding file data},
    first={blob (binary large object)},
    see={DAG},
]
{blob}{blob}{binary large object}

\newglossaryentry{chunkedblob}{
    name={chunked blob},
    description={In DMV, a DAG object that is an index of blobs that make up a
    larger blob},
    see={DAG,blob},
}

\newglossaryentry{tree}{
    name={tree},
    description={In Git and DMV, a DAG object representing a particular state of
    a file hierarchy},
    see={DAG},
}

\newglossaryentry{commit}{
    name={commit},
    description={In version control, the operation for storing a particular
    version of the data. Also, the resulting DAG object that represents that
    version in the history},
    see={DAG},
}

\newacronym[
    description={A reference to a commit object in the DAG},
    see={DAG},
]
{ref}{ref}{reference}

\newglossaryentry{packfile}{
    name={pack file},
    description={An object store file format that aggregates many objects in one
    file},
    see={objectstore},
}

\newglossaryentry{filelog}{
    name={filelog},
    description={Mercurial's file format that stores different versions of the
    same file as a base version followed by a series of delta},
}


% Rolling Hash

\newglossaryentry{rollinghash}{
    name={rolling hash},
    first={rolling hash algorithm},
    description={A hash checksum that operates over a moving window of data in a
    byte stream that can be used to find repeating patterns},
    see={windowsize,divisor},
}

\newglossaryentry{windowsize}{
    name={window size},
    symbol={\ensuremath{w}},
    description={In a rolling hash algorithm, the number of previous bytes used
    in the rolling sum},
    see={rollinghash,divisor},
}

\newglossaryentry{divisor}{
    name={divisor},
    symbol={\ensuremath{d}},
    description={In a rolling hash algorithm, the divisor in the modulus
    operation. A chunk boundary is created when the sum of the bytes in the
    window, modulo this divisor, is equal to zero},
    see={rollinghash,windowsize},
}


% low-level

\newglossaryentry{inode}{
    name={inode},
    description={A data structure in a Unix filesystem that stores file
    metadata. Each filesystem has a fixed number of inodes, which limits the
    total number of files and directories the filesystem can hold},
}

\newacronym[
    description={Application binary interface, the public interface between a
    system library and a client application}
]{ABI}{ABI}{application binary interface}

\newacronym[
    description={random number generator}
]{RNG}{RNG}{random number generator}

% Emphasize first use of glossary term
\defglsentryfmt[main]{\ifglsused{\glslabel}{\glsgenentryfmt}{\emph{\glsgenentryfmt}}}
% Disable glossary hyperlinks because we will not print a glossary in this paper
\glsdisablehyper



% Repeated text snippets

\newcommand{\explainlogsubfig}{

    Subfigure (a) shows the full range on a logarithmic scale, while the others
    are linear-scale for specific ranges and include error bars.

}

\newcommand{\explaindiskspaceplot}{

    Subfigure (a) shows repository size on a logarithmic scale, while subfigure
    (b) shows the ratio of total repository size to input data size.

}

\newcommand{\muninurl}{\url{http://munin.uit.no}}
\newcommand{\dmvurl}{\url{http://dmv.sleepymurph.com/}}

% towrite / written
% Write points that need to be made in a "towrite", then change to "written"
% when its done.
% Add `final` to the document to disable all.
\newcommand\towrite[1]{
    \todo[inline,color=white,bordercolor=white]{\textcolor{blue}{Write: #1}}
}
\newcommand\written[1]{%
    \ifoptionfinal{}{
        ~\\
        \textcolor{gray}{Write: #1}
        ~\\
    }
}


% An enumerated list with less spacing between items
\newenvironment{tight_enumerate}{
\begin{enumerate}
  \setlength{\itemsep}{0pt}
  \setlength{\parskip}{0pt}
}{\end{enumerate}}


\title{DMV: Distributed Media Versioning across devices}
\author{Michael J. Murphy \and Otto J. Anshus \and John Markus Bjørndalen}
\date{November 2017}

\begin{document}
\maketitle

\begin{abstract}
\input{dmv-nik-abstract.txt}
\end{abstract}

\section*{OUTLINE}

\begin{verbatim}

- Problem: many devices, more data, difficult to follow what is where

- Cloud not solution. Relies on third party. Connection, privacy, etc.

- DVCS: An alternate approach

    - Extreme availability
    - Version history gives chain of causality for later reconciliation

- Interesting properties of DAG

    - DAG gives de-duplication
    - Content addressing gives tampering/bitrot protection
    - DAG also gives convenient ways to shard data

- Problem 1: dealing with larger files: chunking

    - Bup has chunking but locked into backup workflow

- Problem 2: dealing with many files: packing

    - Git has packing but in separate step that fails for large files

- Problem 3: increasing data: sharding

- In-between: de-duplication

- DMV prototype

- Experiments

    - File size and number of files
    - Random writes

- Results

- Conclusion

    - chunk, content-address, re-pack
    - DMV not yet viable, but it's a start

\end{verbatim}

\section{Introduction}

\towrite{A typical computer user has multiple devices holding an increasing amount of data.}
\towrite{Most users will have at least a computer and a mobile phone.}
\towrite{Many will also have a work computer, tablet, or other devices.}
\towrite{These devices have varying resources, including processing, memory, and storage.}
\towrite{They may also be in different locations, on different networks, or turned off at any time.}
\towrite{The user's data will be in files of varying sizes and media types, from kilobyte text documents to multi-gigabyte videos and beyond.}
\towrite{The volume of data is also always increasing as data is authored, collected from the internet, or gathered from mobile sensors.}
\towrite{This data is strewn across these devices in an ad-hoc fashion, according to where it is produced and consumed.}
\towrite{When the user needs a particular file, they must either remember where it is or perform a frustrating, manual, multi-device search.}
\towrite{Also, copies of data on different devices will diverge if updates are made separately and not reconciled.}

\subsection{Shortcomings of Cloud-Based Solutions}

\towrite{Cloud computing eases these problems by centralizing storage, searching, and update reconciliation.}
\towrite{However, the user's access to their data depends on the reliability of their network connection and the reliability and longevity of the cloud service.}
\towrite{Handing data over to a third party also raises concerns about privacy.}
\towrite{The cloud service may also charge a recurring subscription fee.}
\towrite{The user might prefer to use the devices they already own, provided there is an easier way to manage the data.}

\subsection{Potential of Version Control}

\towrite{This paper explores distributed version control systems as an alternative approach to managing data across a spectrum of devices.}

\towrite{A DVCS keeps writable copies of a data set at multiple locations, tracks update history, and allows diverging versions to be merged at a later date.}

Git stores its data in a directed acyclic graph (DAG) structure.
Blob objects contain file data;
tree objects store lists of blobs, representing directories;
and a commit objects each associate a particular tree state with metadata such as time, author, and previous commit state, placing that tree state into a history.
Each object is stored in an content-addressed object database, indexed by a cryptographic hash of its contents \cite{git_initial_readme}.
%Mercurial's data is stored differently on disk, but it can still be modelled conceptually with this same data structure \cite[Chapter 4]{hgbook}.

Objects, once stored, are immutable.
Updating an object would change its hash and thus its ID, creating a new object.
Because objects refer to other objects by hash ID, a new object can only refer to a pre-existing object with known content.
The graph is directed because these links flow in one direction, and it is acyclic because links cannot be created to objects that do not exist yet, and existing objects cannot be updated to point to newer objects.
The DAG is append-only.

Such a DAG structure has several interesting properties for data storage.
\begin{description}
    \item[De-duplication]
        Identical objects are de-duplicated because they will have the same ID and naturally collapse into a single object in the data store.
        This results in a natural compression of redundant objects.
        The efficiency of the compression depends on how well identical pieces of data map to independent objects.
        In Git, the redundant objects are the files and directories that do not change between commits.
    \item[A record of causality]
        Copies of the DAG can be distributed and updated independently.
        Concurrent updates will result in multiple branches of history, but references from child commit to parent commit establish a happens-before relationship and give a chain of causality.
        Branches can be merged by manually reconciling the changes, and then creating a merge commit that refers to both parent commits.
        When transferring updates from one copy to another, only new objects need to be transferred.
    \item[Atomic updates]
        When a new commit is added, all objects are added the database first, then finally the reference to the current commit is updated.
        This reference is a 160-byte SHA-1 hash value, and which can be updated atomically.
    \item[Verifiability]
        Because every object is identified by its cryptographic hash, the data integrity of each object can be verified at any time by re-computing and checking its hash.
\end{description}

\towrite{However, version control systems are designed for the small text files of source code and are not suited to larger binary files.}


\section{DMV Architecture and Design}

\towrite{We describe the architecture, design, and implementation of a new system we call Distributed Media Versioning (DMV) that resembles version control but is more flexible.}
\towrite{DMV will allow the user to shard and replicate data across many devices with fine-grained control.}
\towrite{It will keep a unified view of the data set as subsets of the data are copied or moved between devices by user request.}
\towrite{It will allow data to be updated on any device, and it will track history so that}
\towrite{diverging versions can be merged later.}

\begin{figure}[]
    \centering
    \includegraphics[width=0.95\textwidth]{dia_architecture}
    \caption{Repositories in an ad-hoc network}
    \label{fig:dia_architecture}
\end{figure}

\begin{figure}[]
    \centering
    \includegraphics[width=0.9\textwidth]{dia_dmv_dag_example_three_commits}
    \caption{A simple DMV DAG with three commits}
    \label{dia_dmv_dag_example_three_commits}
\end{figure}


\newcommand{\slicediagramwidth}{0.45\textwidth}

\begin{figure}[]

    \centering

    \begin{subfigure}[]{\slicediagramwidth}
        \includegraphics[width=\textwidth]{dia_dmv_dag_slice_partial_history}
        \caption{Partial history of full data set}
        \label{dia_dmv_dag_slice_partial_history}
    \end{subfigure}
    ~
    \begin{subfigure}[]{\slicediagramwidth}
        \includegraphics[width=\textwidth]{dia_dmv_dag_slice_history_of_subset}
        \caption{Full history of part of data set}
        \label{dia_dmv_dag_slice_history_of_subset}
    \end{subfigure}
    ~
    \begin{subfigure}[]{\slicediagramwidth}
        \includegraphics[width=\textwidth]{dia_dmv_dag_slice_history_of_metadata}
        \caption{Full history of metadata}
        \label{dia_dmv_dag_slice_history_of_metadata}
    \end{subfigure}

    \caption{A DMV DAG, sliced in different dimensions}
\end{figure}



\section{Evaluation}

\towrite{We perform experiments to explore the scalability limits of selected version control systems.}
\towrite{We find that the maximum file size is limited by available RAM, and that commit times increase sharply as the number of files increases into the}
\towrite{millions.}
\towrite{We also perform the same experiments against a DMV prototype for comparison.}
\towrite{DMV avoids the file-size limitations by using a rolling hash algorithm to break larger files into smaller chunks.}
\towrite{Unfortunately, our early DMV prototype suffers the same problems with numerous files because it uses the underlying filesystem in a similar way.}

\section{Conclusion}

\towrite{We conclude that the key to processing large files is to break them into many smaller chunks, and the key to storing many small files is to aggregate them into larger packs.}
\towrite{We propose corrective changes for future work on DMV.}


\printbibliography[]

\listoftodos

\end{document}
