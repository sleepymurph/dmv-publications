% File input and fonts
\usepackage[utf8]{inputenc}
\usepackage{mathptm}                % Times New Roman (text and math)

% Let us check for draft/final conditions
\usepackage{ifdraft}

% Figures
\usepackage{graphicx}
\usepackage[margin=1cm,justification=centering]{caption}
\usepackage{subcaption} % sub-figures

\ifoptionfinal{}{
%\usepackage{endfloat}   % Move figures out of text for now
}


% Listings
\usepackage{listings}
\lstset{basicstyle=\ttfamily\footnotesize,breaklines=false}


% Numbers and units of measurement
\usepackage[binary-units]{siunitx}

\newcommand{\gib}{\gibi\byte}
\newcommand{\mib}{\mebi\byte}
\newcommand{\kib}{\kibi\byte}


%%% Typographical tweaks

\usepackage{booktabs}               % Nicer hlines (\midrule) in tables
\usepackage[inline]{enumitem}       % Compact inline lists with enumerate*

% An enumerated list with less spacing between items
\newenvironment{tight_enumerate}{
\begin{enumerate}
  \setlength{\itemsep}{0pt}
  \setlength{\parskip}{0pt}
}{\end{enumerate}}



%%% Bibliography

\usepackage[
  backend=bibtex,
  urldate=long, % use date like "Apr. 7, 2017" instead of American "04/07/2017"
  firstinits=true,    % use only initials of given names
]{biblatex}

\addbibresource{dmv-nik.bib}
% Bib names last-first
\DeclareNameAlias{sortname}{last-first}
\DeclareNameAlias{default}{last-first}
% Bib names small caps
\renewcommand{\mkbibnamelast}[1]{\textsc{#1}}



%%% TODO notes

\usepackage[obeyFinal]{todonotes}
\ifoptionfinal{}{
    \setlength{\marginparwidth}{2.5cm}
}

% towrite / written
% Write points that need to be made in a "towrite", then change to "written"
% when its done.
% Add `final` to the document to disable all.
\newcommand\towrite[1]{
    \todo[inline,color=white,bordercolor=white]{\textcolor{blue}{Write: #1}}
}
\newcommand\written[1]{%
    \ifoptionfinal{}{
        {
            \setlength{\parindent}{0em}
            \par
            \vspace{.5em}
            \textcolor{gray}{Written: #1}
            \vspace{.5em}
            \par
        }
    }
}

% Feedback from specific people
\newcommand{\perotto}[1]{\todo[color=blue!40]{Otto: #1}}
\newcommand{\perottoinline}[1]{\todo[color=blue!40,inline]{Otto: #1}}
\newcommand{\askotto}[1]{\todo[color=violet!40]{Ask Otto: #1}}
\newcommand{\askottoinline}[1]{\todo[color=violet!40,inline]{Ask Otto: #1}}

\newcommand{\perjmb}[1]{\todo[color=blue!20]{JMB: #1}}
\newcommand{\perjmbinline}[1]{\todo[color=blue!20,inline]{JMB: #1}}


%%% ↑↑↑ Hyperref unaware packages above this line ↑↑↑ %%%%

% The hyperref docs recommend declaring it after the other \usepackage
% declarations, because it has to redefine several commands to work properly,
% and other later redefinitions might interfere.
%
% However, other packages are aware of hyperref and ask to be declared AFTER it.

% Link URLs in the PDF, and link references within the PDF itself
\usepackage[
  hidelinks,    % Do not style links. I think this is classier.
  pdfusetitle,  % Use doc title metadata for PDF title metadata
  pdfdisplaydoctitle,           % Display document title instead of filename in title bar
  bookmarksnumbered,            % use section numbers in PDF index
  pdfpagemode={UseOutlines},    % Show bookmarks
  pdfstartview={FitV},          % Fit height of page to PDF viewer window
]{hyperref}

%%% ↓↓↓ Hyperref aware packages below this line ↓↓↓ %%%%



% Combined references (chapters 1, 2 and 3)
\usepackage{cleveref}

% Glossaries
\usepackage[toc]{glossaries}
\makeglossaries
% Glossary entries
\input{dmv-nik-glossary}
% Emphasize first use of glossary term
%\defglsentryfmt[main]{\ifglsused{\glslabel}{\glsgenentryfmt}{\emph{\glsgenentryfmt}}}
% Disable glossary hyperlinks because we will not print a glossary in this paper
\glsdisablehyper



% Repeated text snippets

\newcommand{\explainlogsubfig}{

    Subfigure (a) shows the full range on a logarithmic scale, while the others
    are linear-scale for specific ranges and include error bars.

}

\newcommand{\explaindiskspaceplot}{

    Subfigure (a) shows repository size on a logarithmic scale, while subfigure
    (b) shows the ratio of total repository size to input data size.

}

\newcommand{\muninurl}{\url{http://munin.uit.no}}
\newcommand{\dmvurl}{\url{http://dmv.sleepymurph.com/}}
