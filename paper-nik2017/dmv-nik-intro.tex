\section{Introduction}

The \gls{CAP-theorem}~\cite{cap_origin} states that a distributed system cannot be completely consistent (C), available (A), and tolerant of network partitions (P) all at the same time.
When communication between nodes breaks down and they cannot all acknowledge an operation, the system is faced with "the \gls{partitiondecision}: block the operation and thus decrease availability, or proceed and thus risk inconsistency."~\cite{cap_years_later}

Much research is focused on consistency.
However, distributed version control systems focus on availability.

A \gls{DVCS} is a small-scale distributed system, where nodes are autonomous.
Rather than a set of nodes that is connected by default, a \gls{DVCS}'s \glspl{repository} are self-contained and offline by default.
A DVCS allows writes to local data at any time, and only connects to other \glspl{repository} intermittently to exchange updates at the user's command.
Concurrent updates are not only allowed but embraced as different \glspl{branch} of development.
A \gls{DVCS} can track many different \glspl{branch} at the same time, and conflicting \glspl{branch} can be combined and resolved by the user in a \gls{merge} operation.

Version control systems are historically designed primarily to store program source code~\cite{history_of_version_control}, plain text files in the range of tens of kilobytes.
Checking in larger binary files such as images, sound, or video affects performance.
Actions that require copying data in and out of the system slow from hundredths of a second to full seconds or minutes.
And since a \gls{DVCS} keeps every version of every file in every \gls{repository} forever, the disk space needs only increase.

This has lead to a conventional wisdom that binary files should never be stored in version control, inspiring blog posts with titles such as
"Don't ever commit binary files to Git! Or what to do if you do"~\cite{dont_ever_commit_binaries_to_version_control},
even as the modern software development practice of continuous delivery was commanding teams to "keep absolutely everything in version control."~\cite[p.33]{continuousdeliverybook}

This paper evaluates the behavior of current version control systems when storing larger binary files, with the goal of building a scalable, highly-available, distributed storage system with versioning for media files such as images, audio, and video.
