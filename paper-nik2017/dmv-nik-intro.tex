\section{Introduction}

\written{Distributed version control is an interesting form of distributed system because it takes eventual consistency to the extreme.}

Distributed systems are ruled by the \gls{CAP-theorem}~\cite{cap_origin}, which states that a system cannot be completely consistent (C), available (A), and tolerant of network partitions (P) all at the same time.
When communication between replicas breaks down and they cannot all acknowledge an operation, the system is faced with "the \gls{partitiondecision}: block the operation and thus decrease availability, or proceed and thus risk inconsistency."~\cite{cap_years_later}

Much research is aimed at improving consistency.
Vector clocks~\cite{lamport_ordering} and consensus algorithms such as Paxos~\cite{paxos_made_simple,paxos_made_moderately_complex} make sure the same updates are applied in the same order on all replicas even, if a minority of nodes cannot respond.
There are also data types are cleverly designed to be commutative, so that the resulting data will be the same regardless of the order in which updates are applied~\cite{crdt_orig}.
But in general, when systems cannot communicate, the CAP theorem cannot be avoided~\cite{cap_proof}, and the system is still faced with the \gls{partitiondecision}.

\written{Every replica of a repository contains the full history in an append-only data structure, any replica may add new commits, and conflicting updates are reconciled later in a merge operation.}

Though maybe not designed with the CAP theorem explicitly in mind, a \gls{DVCS} is in fact a small-scale distributed system that takes the availability-first approach to the extreme.
Rather than a set of connected nodes that may occasionally lose contact in a network partition, a \gls{DVCS}'s \glspl{repository} are self-contained and offline by default.
They allow writes to local data at any time, and only connect to other \glspl{repository} intermittently by user command to exchange updates.
Concurrent updates are not only allowed but embraced as different \glspl{branch} of development.
A \gls{DVCS} can track many different \glspl{branch} at the same time, and conflicting \glspl{branch} can be combined and resolved by the user in a \gls{merge} operation.

The \glsdisp{DVCS}{distributed version control} concept may have something to
teach larger-scale systems about availability.

\written{These systems are popular, but their use is generally limited to the small text files of source code.}

\Glspl{DVCS} are designed primarily to store program source code: plain text files in the range of tens of kilobytes.
Checking in larger binary files such as images, sound, or video affects performance.
Actions that require copying data in and out of the system slow from hundredths of a second to full seconds or minutes.
And since a \gls{DVCS} keeps every version of every file in every \gls{repository}, forever, the disk space needs compound.

This has lead to a conventional wisdom that binary files should never be stored in version control, inspiring blog posts with titles such as
"Don't ever commit binary files to Git! Or what to do if you do"~\cite{dont_ever_commit_binaries_to_version_control},
even as the modern software development practice of continuous delivery was commanding teams to "keep absolutely everything in version control."~\cite[p.33]{continuousdeliverybook}

\towrite{This paper explores the challenges of using version control to store larger binary files, with the goal of building a scalable, highly-available, distributed storage system for media files such as images, audio, and video.}
