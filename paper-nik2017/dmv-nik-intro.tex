\section{Introduction}

The \gls{CAP-theorem}~\cite{cap_origin} states that a distributed system cannot be completely consistent (C), available (A), and tolerant of network partitions (P) all at the same time.
When communication between nodes breaks down and they cannot all acknowledge an operation, the system is faced with "the \gls{partitiondecision}: block the operation and thus decrease availability, or proceed and thus risk inconsistency."~\cite{cap_years_later}

Much research is focused on consistency.
However distributed version control systems focus on availability.

Though maybe not designed with the CAP theorem explicitly in mind, a \gls{DVCS} is in fact a small-scale distributed system, where nodes are completely autonomous.
Rather than a set of connected nodes that may occasionally lose contact in a network partition, a \gls{DVCS}'s \glspl{repository} are self-contained and offline by default.
They allow writes to local data at any time, and only connect to other \glspl{repository} intermittently by user command to exchange updates.
Concurrent updates are not only allowed but embraced as different \glspl{branch} of development.
A \gls{DVCS} can track many different \glspl{branch} at the same time, and conflicting \glspl{branch} can be combined and resolved by the user in a \gls{merge} operation.

\Glspl{DVCS} are designed primarily to store program source code: plain text files in the range of tens of kilobytes.
\perjmb{it would be nice if we had a reference for this, but it's not critical to get it accepted.}
Checking in larger binary files such as images, sound, or video affects performance.
Actions that require copying data in and out of the system slow from hundredths of a second to full seconds or minutes.
And since a \gls{DVCS} keeps every version of every file in every \gls{repository} forever, the disk space needs only increase.

This has lead to a conventional wisdom that binary files should never be stored in version control, inspiring blog posts with titles such as
"Don't ever commit binary files to Git! Or what to do if you do"~\cite{dont_ever_commit_binaries_to_version_control},
even as the modern software development practice of continuous delivery was commanding teams to "keep absolutely everything in version control."~\cite[p.33]{continuousdeliverybook}

This paper evaluates version control when storing larger binary files, with the goal of building a scalable, highly-available, distributed storage system with versioning for media files such as images, audio, and video.

