\section{Distributed Media Versioning}

\written{We developed an early prototype of such a system, which we call Distributed Media Versioning (DMV).}

\glsreset{DMV}

\gls{DMV} is our new distributed data storage platform.
The core idea is relatively simple --- store data in a Git-like \gls{DAG}, but make the following changes:

\begin{tight_enumerate}

    \item{Store data at a finer granularity than the file}

    \item{Allow nodes to store only a portion of the \gls{DAG} as a whole}

\end{tight_enumerate}

Doing so allows a data set to be replicated or sharded across many nodes according to the capacity of nodes and the needs of local users.
The focus is on data locality: tracking what data is where, presenting that information to the user, and making it easy to transfer data to other nodes as desired.
The ultimate goal is to create a new abstraction, of \emph{many devices, one data item} in varying states of synchronization.

%

\gls{DMV}'s \gls{DAG} is based on Git's, but it adds a new object type, the \gls{chunkedblob}, which represents a \gls{blob} that has been broken into several smaller chunks.
An example \gls{DMV} \gls{DAG} is shown in \autoref{dia_dmv_dag_example_three_commits},and the relationships between object types are shown in \autoref{fig:dia_new_dag}.


\begin{figure}[]
    \centering

    \begin{minipage}{.65\textwidth}
        \includegraphics[width=\textwidth]{dia_dmv_dag_example_three_commits}
        \caption{A simple DMV DAG with three commits}
        \label{dia_dmv_dag_example_three_commits}
    \end{minipage}%
    \begin{minipage}{.35\textwidth}
        \includegraphics[width=\textwidth]{dia_new_dag}
        \caption{DMV DAG Object Types}
        \label{fig:dia_new_dag}
    \end{minipage}
\end{figure}


Files are split into chunks using the rsync \gls{rollinghash} algorithm.
This splits the files into chunks by content rather than position, so that identical chunks within files (and especially different versions of the same file) will be found and stored as identical objects, regardless of their position within the file.
This way, identical chunks will be naturally de-duplicated by the \gls{DAG}, and only the changed portions of files need to be stored as new objects.

\written{DMV is different because it starts with the assumption that the repo is incomplete.
The repository itself can be distributed.}

DMV will also distribute the repository itself.
Repositories will have the option of only storing a portion of the data set or a portion of its history, in order to save space.
A DMV repository will start with the assumption that it does not hold all objects in the data set.
The goal is to allow DMV to run on devices with widely varying available resources, from servers to mobile devices.

We have written a \gls{DMV} prototype in the Rust programming language.
The current early prototype can store and retrieve data locally, but the distributed features are not yet implemented.

The \gls{DMV} prototype was developed with Rust stable versions 1.15 and 1.16 on Debian Linux 8.6 ("Jessie").
The current DMV prototype stands at \num{7592} lines of Rust code (\num{6565} excluding comments).
Source code is available at \dmvurl .
