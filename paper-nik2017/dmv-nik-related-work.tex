\section{Related Works}

The obvious related works are distributed version control systems such as Git and Mercurial.
These are the inspirations for DMV.

There are existing projects that extend Git with special handling for larger files, such as Git-annex~\cite{git_annex_homepage} and Git-media~\cite{git_media_github}.
Both store information about the larger files in the normal Git \gls{repository} and then store the files themselves in a separate location.
Git-annex files may be spread across several different remote \gls{repository} clones or data stores, and Git-annex has features for tracking the locations of annex files in different remote \glspl{repository} and moving them from one \gls{repository} to another.
These tracking and distribution features are very similar to our goals for DMV
However, Git-annex is not quite as flexible as we aim for with \gls{DMV}.
It considers the large files atomic units, and it does not break them into smaller chunks for de-duplication.

Boar~\cite{boar_homepage}, and Bup~\cite{bup_homepage} are open-source backup systems based on version control.
Bup even uses Git's data format.
They are also an inspiration for DMV.
Both use a rolling hash algorithm to break files into chunks by content.
However both are focused on a backup workflow.
Both also assume that the whole repository will fit on one filesystem, and both have limited distributed capabilities.
Apple's Time Machine~\cite{timemachine_patent} is another backup system that de-duplicates unchanged files, but it uses filesystem hardlinks rather than content-addressing.
Time Machine's functionality can be mimicked by using Rsync with the \lstinline{--link-dest} option~\cite{timemachine_foreveryunix}.

Dat~\cite{dat_homepage} is an open-source project for publishing and sharing scientific data sets for research.
IPFS~\cite{ipfs_github_main} is an open-source pro\-ject to create a global content-addressed filesystem.
Both use content addressing to keep versions of content.
Where Dat and IPFS focus on publishing on the open internet, DMV will focus on ad-hoc networks and data that may be private.

Camlistore~\cite{camlistore_homepage} and Eyo~\cite{Strauss:2011:EDP:2002181.2002216} are systems for storing personal media collections.
Both eschew traditional filesystems for databases to store various media types and their metadata.
Eyo in particular leans on the insight that in media files, the metadata is more likely to change than the image/audio/video data.
And so it separates metadata from data.
This allows efficient storage and syncing of metadata.
However, it requires that the software be able to parse many different media formats, and it requires client software to be rewritten to open the separate metadata and data streams.
This requires extra effort on the part of Eyo's developers and application developers.

Rsync~\cite{rsynctechreport} is the origin of the rolling hash algorithm that DMV and Bup use.

\perjmbinline{Mention COW filesystems}
