\section{Discussion}

\towrite{Discuss implementation details that lead to size limits}

\towrite{Mercurial: Diff during commit}

\towrite{Git: Compress during gc}

% DMV sluggishness

DMV's parabolic increase is due to the way it breaks the large file into chunks and stores objects as individual files on the filesystem.
While it is reading one large file, it is writing many small files, which incurs filesystem overhead.
So its performance characteristic for storing a large file is closer to that of storing many files (\autoref{results-num-files}).
Bup also breaks the file into many chunks, but it avoids the filesystem overhead by recombining the chunks into \glspl{packfile}.
We investigate the filesystem overhead further in \autoref{perf-tuning-exp-chapter}.

This sluggishness is due to the way DMV stores chunks of the file as individual files on the filesystem, turning the problem of storing one large file into the problem of storing many small files.
Storing many small files in this way incurs filesystem overhead, as we discovered in the results of the number-of-files experiment (\autoref{results-num-files--c1-time}), and later performed more experiments to examine in detail (\autoref{perf-tuning-exp-chapter}).

\subsection{File Quantity Limits}

Git, Mercurial, DMV, and the copy all create one file in their \glspl{objectstore} for each input file.
So to store \num{7.5} million files, they will create \num{7.5} million more, resulting in \num{15} million files on the filesystem, plus directories.
However, the \SI{197}{\gib} experiment partition has \num{13107200} total \glspl{inode}, so storing \num{15} million files is impossible.

Bup is able to store more files because it does not write a separate object file for each input file.
Bup aggregates its DAG objects into \glspl{packfile}, writing several large files instead many small files.
As such, it does not exhaust the disk's \glspl{inode}, and can continue until the experiment itself exhausts the system's \glspl{inode} when it tries to go up from \num{10} million files to the next step and run a trial with \num{25} million files.

\subsubsection{Hash-Based Directory Names Cause Disk Seeking}

\towrite{Discuss Git and DMV slowing down due to random writes}

We saw in the file-size commit times (\autoref{results-file-size--c1-time}) that DMV's time increased quadratically, and we suspected that was because it was creating many small files and incurring filesystem overhead.
This effect would explain why both Git and DMV do so poorly here while Bup would fare much better.
But why then would Mercurial and the copy also have a linear increase instead of an quadratic one?

The difference is the naming schemes of stored files.
Git and DMV name each object file according to the SHA-1 hash of the object's contents, while Mercurial, like the copy, uses the original input file's name.
This means that Git and DMV write files in a random order with respect to their names, jumping between different \gls{objectstore} subdirectories, while Mercurial and the copy can write files in the order they read them, one subdirectory at a time.
The filesystem is most likely optimized for that kind of sequential write.
