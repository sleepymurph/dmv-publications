\section{Results}

\subsection{File Size}

\subsubsection{File Size Limits: RAM, Time, Disk Space}

In the experiments, both Git and Mercurial had file size limits that were related to the size of RAM.
Mercurial refused to commit a file \SI{2}{\gib} or larger.
It exited with an error code and print an error message saying "up to 6442 MB of RAM may be required to manage this file."
This is because Mercurial stores file revisions as deltas against a base revision, so it has to do its delta calculation up front.
It loads each revision of the file into memory to do the calculations, plus it allocates memory to write the output.
As a result, Mercurial needs to be able to fit the file into memory three times over in order to commit it.
We saw that in each case, the commit was not stored, and the repository was left unchanged.
Mercurial commits are atomic.

\perjmb{would work (slowly) with larger swap then...}
Git's commit operation appeared to fail with files \SI{12}{\gib} and larger.
It exited with an error code and print an error message saying "fatal: Out of memory, malloc failed (tried to allocate 12884901889 bytes)."
However, the commit was be written to the repository, and git's \lstinline{fsck} operation reported no errors.
So the commit operation completes successfully, even though an error is reported.
\perotto{Is this A, D, or I}

With files \SI{24}{\gib} and larger, Git's \lstinline{fsck} operation itself failed.
In each case, the \lstinline{fsck} command exited with an error code and give a similar "fatal ... malloc" error.
However, the \SI{24}{\gib} file could still be checked out from the repository without error.
So we continued the trials assuming that these were also false alarms.

Git's delta compression takes place in a separate garbage collection step.
For Git, we ran the garbage collector at the end of each trial and measured repository size before and after garbage collection.
We measured total size, which included input data.
With file sizes up to and including \SI{1}{\gib}, the garbage collection resulted in a reduction in repository size from approximately three times the input data size (the input file, and two separately stored revisions) to approximately twice the input data size (the input file, and the base revision, and a delta).
At \SI{1.5}{\gib} and above, the repository size remained approximately three times the input data size after garbage collection.
This indicates that Git's delta compression also requires that the file be able to fit into disk space three times over.

The \gls{DMV} prototype was able to store a file up to \SI{64}{\gibi\byte} in size, but time became a limiting factor as file size increased.
We set an arbitrary five and a half hour timeout for commits in our experiment script.
At \SI{96}{\gib}, the DMV commit operation hit this limit and was terminated.

The largest file size committed in the trials was \SI{96}{\gib}.
This was a limitation of the experiment environment, not a limit of the systems under test.
The experiments were performed on a \SI{197}{\gib} partition.
The next trial size \SI{128}{\gib} is too large to fit two copies on the partition.
So no version control system was able to commit it, since they all save a copy of the file.

Bup was able to store a \SI{96}{\gibi\byte} file with no errors in just under two hours.
Git could also store such a large file, but one must ignore the false-alarm "fatal" errors being reported by the user interface.

These findings are summarized in \autoref{file-sizes-table}.

\begin{table}[]
    \caption{Observations as file size increases}
    \label{file-sizes-table}
    \centering
    \begin{tabular}{r l}
        Size & Observation \\
        \midrule
        \SI{1.5}{\gibi\byte} & Largest successful commit with Mercurial \\
        \SI{1.5}{\gibi\byte} & Git commit successful, but garbage collection fails to compress \\
        \SI{2}{\gibi\byte} & Mercurial commit rejected \\
        \SI{8}{\gibi\byte} & Largest successful commit with Git \\
        \SI{12}{\gibi\byte} & Git false-alarm errors begin, but commit still intact \\
        \SI{16}{\gibi\byte} & Largest successful Git fsck command \\
        \SI{24}{\gibi\byte} & Git false-alarm errors begin during fsck, but commit still intact \\
        \SI{64}{\gibi\byte} & Largest successful DMV commit \\
        \SI{96}{\gibi\byte} & DMV timeout after \SI{5.5}{\hour} \\
        \SI{96}{\gibi\byte} & Last successful commit with Bup (and Git, ignoring false-alarm errors) \\
        \SI{128}{\gibi\byte} & All fail due to size of test partition \\
    \end{tabular}
\end{table}

%

\subsubsection{Commit Times for Increasing File Sizes}

\autoref{fig:plot-file-size--c1-time} shows the wall-clock time required for the initial \gls{commit}, adding a single file of the given size to a fresh \gls{repository}.
Over all, the trend is clear and unsurprising: \gls{commit} time increases with file size.
It increases linearly for Git, Mercurial, and Bup.
DMV's commit times increase in a more parabolic fashion,
similar to how it and Git respond to increasing numbers of files.
This is because DMV breaks the large files into many smaller objects, trading the large file problem for the many file problem.


%



\subsection{Number of Files}

\subsubsection{File Quantity Limits: Inodes, Disk Space}

Git, Mercurial, DMV, and the copy operation all failed when trying to store \num{7.5} million files or more, reporting that the disk was full.
However, the disk was not actually out of space.
It was out of \glspl{inode}.

Unix filesystems, ext4 included, store file and directory metadata in a data structure called an \gls{inode}, which reside in a fixed-length table~\cite{unix_timesharing_system}.
When all of the \glspl{inode} in the table are allocated, the filesystem cannot store any more files or directories.
The number of inodes is tunable at filesystem creation by passing a bytes-per-inode parameter (\lstinline{-i}) to \lstinline{mke2fs}, but our experiment partitions used the default setting.
The \SI{197}{\gib} partitions had approximately \num{13} million inodes.

All systems tested except for Bup store a new copy of the input data, with one stored file per input file.
So committing \num{7.5} million input files would create an additional \num{7.5} million stored files, for a total of \num{15} million inodes, \num{2} million more than the \num{13} million on the filesystem.

Bup avoided the \gls{inode} limit because it writes directly into Git's pack file format.
It aggregates objects and conserves inodes.
Bup trials could continue until the input data itself exhausted the system's \glspl{inode} attempting to generate \num{25} million input files.

Bup also made more efficient use of disk space.
The input files were \SI{1}{\kib}, while the filesystem's block size was the default \SI{4}{\kib}.
Therefore, the input data set used four times the amount of disk space as it needed.
Because Git, Mercurial, DMV, and the copy control all made one new file for each input file, the commit used another \SI{4}{\kib} for each file, for a total of approximately eight times the disk space used.
The total size measured in the Bup experiments, including input data, was \num{5.374} times the size of the input data at \num{5} million files.
And since the input files themselves account for just over \num{4} times the theoretical size, we can see that Bup is storing the data in a form that is much closer to its theoretical size, taking just under \num{1.374} times the space.

%


\subsubsection{Commit Times for Increasing Numbers of Files}

\autoref{fig:plot-num-files--c1-time} shows the time required for the initial \gls{commit}, storing all files into a fresh empty \gls{repository}.
Here we see the commit times for Git and DMV increasing quadratically with the number of files, while Mercurial, Bup, and the copy increase linearly.

\begin{figure}
    \centering
    \begin{minipage}{.5\textwidth}

        \caption{Wall-clock time to commit one large file to a fresh repository}
        \label{fig:plot-file-size--c1-time}
        \centering
        \includegraphics[width=\textwidth]{plot-file-size--c1-time}

    \end{minipage}%
    \begin{minipage}{.5\textwidth}

        \caption{Wall-clock time to commit many 1KiB files to a fresh repository}
        \label{fig:plot-num-files--c1-time}
        \centering
        \includegraphics[width=\textwidth]{plot-num-files--c1-time}

    \end{minipage}
\end{figure}
