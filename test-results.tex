\documentclass[a4paper]{article}
% vim: set ts=2 sts=2 sw=2 :

\usepackage{graphicx}
\usepackage[utf8]{inputenc}

% Use parskip to use a blank space between paragraphs instead of an indent.
\usepackage[parfill]{parskip}

% Source code listings
\usepackage{listings}
\lstset{basicstyle=\ttfamily\footnotesize,breaklines=false}

\usepackage{url}
% To actually link URLs in the PDF.
%
% The hyperref docs recommend declaring it after the other \usepackage
% declarations, because it has to redefine several commands to work properly,
% and other later redefinitions might interfere.
% Remove the hidelinks parameter to get (ugly) visual highlights around the
% links.
\usepackage[hidelinks]{hyperref}

% Create a short macro for newly-defined key terms.
\newcommand{\newterm}{\textit}


\begin{document}

\title{Performance Testing Version Control Systems}
\author{Michael Murphy}
\date{October 2016}
\maketitle

\section{Test Results}

\subsection{Performance as file size increases}

We performed a test where a single file of a given size was added to a version
control repository. We increased the size until the version control system could
no longer process the file without error. The basic procedure was as follows:

\begin{enumerate}
    \setlength{\itemsep}{0pt}
    \setlength{\parskip}{0pt}
    \setlength{\parsep}{0pt}
    \item Initialize an empty repository
    \item Generate a test file of the given size from random binary data
    \item Commit the file
    \item Overwrite a small part of the file (1/1024th of the data)
    \item Commit the file again
    \item Run the garbage collection / compaction algorithm (Git only)
\end{enumerate}

Tests were performed on identical machines with 8GiB of RAM and about 200GiB of
free hard disk space.

Figure \ref{fig:increasing-file-size--time-comparison} shows commit performance
as file size increases, and figure
\ref{fig:increasing-file-size--space-comparison} shows disk space performance.

\begin{figure}[h]
  \caption{Increasing file size: commit time comparison}
  \label{fig:increasing-file-size--time-comparison}
  \centering
    \includegraphics[]{increasing-file-size--time-comparison}
\end{figure}

\begin{figure}[h]
  \caption{Increasing file size: repository size comparison}
  \label{fig:increasing-file-size--space-comparison}
  \centering
    \includegraphics[]{increasing-file-size--space-comparison}
\end{figure}

Observed limitations:

\begin{itemize}

    \item Git initially stores a full copy of every revision and then has a
        separate garbage collection phase to compact stored data in the
        repository. So for a single commit, the operation uses 2x the disk space
        as the size of the file: one for the file itself, and one for the copy
        in the repository. And for a single commit and update, the operation
        requires 3x the disk space: one for the file itself, one for the initial
        copy in the repository, and one for the updated copy.

        Disk space usage reduces after garbage collection (in this case to
        2.001x), because git compacts and de-duplicates the data in its
        repository. However, it cannot be reliably used unless there is enough
        space to store the uncompressed version first. Therefore Git starts to
        fail the test with files about 1/3 of the size of the available disk
        space.

    \item When the file is too large to fit into RAM, Git's commit operation
        prints an error message and exits with an error code, but the operation
        still completes successfully.

    \item Starting at just under 1GiB, the Git garbage collection seems to fail
        silently. No errors are reported, but on the graph one can see that the
        garbage-collected disk space usage jumps from 2x to 3x, indicating that
        the garbage collection phase is not doing any compacting. This might
        have to do with RAM requirements, similar to Mercurial's commit
        operation.

    \item Mercurial stores only deltas of files. In a sense, it is doing Git's
        garbage collection phase during each commit. This means more efficient
        disk usage. But it places a strong limitation on file size: Mercurial
        commits fail unless it has 3x as much RAM available as the size of the
        file. We suspect this has to do with the way Mercurial calculates. On
        the 8GiB test machines, Mercurial could only store files up to 2GiB in
        size.

\end{itemize}

Performance observations:

\begin{itemize}

    \item After some initial overhead, performance increases linearly with size.
        This is to be expected, since the operations are IO bound, copying all
        data to the repository.

    \item Times for the Mercurial update are faster than for Git's update,
        because Mercurial's archive format only saves deltas to the file.

    \item Mercurial has more initial time overhead, this is probably due to the
        fact that it is written in Python and requires starting the Python
        interpreter each time. This overhead is only 50 or 100ms, and quickly
        becomes insignificant compared to the IO operation time.

    \item Mercurial has less initial space overhead than Git.

        \begin{itemize}
            \setlength{\itemsep}{0pt}
            \setlength{\parskip}{0pt}
            \setlength{\parsep}{0pt}
            \item Minimum Mercurial repository size: 80KiB.
            \item Minimum Git repository size: 16KiB.
        \end{itemize}

        This can be seen in the disk space graph in the way the Mercurial usage
        converges towards 2x faster than Git usage does. But again, this quickly
        becomes insignificant compared to the size of the file.

\end{itemize}

\subsection{Performance as file count increases}

We performed a test where increasingly large sets of files were committed to the
different version control repositories. The procedure was as follows:

\begin{enumerate}
    \setlength{\itemsep}{0pt}
    \setlength{\parskip}{0pt}
    \setlength{\parsep}{0pt}
    \item Initialize an empty repository
    \item Generate a test file set of the given size. Each file is 1KiB of
        random binary data
    \item Commit the file set
    \item Check the status of the files
    \item Overwrite a small part of some of the files (1/1024th of the data in
        1/16 files)
    \item Check the status of the files again
    \item Commit the file set again
\end{enumerate}

Figure \ref{fig:increasing-file-count--commit-time-comparison} shows the time
required for the different commit operations. Figure
\ref{fig:increasing-file-count--status-time-comparison} shows the time required
for the different status operations. And Figure
\ref{fig:increasing-file-count--space-comparison} shows the total disk space
used after each commit operation.

Unlike the test with a single large file, the numerous small files did not
quickly hit error-causing disk space or RAM limitations. The version control
systems happily crunched the data as test times grew into hours.

\begin{figure}[h]
  \caption{Increasing file count: commit time comparison}
  \label{fig:increasing-file-count--commit-time-comparison}
  \centering
    \includegraphics[]{increasing-file-count--commit-time-comparison}
\end{figure}

\begin{figure}[h]
  \caption{Increasing file count: status check time comparison}
  \label{fig:increasing-file-count--status-time-comparison}
  \centering
    \includegraphics[]{increasing-file-count--status-time-comparison}
\end{figure}

\begin{figure}[h]
  \caption{Increasing file count: repository size comparison}
  \label{fig:increasing-file-count--space-comparison}
  \centering
    \includegraphics[]{increasing-file-count--space-comparison}
\end{figure}

Observations:

\begin{itemize}

    \item Git in general is faster then Hg up to about 500k files.

    \item 512Ki files is where Hg and Git are neck and neck. At 768Ki and over,
        Hg is faster.

    \item Git's update times hit a peak at 1M and then starts getting faster.
        Weird. Maybe at that point there are so many random files that some of
        them are converging?

\item Status check times are comparable between Git and Hg. Git is faster up to
    1Mi files, when Hg becomes faster than it was before and the two are about
    neck and neck.

\item Git's update status is faster than initial status up to about 64Ki, then
    it becomes slower.

\item Hg update status is consistently slower than initial status, often by
    about 2-3x.

\item Both Git and Hg converge to a little over 8x the space required. This
    probably has more to do with the filesystem block size than anything else.
    If the file system is 4k blocks, the files are 1k, and there are two copies.
    Then that's 2 blocks and 8k per file.

\item Mercurial converges towards that faster though. I'm guessing because of
    lower repo overhead, and also because Git is going to be making 4k tree
    objects for all the subdirectories too.

\end{itemize}

\end{document}
