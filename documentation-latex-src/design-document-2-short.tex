\documentclass[a4paper]{article}
% vim: set ts=2 sts=2 sw=2 :

\usepackage{graphicx}
\usepackage[utf8]{inputenc}

% Use a blank space between paragraphs instead of an indent.
\usepackage[parfill]{parskip}

% Source code listings
\usepackage{listings}
\lstset{basicstyle=\ttfamily\footnotesize,breaklines=false}

\usepackage{url}
% To actually link URLs in the PDF.
%
% The hyperref docs recommend declaring it after the other \usepackage
% declarations, because it has to redefine several commands to work properly,
% and other later redefinitions might interfere.
% Remove the hidelinks parameter to get (ugly) visual highlights around the
% links.
\usepackage[hidelinks]{hyperref}

% Create a short macro for newly-defined key terms.
\newcommand{\newterm}{\textit}


\begin{document}

\title{Master's Thesis Design Document \\ (Terse Version)}
\author{Michael Murphy}
\maketitle

\section{Idea}\label{idea}

A distributed data store that makes location and availability explicit, rather
than trying to hide or abstract them away.

The system will be designed for high-latency or intermittent links (minutes to
weeks), by assuming that network partitions are the norm. It will make
availability explicit by providing location and latency hints and estimates that
client applications can use to schedule processing. It will allow local writes
at all times, and will handle inconsistency by storing competing versions of
data until a client that is familiar with the data model can merge them.

The system should work with a wide range of file sizes (kilobytes to gigabytes),
focusing on media storage and synchronization (binary blobs, 1 MiB to 4 GiB in
size).


\section{Architecture}\label{architecture}

Data will be distributed across several \newterm{stores}, which will each hold a
portion of the data set.

The data store network will be decentralized and unstructured. Stores can be
connected to neighbors in an ad-hoc manner, with a user- or application-defined
topography (one to hundreds of neighbors).

(Possible extension: a gossip protocol to relay information about stores that
are not direct neighbors.)

% TODO: arch diagram

Each store will be autonomous and able to perform processing on any data that it
has a local copy of. Data not currently available locally can be listed and
requested. The listing will give hints about the estimated time of availability,
calculated based on latency, bandwidth, and file size. This metadata will
naturally fall out of date between syncs with neighbors. The age of that data
can be part of the hint.

We are envisioning a filesystem with an \lstinline{ls} command that lists these
hints as part of its output:

\begin{lstlisting}
    -rw-r--r-- 1 user user  121306 Oct 21 18:28 local   filex
    -rw-r--r-- 1 user user   25475 Oct 21 17:52 100ms   filey
    -rw-r--r-- 1 user user   32031 Oct 21 17:52 20min   filez
    -rw-r--r-- 1 user user   74968 Oct 18 17:12 missing filexx
    -rw-r--r-- 1 user user   83977 Sep 22 21:23 unknown fileyy
\end{lstlisting}

The system should detect storage errors and never lose data inadvertently. Local
data stores should detect corruption, and replication between stores should
provide data safety. However, data can be explicitly removed from the data set,
or it can be removed from the local store, relying on remote stores to keep
replicas.


\section{Design}\label{design}

The data set will be a normal hierarchical filesystem, and it will be presented
as a normal (if virtual) filesystem. We do not want to reinvent the filesystem,
and we do not to have to modify local applications to work with the data.

The storage will be modeled on version control systems, specifically Git and
it's directed acyclic graph data structure. There will be a content-addressable
blob store to de-duplicate and store data, and a working tree where the files
can be read and written by normal applications.

(Possible optimization: the working directory can be a virtual filesystem that
is a view into the object database.)

Files over a certain threshold will be broken into smaller chunks in the blob
store.

Like a version control system, history of the data set will be kept, and the
history can diverge into branches that can be merged later. Indeed, like Git,
each individual store is also a branch. But unlike Git, not all blobs are
required to be present on all systems. Some blobs may not be available on a
given store, and history may be deliberately pruned to save space. Algorithms
and client applications must work with the blobs they have locally and the hints
about the availability of remote blobs.


\section{Implementation}\label{implementation}

% TODO: DAG diagram



% \nocite{*}  % Print all references even if they're not used
% \bibliographystyle{plain}
% \bibliography{research}

\end{document}
