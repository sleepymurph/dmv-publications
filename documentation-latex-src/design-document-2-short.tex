\documentclass[a4paper]{article}
% vim: set ts=2 sts=2 sw=2 :

\usepackage{graphicx}
\usepackage[utf8]{inputenc}

% Use a blank space between paragraphs instead of an indent.
\usepackage[parfill]{parskip}

% Source code listings
\usepackage{listings}
\lstset{basicstyle=\ttfamily\footnotesize,breaklines=false}

\usepackage{url}
% To actually link URLs in the PDF.
%
% The hyperref docs recommend declaring it after the other \usepackage
% declarations, because it has to redefine several commands to work properly,
% and other later redefinitions might interfere.
% Remove the hidelinks parameter to get (ugly) visual highlights around the
% links.
\usepackage[hidelinks]{hyperref}

% Create a short macro for newly-defined key terms.
\newcommand{\newterm}{\textit}


\begin{document}

\title{Master's Thesis Design Document \\ (Terse Version)}
\author{Michael Murphy}
\maketitle

\section{Idea}\label{idea}

A low-level distributed data storage platform that makes data location and
availability explicit, rather than trying to hide or abstract them away.

The target use for this system is to spread a data set across many compute
nodes, and to allow the sharding and replication of the data set to be
customizable according to the location and availability of the nodes, and what
data is needed where. The ultimate goal is to track the scattered data as a
coherent data whole, and to allow the end user or client application to
visualize and manage what data is stored on what nodes.

The system aims to be a low-level service, providing information about the data
and tools to manage it, but ultimately leaving decisions to the end user or
client application.

We assume that the data set and all nodes in the system are controlled by the
user.

The target data sets for this system include collections of media files (blobs
ranging in size from hundreds of kilobytes to several gigabytes) numbering in
the hundreds to millions. The system will not attempt to understand the
internals of different media formats and will treat all files as opaque blobs.

The system will be designed for high availability in the face of connections
that are intermittent, high-latency, expensive, or otherwise restricted. At each
node, the portion of data that is stored on that node will be available for
reading and writing by local applications. Metadata about the non-local portions
of the data set will be available as well, along with hints about the location
of that data. Clients can use that information to schedule transfers and
updates. Consistency will be resolved during updates. Conflicting versions will
be presented to the client application to resolve.


\section{Architecture}\label{architecture}

Data will be distributed across several \newterm{stores}, which will each hold a
portion of the data set. Stores can be diverse, from mobile devices to storage
clusters, and each will choose which data to hold according to its capacity and
processing needs.

The data store network will be decentralized and unstructured (Figure
\ref{fig:architecture}). Stores can be connected to neighbors in an ad-hoc
manner, with a user- or application-defined topography. The number of neighbors
will be limited by the capabilities of the store's device. We are aiming for one
neighbor to tens, or perhaps hundreds of neighbors.

(Possible extension: a gossip protocol to relay information about stores that
are not direct neighbors.)

\begin{figure}[h]
  \caption{Stores in an ad-hoc network}
  \label{fig:architecture}
  \centering
    \includegraphics[width=0.95\textwidth]{architecture}
\end{figure}

Each store will be autonomous and able to perform processing on any data that it
has a local copy of. Data not currently available locally can be listed and
requested. The listing will give hints about the estimated time of availability,
calculated based on latency, bandwidth, and file size. This metadata will
naturally fall out of date between syncs with neighbors. The age of that data
can be part of the hint.

We are envisioning a filesystem with an \lstinline{ls} command that lists these
hints as part of its output:

\begin{lstlisting}
    -rw-r--r-- 1 user user  121306 Oct 21 18:28 local   filex
    -rw-r--r-- 1 user user   25475 Oct 21 17:52 100ms   filey
    -rw-r--r-- 1 user user   32031 Oct 21 17:52 20min   filez
    -rw-r--r-- 1 user user   74968 Oct 18 17:12 missing filexx
    -rw-r--r-- 1 user user   83977 Sep 22 21:23 unknown fileyy
\end{lstlisting}

The system should detect storage errors and never lose data inadvertently. Local
data stores should detect corruption, and replication between stores should
provide data safety. However, data can be explicitly removed from the data set,
or it can be removed from the local store, relying on remote stores to keep
replicas.


\section{Design}\label{design}

The data set will be a normal hierarchical filesystem, and it will be presented
as a normal filesystem. We do not want to reinvent the filesystem, and we do not
to have to modify local applications to work with the data.

The storage will be modeled on version control systems, specifically Git and
it's directed acyclic graph data structure (DAG). There will be a
content-addressable blob store to de-duplicate and store data in a DAG
structure, and a working tree where the files can be read and written by normal
applications.

(Possible optimization: the working directory can be a virtual filesystem that
is a view into the object database.)

Like in a version control system, history of the data set will be kept. Ordering
of updates will be determined by the child-parent relationship of commits in the
DAG. The history can diverge into branches that can be merged later, and each
individual store is also naturally a branch. Also like in version control,
updates must be explicitly committed to the store, and fetches and merges from
neighbors are explicitly initiated.

(Possible add-on feature: daemons which auto-commit, auto-fetch, and auto-merge
if possible.)

Individual stores are not required to store all blobs in the DAG. Some blobs may
not be available on a given store, and history may be deliberately pruned to
save space. Algorithms and client applications must work with the blobs they
have locally and the hints about the availability of remote blobs.


\section{Implementation}\label{implementation}

% TODO: DAG diagram

Files over a certain threshold will be broken into smaller chunks in the blob
store, using a content-based splitting technique such as Rabin fingerprinting.


% \nocite{*}  % Print all references even if they're not used
% \bibliographystyle{plain}
% \bibliography{research}

\end{document}
