\documentclass[a4paper]{article}
% vim: set ts=2 sts=2 sw=2 :

\usepackage{graphicx}
\usepackage[utf8]{inputenc}

% Use a blank space between paragraphs instead of an indent.
\usepackage[parfill]{parskip}

% Source code listings
\usepackage{listings}
\lstset{basicstyle=\ttfamily\footnotesize,breaklines=false}

\usepackage{url}
% To actually link URLs in the PDF.
%
% The hyperref docs recommend declaring it after the other \usepackage
% declarations, because it has to redefine several commands to work properly,
% and other later redefinitions might interfere.
% Remove the hidelinks parameter to get (ugly) visual highlights around the
% links.
\usepackage[hidelinks]{hyperref}

% Create a short macro for newly-defined key terms.
\newcommand{\newterm}{\textit}


\begin{document}

\title{Version Control for Large Files\\
    Specification}
\author{Michael Murphy}
\date{August 2016}
\maketitle

\section{Idea}\label{idea}

\subsection{Main idea: Distributed version control for large data
sets}\label{main-idea-distributed-version-control-for-large-data-sets}

Main principles:

\begin{itemize}
\item
  Data must never be lost accidentally.
\item
  However, history may be deliberately truncated to save space, and
  sensitive data may be deliberately redacted.
\item
  Data integrity must be verifiable: The system must be able to detect
  errors and, if possible, repair them.
\item
  Changes to the dataset should be tracked, versions should be
  explicitly labeled, and history should be kept.
\item
  Like with distributed version control, updates can be made
  independently and merged later. Different sites can have different
  versions. Updates (commits) and synchronization are deliberate,
  explicit, and manual.
\end{itemize}

Important assumptions:

\begin{itemize}
\item
  Contact between repositories is intermittent. Repositories may be on
  removable drives or mobile devices. Updates may require physical
  connection and reconnection. It is important to track the state of
  other repositories, so that the user can know what needs to be
  synchronized.
\item
  Assume all actors are honest for now. No malicious components.
\item
  However, components can and will fail. The system must discover and
  recover from errors (checksums, replication).
\end{itemize}

Scale:

\begin{itemize}
\item
  Must work with many terabytes of data.
\item
  Must be able to spread data across several stores.
\item
  Must handle a wide range of file sizes (text, images, audio, video).
\item
  Must handle file counts in the millions.
\end{itemize}

\subsection{Main inspiration: Git}\label{main-inspiration-git}

Can we take the elegance of Git's block chain and generalize and
distribute it so that it works well with large files, and so that all
the data does not have to be present in every repository?

Differences from git:

\begin{itemize}
\item
  Partial repositories: Individual repositories need not store entire
  history. Several repositories can work together to spread the data
  across many machines.
\item
  Partial checkouts: Working directories need not check out entire data
  set.
\item
  Support for large files: Large files can be split into chunks and
  spread across repositories.
\item
  Repositories work more closely together to form a whole:

  \begin{itemize}
  \item
    Repositories should be aware of each other and what pieces of data
    each one stores.
  \item
    Should know where to find data when needed.
  \item
    Should be able to visualize how complete data set storage is and how
    well data is replicated to protect against data loss.
  \end{itemize}
\end{itemize}

Other considerations:

\begin{itemize}
\item
  Stored data set should be a normal file tree, with history, like any
  normal version control system.

  \begin{itemize}
  \item
    We do not want to reinvent the filesystem.
  \item
    We do not want to deal with media metadata and categorization. That
    is a whole other problem.
  \item
    This is meant for long-term storage. Classic files are the best
    chance for that. New data schemas are also a whole other problem.
  \item
    However: it may be possible to generalize, and allow flexible
    aggregation besides trees.
  \end{itemize}
\item
  However, knowledge of media formats might be used for
  behind-the-scenes optimization such as more efficient compression.
  E.g. recognizing that only tag data has changed in an audio file.
\end{itemize}

\section{Architecture}\label{architecture}

\subsection{Core data structure: DAG}\label{core-data-structure-dag}

\textbf{Start with data structure. If the data structure is right, the
rest should follow.}

Again, the main inspiration is Git.

\begin{itemize}
\item
  Data structure will be based on Git's DAG: immutable blobs, trees, and commits
  that are stored in a content-addressable object database, and referred to by
  cryptographic hash.
\item
  The immutability, cryptographic hashing, and DAG data structure make it easy
  to synchronize between repositories and check data integrity.
\item
  Like Git, current state of local branches, and known state of remote
  repositories will be pointers to commits in the DAG.
\end{itemize}

\subsection{Like version control, interactions based on files and
check-ins}\label{like-version-control-interactions-based-on-files-and-check-ins}

\begin{itemize}
\item
  Like Git, there will be an object database that stores the DAG, and a
  working directory where files are checked out for normal use and
  editing.
\item
  We prefer a version-control style interaction, checking files in and
  out explicitly to the filesystem where applications can work with them
  as normal files. Again, we do not want to reinvent the filesystem.

  However, to save space when working with large collections, perhaps a FUSE
  filesystem could present a virtual working directory that is actually a
  copy-on-write snapshot of files in the object database.
\item
  Like Git, no repository is inherently more important or more central than any
  other. Importance given to a certain repository or specific push-pull flows
  will be the result of the user's workflow, not of the underlying system.
\end{itemize}


\begin{figure}[h!]
  \caption{Distributed Version Control}
  \label{fig:object-db-and-working-directory}
  \centering
    \includegraphics[width=0.5\textwidth]{object-db-and-working-directory-dot}
\end{figure}



\subsection{Unlike Git, repositories will not be required to store
all
objects}\label{unlike-git-repositories-will-not-be-required-to-store-all-objects}

\begin{itemize}
\item
  Commits and trees by themselves carry information about the structure
  and history of the data, the metadata of the repository. They are a
  kind of ``backbone'' that carries the actual data.
\item
  We could have all repositories keep this ``backbone,'' but allow them
  to selectively store only certain blobs. This will allow repositories
  to balance space vs completeness of history.

  \begin{itemize}
  \item
    A full archive could store all history, just as in Git.
  \item
    Several partial archives could work together to store the full history.

    Can configure which repository holds which data according to storage size
    and network topography. Old infrequently-accessed versions could be kept on
    larger, slower data stores.
  \item
    A shallow repository could store only a few recent versions, to be
    compared against the working directory or to be restored to correct
    mistakes. However, it would still have the full ``backbone,'' so it
    would know what blobs would be needed to checkout different states.
  \item
    A ``backbone-only'' repository could store no blobs, just the
    working directory and the ``backbone.'' This would allow the working
    directory to detect changes, and it could create new commits, only
    storing new blobs until they were pushed.
  \item
    A repository could also focus on a particular subtree, storing blobs
    for its entire history, but none of the blobs outside. This would
    allow detailed work on one part of the larger data set.
  \item
    Which blobs to keep could be configurable by rules that work along
    dimensions of time and parts of the tree.

\begin{verbatim}
/       last 1 versions
/foo    last 5 versions
/bar    all versions
/baz    no versions
\end{verbatim}
  \item
    We could also provide tools to recommend which blobs to store based
    on usage frequency, available storage space, and repository
    availability.
  \item
    Perhaps it is not even necessary to store the full backbone. The
    backbone will be tiny compared to the whole data set, but for large
    (millions of files) or long-lived data sets, the full backbone could
    be a burden on small, focused repositories.
  \end{itemize}
\item
  Trees and commits may hold more information than in Git.

  \begin{itemize}
  \item
    Objects could include a measure of the cumulative size of all the objects
    they refer to, so that repositories could make decisions about space
    trade-offs, and choosing to drop unneeded blobs to save space.
  \end{itemize}

\item
  Remote pointers will hold more information than with Git.

  \begin{itemize}
  \item
    Because we cannot assume that every repository has all objects,
    remote pointers must also keep metadata on which blobs are available
    at which repository. So that it knows where to look if needed.
  \item
    This availability data will be used to gather health metrics about
    what parts of history and hierarchy are safely replicated over many
    stores, and which are in danger of being lost.
  \end{itemize}
\item
  All algorithms will have to be written around the idea that data might
  not be available immediately, or at all.

  \begin{itemize}
  \item
    Repositories and working directories will work with what is
    available locally, and what is available locally will be chosen by
    the user based on what they need to work on now.
  \item
    When those needs change, it will be easy to push and pull blobs to
    and from other repositories.
  \item
    However, if a blob is lost, it is lost. This should never happen by
    accident, but it may happen deliberately by dropping old history to
    save space, or to deliberately expunge sensitive blobs from the
    records. If the algorithms can deal with missing objects locally,
    then they should naturally also be able to handle objects that are
    missing permanently.
  \end{itemize}
\end{itemize}


\subsection{Possible workflows}\label{possible-workflows}

\subsubsection{Personal Workflow}

  \begin{figure}[h]
    \caption{Personal Workflow}
    \label{fig:workflow-personal}
    \centering
      \includegraphics[width=.75\textwidth]{workflow-personal-dot}
  \end{figure}

  \begin{itemize}
  \item
    Repository on removable drive stores current version of whole data
    set, and historical versions as far back as space will allow.
  \item
    Second removable drive repository configured the same way for
    redundancy.
  \item
    Laptop has several partial repositories + working directories:

    \begin{itemize}
    \item
      Full history of \texttt{/Documents}
    \item
      A year or so of \texttt{/Pictures}
    \item
      Recent history of \texttt{/Music}
    \item
      No history and selected \texttt{/Videos}
    \end{itemize}
  \item
    Phone has several thin repositories + working directories:

    \begin{itemize}
    \item
      Selected \texttt{/Music} checked out with no history. Just used to
      sync.
    \item
      Selected \texttt{/Pictures} are checked out with no history, to be
      displayed on the phone.
    \item
      A \texttt{/Pictures/new} directory is checked out as the phone's
      new directory in which to put photos as they're taken. It stores
      only history that has not been synced. A new state is pushed to
      the laptop. The user categorizes the photos on the laptop, commits
      the new state, and pushes it to the phone. The selected photos
      show up in the categorized areas, and the \texttt{new} directory
      is emptied.
    \end{itemize}
  \end{itemize}

\subsubsection{Corporate/Scientific Workflow}

  \begin{figure}[h]
    \caption{Corporate Workflow}
    \label{fig:workflow-corporate}
    \centering
      \includegraphics[width=.95\textwidth]{workflow-corporate-dot}
  \end{figure}

  \begin{itemize}
  \item
    Main repository uses a DHT as a backing store, keeps all history.
  \item
    Users check out pieces of the data set as needed, work with it, and
    push their changes back.
  \end{itemize}

\subsubsection{Remote Sensors Workflow}

  \begin{figure}[h]
    \caption{Remote Sensor Workflow}
    \label{fig:workflow-sensors}
    \centering
      \includegraphics[width=.95\textwidth]{workflow-sensors-dot}
  \end{figure}

  \begin{itemize}
  \item
    Archive on computers in office stores all data
  \item
    A directory in the hierarchy is designated for new data from each sensor
  \item
    The sensor has a thin repository + working directory for just its own new
    data directory. It commits new data.
  \item
    A courier has a thin repository on their phone, holding just the new data
    directories for all sensors. The courier visits a sensor and connects to it,
    pulling in the new data. They visit another sensor and pulls in its data
    too.
  \item
    The courier gets back to the office and syncs with the main archive.
  \item
    The archive now has a state with new data from all visited sensors. A
    process moves the new data to a permanent directory, and commits that new
    state.
  \item
    The courier syncs the phone again, this clears the space on their phone.
  \item
    When they visit the sensor again, it syncs and merges, deleting the data
    that was stored safely, and creates a new state with just new data.
  \end{itemize}


\section{Design}\label{design}

\subsection{Modified Git DAG}

\begin{figure}[h!]
    \caption{Modified Git DAG}
    \label{fig:new-dag}
    \centering
        \includegraphics[width=0.5\textwidth]{new-dag-dot}
\end{figure}

We start with the Git DAG and modify it (Figure \ref{fig:new-dag}) to
accommodate our desired features.

\begin{itemize}
    \item
        Unlike Git, the repository is not required to store all objects in the
        DAG. A repository needs only to include the bare minimum of objects to
        record the state of its references. These include a reference to the
        current commit, that commit object, and all of that commit's trees.
        Large blob indexes should also be included. These required objects are
        shaded grey in Figure \ref{fig:new-dag}. Those connections that can be
        left dangling by not storing the referenced object are shown with dashes
        in Figure \ref{fig:new-dag}.
    \item
        Large blobs in the object database can be broken into \newterm{chunks}
        to make them easier to store, sync, and transfer. We introduce a new
        \newterm{large blob index} object type to point to the chunks that make
        up the larger blob (shown light grey in Figure \ref{fig:new-dag}.
        Chunks themselves are just blobs.
\end{itemize}


\section{Implementation}\label{implementation}

\begin{itemize}
  \item
    Object database storage should be pluggable. Flat files by default, but
    should be able to use a DHT as a large highly-available object store.
  \item
    Should be able to sync with a phone, either with an on-phone app, or via USB
    mount of filesystem.
\end{itemize}

% \nocite{*}  % Print all references even if they're not used
% \bibliographystyle{plain}
% \bibliography{specification}

\end{document}
